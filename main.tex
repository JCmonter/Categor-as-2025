\documentclass{comunicaciones}

\usepackage[utopia,sfscaled]{mathdesign}
\usepackage[scaled]{helvet}

\usepackage[utf8]{inputenc} %Por favor guardar los archivos en unicode
\usepackage[T1]{fontenc}


\usepackage{microtype}
\usepackage[spanish,mexico,es-noindentfirst]{babel}

\usepackage[paperwidth=170mm,paperheight=230 mm,total={125mm,170mm},top=29mm,left=23mm,includeheadfoot]{geometry}

\usepackage{lastpage}
\newcounter{FirstPage}
\newcommand{\Primera}[1]{\setcounter{FirstPage}{#1}}

\usepackage{mathtools}

\usepackage{enumerate}
\usepackage{float}

%%%%%%%%%%%%%%%%%%%%%%%%%%%%%%%%%%%%%%%%%%%%%%%%%%%%%%%%%%%%%%%%%%%%%%%%%
\renewcommand{\abstractname}{Resumen}
\renewcommand{\keywordsname}{Palabras Claves}
\renewcommand{\figurename}{Figura}
\renewcommand{\tablename}{Tabla}
\renewcommand{\refname}{Bibliografía}
%%%%%%%%%%%%%%%%%%%%%%%%%%%%%%%%%%%%%%%%%%%%%%%%%%%%%%%%%%%%%%%%%%%%%%%%%

\usepackage[mathscr]{eucal}
\usepackage{enumerate}
\usepackage{times}
\usepackage{tikz}
\usepackage{graphicx}
\usepackage{tikz-cd}\usetikzlibrary{decorations.pathmorphing}
%\usepackage{amsmath,amssymb,latexsym,amscd}   
\usetikzlibrary{babel}
\usepackage{hyperref}
\hypersetup{colorlinks=true,linkcolor=blue,citecolor=brown,linktocpage=true,pagebackref=true,hyperindex=true}
\usepackage{amsthm}
\usepackage{graphicx}
\usepackage[all,cmtip]{xy}
\usepackage{fancyhdr}
\usepackage{mathalfa}
\usepackage{mathrsfs}
\usepackage[sans]{dsfont}
\usepackage{upgreek}

\newcommand{\Frm}{\mathrm{Frm}}
\newcommand{\Pos}{\mathrm{Pos}}
\newcommand{\Dpos}{\mathrm{Dpos}}
\newcommand{\cbd}{\mathrm{cbd}}
\newcommand{\Cbd}{\mathrm{Cbd}}
\newcommand{\CBD}{\mathrm{CBD}}
\newcommand{\Obj}{\mathrm{Obj}}
\newcommand{\Hom}{\mathrm{Hom}}
\newcommand{\Loc}{\mathrm{Loc}}
\newcommand{\CBA}{\mathrm{CBA}}
\newcommand{\Ord}{\mathrm{Ord}}
\newcommand{\Top}{\mathrm{Top}}
\newcommand{\id}{\mathrm{id}}
\newcommand{\Id}{\mathrm{Id}}
\newcommand{\ID}{\mathrm{ID}}
\newcommand{\tp}{\mathrm{tp}}
\newcommand{\Tp}{\mathrm{Tp}}
\newcommand{\TP}{\mathrm{TP}}
\newcommand{\op}{\mathrm{op}}
\newcommand{\pt}{\mathrm{pt}}
\newcommand{\cl}{\mathrm{cl}}
\newcommand{\inte}{\mathrm{int}}
\newcommand{\ob}{\mathrm{ob}}
\newcommand{\ins}{\mathrm{ins}}
\newcommand{\coht}{\mathrm{coht}}
\newcommand{\dep}{\mathrm{dp}}
\newcommand{\sob}{\mathrm{sob}}
\newcommand{\Sob}{\mathrm{Sob}}
\newcommand{\Haus}{\mathrm{Haus}}
\newcommand{\DLat}{\mathrm{DLat}}


\theoremstyle{plain}

\newtheorem*{thm*}{Teorema}
\newtheorem{thm}{\protect\theoremname}[section]
  \theoremstyle{remark}
  \newtheorem{obs}[thm]{\protect\remarkname}
  \theoremstyle{remark}
  \newtheorem{ej}[thm]{\protect\examplename}
    \theoremstyle{plain}
  \newtheorem{subej}[thm]{\protect\subexamplename}
  \theoremstyle{plain}
  \newtheorem{cor}[thm]{\protect\corollaryname}
  \theoremstyle{plain}
  \newtheorem{lem}[thm]{\protect\lemmaname}
  \theoremstyle{plain}
  \newtheorem{prop}[thm]{\protect\propositionname}
    \theoremstyle{definition}
\newtheorem{dfn}[thm]{\protect\definitionname}
\theoremstyle{plain}
\newtheorem*{dfn*}{Definición}
% \theoremstyle{definition}
% \newtheorem{algorithm}[theorem]{Algoritmo}
% \newtheorem{axiom}[theorem]{Suposición}
% \newtheorem{case}[theorem]{Caso}
% \newtheorem{claim}[theorem]{Ayuda}
% \newtheorem{conclusion}[theorem]{Conclusión}
% \newtheorem{condition}[theorem]{Condición}
% \newtheorem{conjecture}[theorem]{Conjetura}

% \newtheorem{criterion}[theorem]{Criterio}
% \newtheorem{definition}[theorem]{Definición}
% \newtheorem{example}[theorem]{Ejemplo}
% \newtheorem{exercise}[theorem]{Ejercicio}

% \newtheorem{notation}[theorem]{Notación}
% \newtheorem{problem}[theorem]{Problema}

% \newtheorem{remark}[theorem]{Observación}
% \newtheorem{solution}[theorem]{Solución}
% \newtheorem{summary}[theorem]{Summary}
% \numberwithin{equation}{section}



% \newcommand{\av}{\mbox{{\bf Av}}\,}
% \newcommand{\be}{\mbox{{\bf E}}}
% \newcommand{\gik}{g_{i,k}}
% \newcommand{\gaik}{\gamma_{i,k}}
% \newcommand{\sik}{\sigma_{i,k}}
% \newcommand{\hz}{\hat Z}
% \newcommand{\nut}{\nu_t}
% \newcommand{\ou}{[0,1]}
% \newcommand{\rud}{R_{1,2}}
% \newcommand{\sii}{\sigma_i}
% \newcommand{\siN}{\sigma_N}
% \newcommand{\siu}{\sigma_{i_1}}
% \newcommand{\sip}{\sigma_{i_p}}
% \newcommand{\var}{\mbox{Var}}
% \newcommand{\skm}{\sum_{k\le M}}
% \newcommand{\sln}{\sum_{l\le n}}
% \newcommand{\sli}{\sum_{i\le n}}
% \newcommand{\slp}{\sum_{1\le l<l'\le n}}
% \newcommand{\snn}{\sum_{i\le N}}
% \newcommand{\ssn}{\Sigma_N}
% %\newcommand{\1}{{\bf 1}}
% \newcommand{\la}{\lambda}

% %%%%%%%%%%%%%%%%%%%%%%%%%%%%%%%%%%%%%%%%%%%%%%%%%%%%%%%%%%%%%%%%%%%
% %%%%%%%%%%% calligraphic %% %%%%%%%%%%%%%%%%%%%%%%%%%%%%%%%%%%%%%%%
% %%%%%%%%%%%%%%%%%%%%%%%%%%%%%%%%%%%%%%%%%%%%%%%%%%%%%%%%%%%%%%%%%%%
% \newcommand{\ca}{{\cal A}}
% \newcommand{\cb}{{\cal B}}
% \newcommand{\cc}{{\cal C}}
% \newcommand{\cd}{{\cal D}}
% \newcommand{\ce}{{\cal E}}
% \newcommand{\cf}{{\cal F}}
% \newcommand{\cg}{{\cal G}}
% \newcommand{\ch}{{\cal H}}
% \newcommand{\cj}{{\cal J}}
% \newcommand{\cl}{{\cal L}}
% \newcommand{\cm}{{\cal M}}
% \newcommand{\cn}{{\cal N}}
% \newcommand{\co}{{\cal O}}
% \newcommand{\cp}{{\cal P}}
% \newcommand{\ccr}{{\cal R}}
% \newcommand{\cs}{{\cal S}}
% \newcommand{\ct}{{\cal T}}
% \newcommand{\cu}{{\cal U}}

% %%%%%%%%%%%%%%%%%%%%%%%%%%%%%%%%%%%%%%%%%%%%%%%%%%%%%%%%%%%%%%%%%%%
% %%%%%%%%%%%%%%% greek %%%%%%%%%%%%%%%%%%%%%%%%%%%%%%%%%%%%%%%%%%%%%
% %%%%%%%%%%%%%%%%%%%%%%%%%%%%%%%%%%%%%%%%%%%%%%%%%%%%%%%%%%%%%%%%%%%
% \newcommand{\al}{\alpha}
% \newcommand{\ga}{\gamma}
% \newcommand{\ep}{\varepsilon}
\newcommand{\si}{\sigma}
% \newcommand{\vp}{\varphi}

% \newcommand{\laa}{\Lambda}

% %%%%%%%%%%%%%%%%%%%%%%%%%%%%%%%%%%%%%%%%%%%%%%%%%%%%%%%%%%%%%%%%%%%
% %%%%%%%%%%%%% mathbb %%%%%%%%%%%%%%%%%%%%%%%%%%%%%%%%%%%%%%%%%%%%%%
% %%%%%%%%%%%%%%%%%%%%%%%%%%%%%%%%%%%%%%%%%%%%%%%%%%%%%%%%%%%%%%%%%%%
% \newcommand{\D}{{\mathbb D}}
\newcommand{\E}{{\mathbf E}}
% \newcommand{\F}{{\mathbb F}}
% \newcommand{\N}{{\mathbb N}}
% \newcommand{\Q}{{\mathbb Q}}
% \newcommand{\R}{{\mathbb R}}
% \newcommand{\Z}{{\mathbb Z}}


% \newcommand{\lla}{\left\langle}
% \newcommand{\rra}{\right\rangle}
% \newcommand{\lcl}{\left\{}
% \newcommand{\rcl}{\right\}}
% \newcommand{\lp}{\left(}
% \newcommand{\rp}{\right)}
% \newcommand{\lc}{\left[}
% \newcommand{\rc}{\right]}
% \newcommand{\lln}{\left|}
% \newcommand{\rrn}{\right|}
% \newcommand{\rat}{\right\rangle_t}
% \newcommand{\ram}{\right\rangle_{t,-}}
% \newcommand{\raz}{\right\rangle^{\circ}}

\providecommand{\corollaryname}{Corolario}
\providecommand{\examplename}{Ejemplo}
\providecommand{\lemmaname}{Lema}
\providecommand{\propositionname}{Proposition}
\providecommand{\remarkname}{Observación}
\providecommand{\theoremname}{Teorema}
\providecommand{\subexamplename}{Subexample}
\providecommand{\definitionname}{Definición}


\def\tresp#1_#2^#3{\mathrel {\mathop{\kern 0pt#1}\limits_{#2}^{#3}}}

\DeclareMathOperator{\senh}{senh\,}
 %Cargar en este archivo los paquetes que requiera el autor (graphicx, tikz, ...), 
% definir comandos (\DeclareMathOperator{command}{definition}, ... ), etc.

\issueinfo{52}{Comunicaciones}{}{2023} 
\Primera{5} % La página de inicio del artículo 

\pagespan{\theFirstPage}{\pageref{LastPage}}
\PII{\ } %Tipo de artículo, e.g., Artículo de Investigación; Exposición; etc.

\begin{document}

\author[Vázquez A., Monter J. C., Medina M., Zaldívar L. A.]{Alejandro Vázquez Aceves, Juan Carlos Monter Cortés, Muaricio Medina Barcena y Luis Ángel Zaldívar Corichi}

\address{Centro Universitario de Ciencias Exactas e Ingenieria,
Universidad de Guadalajara,
Blvd Gral. Marcelino García Barragán 1421, Olimpica, 44430 Guadalajara, Jalisco}
\email{\lowercase{\texttt{luis.zaldivar@academicos.udg.mx}}}

\address{Centro Universitario de Ciencias Exactas e Ingenieria,
Universidad de Guadalajara,
Blvd Gral. Marcelino García Barragán 1421, Olimpica, 44430 Guadalajara, Jalisco}
\email{\lowercase{\texttt{juan.monter2902@alumnos.udg.mx}}}

% Tantas direcciones como autores.

\title[Teoría de categorías]{Una introducción a la teoría de categorías}

\begin{abstract} 
PENDIENTE
\vskip .3cm

\noindent {\sc Abstract.}  PENDIENTE
\end{abstract}

\subjclass[2000]{82B44}

\keywords{Cristales de spin. Medida de Gibbs.}

\maketitle

\section*{Introducción}\label{Introduccion}

\noindent 
PENDIENTE
\tableofcontents


\section{Conceptos básicos}\label{Conceptos basicos}

\begin{dfn}[Categoría]\label{Definicion categoria}
    Una \emph{categoría} $\mathcal{C}$ consiste de los siguientes datos:
    \begin{enumerate}
        \item Una colección de \emph{objetos}, que denotaremos por $\Obj(\mathcal{C})$.
        \item Una colección de \emph{flechas} ó \emph{morfismos} entre objetos. Cada flecha $f$ tiene un objeto
        de salida $\Dom(f)\in\Obj(\C)$\footnote{Aunque la colección de objetos 
        no forme un conjunto, usamos la notación conjuntista de elemento, para decir
        que el dominio de $f$ es uno de los objetos determinados en el punto 1},
        que llamamos \emph{dominio de} $f$, y un objeto de llegada 
        $\Cod(f)\in\Obj(\C)$, que llamamos \emph{codominio de} $f$. Toda la información
        que carga una flecha se condensa en la notación $f:\Dom(f)\to\Cod(f)$\footnote{
        Aunque la notación sugiere que $f$ es una función, en general no lo es. Además podemos
        usar la notación de diagrama $A\xrightarrow{f}B$}. 
        
        A la colección de flechas de la categoría la denotaremos por $\Fle(\C)$. Y a la colección 
        de flechas con dominio $A$ y codominio $B$ la denotaremos por $\C(A,B)$.
        \item Una regla de composicion de flechas. Lo que significa que a cada par de flechas $f,g\in\Fle(\C)$, tales que $\Cod(f)=\Dom(g)$, 
        le asigna una flecha $g\circ f:\Dom(f)\to \Cod(g)$, llamada composición de $g$ con $f$, y es tal que: 
        
        \begin{itemize}
            \item Para cualesquiera objetos $A,B,C,D\in\Obj(\C)$, y flechas $f:A\to B,g:B\to C$ y $h:C\to D$, se tiene que:
            $$h\circ(g\circ f) = (h\circ g)\circ f.$$
            
            \item Para todo objeto $A\in\Obj(\C)$, existe $\id_A\in\Hom_\C( A,A)$ tal que cualquier objeto $B\in\Obj(\C)$ y flechas $g\in\Hom_\C(A,B)$, $h\in\Hom_\C(B,A)$ satisfacen:
            $$g\circ \id_A=g\text{ y }\id_A\circ h=h.$$
        \end{itemize}
    \end{enumerate}
\end{dfn}

Para simplificar la notación, cuando nos refiramos a un objeto o flecha de una categoría $\C$, escribiremos $A\in\C$ o $f\in\C$; 
y en caso de que sea ambigua usaremos la notación $A\in\Obj(\C)$ o $f\in\Fle(\C)$. También al hablar de una flecha en una categoría, podríamos usar
una notación de diagrama como sigue $$A\xrightarrow{f}B\in\C$$
lo que significa que la flecha $f$ está en la categoría $\C$ y tiene dominio $A$ y codominio $B$. De forma similar, podemos denotar que todas
las flechas de un diagrama estén en una categoría $\C$, y al mismo tiempo decir sus dominios y codominios como sea conveniente. Como por ejemplo
$$\begin{tikzcd}
    A\arrow{r}{f} & B\arrow{r}{g} & C
\end{tikzcd}\in\C, 
\begin{tikzcd}
    A \arrow[d, "f"'] \arrow[r, "g"] & B \\
    C                                &  
\end{tikzcd}\in\C, 
\begin{tikzcd}
    A \arrow[r, "g"', shift right] \arrow[r, "f", shift left] & B
\end{tikzcd}\in\C$$

Hay que decir que las flechas en $\C(A,A)$ reciben el nombre
de \emph{endomorfismos}, por lo que denotaremos a tal colección como $\End(A)$, y de haber alguna duda de la categoría en discusión, usaremos la 
notación $\End_\C(A)$.

Por último hay que mencionar que en un diagrama como el siguiente
\[\begin{tikzcd}
    A \arrow[d, "f"'] \arrow[r, "g"] & B \arrow[ld, "h"] \\
    C                                &                  
    \end{tikzcd}\]
donde es posible componer la flecha $h$ con $g$ y tener una flecha con mismo dominio y codominio que $f$, es equivalente decir
que $h\circ g=f$ a que el \emph{diagrama conmuta}. En general decir que un diagrama conmuta es equivalente a decir que cualesquiera dos
\textquotedblleft caminos\textquotedblright que tengan mismos dominios y codominios, respectivamente, son iguales. De esta manera a través de un diagrama
conmutativo, podemos expresar varias ecuaciones de composiciones de flechas. Por ejemplo, que el siguiente diagrama conmute
\[\begin{tikzcd}
    A \arrow[rd, "h"'] \arrow[r, "f"] & C \arrow[r, "g"]                  & D \\
                                      & B \arrow[u, "d"] \arrow[ru, "e"'] &  
\end{tikzcd}\]
condensa varias ecuaciones, como $f=d\circ h$, $e=g\circ d$, $e\circ h=g\circ f$, y algunas otras redundantes.

\begin{dfn}[Isomorfismo]
    Sea $\C$ una categoría y una flecha $A\xrightarrow{f}B\in\C$. Decimos que $f$ es un \emph{isomorfismo} si existe una flecha 
    $B\xrightarrow{g}A\in\C$ tal que $$g\circ f=\id_A \text{  y  } f\circ g=\id_B$$
    
    En este caso, decimos que $A$ y $B$ son objetos \emph{isomorfos} en $\C$ y denotamos $A\cong_\C B$.
\end{dfn}

En general si no hay ambigüedad de la categoría en la que dos objetos son isomorfos, entonces podemos omitir especificarla, y en la notación
simplemente escribir $A\cong B$.
Como curiosidad acerca de diagramas conmutativos, sería bueno hacer notar que una flecha $A\xrightarrow{f}B\in\C$ es un isomorfismo si y solo si
existe una flecha $B\xrightarrow{g}A\in\C$ tal que el siguiente diagrama
\[\begin{tikzcd}
    A \arrow[rd, "f"'] \arrow[r, "\id_A"] & A \arrow[r, "f"]                      & B \\
                                          & B \arrow[u, "g"] \arrow[ru, "\id_B"'] &  
\end{tikzcd}\]
conmuta.

\begin{prop}
    Sea $\C$ una categoría y $A\xrightarrow{f}B\in\C$ un isomorfismo, entonces existe una única flecha $B\xrightarrow{g}A\in\C$ tal que
    $g\circ f=\id_A$ y $f\circ g=\id_B$.
\end{prop}
\begin{proof}
    Supongamos que existen dos flechas $B \overset{g_1}{\underset{g_2}{\rightrightarrows}}A\in\C$ tales que los diagramas
    $$\begin{tikzcd}
        A \arrow[rd, "f"'] \arrow[r, "\id_A"] & A \arrow[r, "f"]                      & B \\
                                              & B \arrow[u, "g_1"] \arrow[ru, "\id_B"'] &  
    \end{tikzcd} \text{ y } \begin{tikzcd}
        A \arrow[rd, "f"'] \arrow[r, "\id_A"] & A \arrow[r, "f"]                      & B \\
                                              & B \arrow[u, "g_2"] \arrow[ru, "\id_B"'] &  
    \end{tikzcd} $$
    conmutan. Entonces
    $$g_1=g_1\circ\id_B=g_1\circ(f\circ g_2)=(g_1\circ f)\circ g_2=\id_A\circ g_2=g_2.$$
\end{proof}

De lo anterior queda claro que el inverso de un isomorfismo $f$ es único, por lo que lo de notaremos como $f^{-1}$. 
\begin{lem}\label{propiedades de isos}
    \begin{enumerate} \text{\\}
        \item Para todo objeto $A\in\C$, $\id_A$ es un isomorfismo.
        \item Si $f\in\C$ es un isomorfismo, entonces $f^{-1}$ es un isomorfismo.
        \item Si $A\xrightarrow{f}B\xrightarrow{g}C\in\C$ son isomorfismos, entonces $g\circ f$ es un isomorfismo.
    \end{enumerate}
\end{lem}
\begin{proof}\text{\\}
    \begin{enumerate}
        \item Es consecuencia directa de que $\id_A\circ\id_A=\id_A$.
        \item Por definición de isomorfismo, existe $B\xrightarrow{f^{-1}}A$ tal que $f\circ f^{-1}=\id_B$ y $f^{-1}\circ f=\id_A$.
        \item Por definición de isomorfismo, existen $B\xrightarrow{f^{-1}}A$ y $C\xrightarrow{g^{-1}}B$ tales que $f\circ f^{-1}=\id_B$, $f^{-1}\circ f=\id_A$, $g\circ g^{-1}=\id_C$ y $g^{-1}\circ g=\id_B$. Entonces
        $$(g\circ f)\circ(f^{-1}\circ g^{-1})=g\circ(f\circ f^{-1})\circ g^{-1}=g\circ\id_B\circ g^{-1}=g\circ g^{-1}=\id_C$$
        y
        $$(f^{-1}\circ g^{-1})\circ(g\circ f)=f^{-1}\circ(g^{-1}\circ g)\circ f=f^{-1}\circ\id_B\circ f=f^{-1}\circ f=\id_A.$$
    \end{enumerate}
\end{proof}

Consideremos un objeto $A$ en una categoría arbitraria $\C$. Si la colección $\End(A)$ forma un conjunto, entonces este conjunto forma un monoide
con la composición de flechas de $\C$. Cuando un endomorfismo es un isomorfismo, lo llamamos \emph{automorfismo} y por lo tanto podemos definir la subcolección
$$\Aut(A):=\{f\in\End(A) \ |\ f \text{ es un isomorfismo}\}$$

\begin{thm}
    Sea $\C$ una categoría y $A\in\C$ un objeto tal que $\Aut(A)$ es un conjunto. Entonces $\Aut(A)$ es un grupo con la composición de flechas de $\C$.
\end{thm}
\begin{proof}
    Es consecuencia directa del lema \ref{propiedades de isos}
\end{proof}

\begin{dfn}[Monomorfismos y epimorfismos]
    Sea $\C$ una categoría y $A\xrightarrow{f}B\in\C$ una flecha. 
    \begin{enumerate}
        \item Decimos que $f$ es un \emph{monomorfismo} si para cualquier objeto $X$ y flechas $X\overset{\alpha}{\underset{\beta}{\rightrightarrows}}A$
              tales que $f\circ\alpha=f\circ\beta$, entonces $\alpha=\beta$. 
        \item Decimos que $f$ es un \emph{epimorfismo} si para cualquier objeto $Y$ y flechas $B\overset{\alpha}{\underset{\beta}{\rightrightarrows}}Y$
              tales que $\alpha\circ f=\beta\circ f$, entonces $\alpha=\beta$.
    \end{enumerate}
\end{dfn}
Hay que hacer énfasis en que básicamente una flecha es monomorfismo (epimorfismo) si es cancelable por la izquierda (derecha). Sin embargo esto no implica
que exista una flecha tal que la composición (en alguna dirección) sea igual a la identidad, pero puede existir, y por eso introducimos la siguiente definición
\begin{dfn}[Secciones y retractos]
    Sea $\C$ una categoría y $A\xrightarrow{f}B\in\C$ una flecha. 
    \begin{enumerate}
        \item Decimos que $f$ es una \emph{sección} si existe $B\xrightarrow{g}A\in\C$ tal que $g\circ f=\id_A$. 
        \item Decimos que $f$ es un \emph{retracto} si existe $B\xrightarrow{g}A\in\C$ tal que $f\circ g=\id_B$.
    \end{enumerate}
\end{dfn}

Observemos que toda sección (retracto) es un monomorfismo (epimorfismo). Y notemos que, para un flecha en una categoría, aunque no es suficiente ser monomorfismo
y epimorfismo para ser isomorfismo, si es suficiente que sea sección y retracto (Demostración como ejercicio para el lector).

Ejemplos de categorías, en cada ejemplo decir que son Isos, monos, epis, secciones y retractos...

Ahora queremos abordar dos conceptos importantes, el de \textquotedblleft propiedad universal\textquotedblright y el de \textquotedblleft dualidad\textquotedblright .
Empezaremos por el concepto de \textquotedblleft dualidad\textquotedblright , ya que de cierta forma lo hemos mencionado anteriormente. Resulta que los conceptos de
monomorfismo y epimorfismo son conceptos duales; así como el de sección y retracto. Para ver esto hay que notar que esencialmente la diferencia 
entre los dos conceptos (monomorfismos y epimorfismos, ó, secciones y retractos, respectivamente) solo está en la dirección de las flechas. 
Definamos lo que es la categorías dual de una categoría. 

\begin{dfn}[Categoría opuesta o dual]
    Sea $\C$ una categoría. Denotaremos como $\C^\op$ (o $\C^\ast$) a la categoría \emph{opuesta o dual} de $\C$, la cual está determinada de la siguiente manera:
    \begin{itemize}
        \item Los objetos de $\C^\op$ son los mismos que los de $\C$. $\Obj(\C^\op)=\Obj(\C)$.
        \item Los morfismos de $\C^\op$ son \textquotedblleft casi\textquotedblright  los mismos que $\C$, en el sentido de que para cualesquiera
        objetos $A,B\in\C^\op$ definimos $\C^\op(A,B):=\C(B,A)$. Se puede decir simplemente que cada flecha en $\C$ toma la \textquoteleft dirección opuesta\textquotedblright .
        \item La regla de composición la definimos en base a la regla de composición de $\C$ de tal manera que dadas dos flechas $A\xrightarrow{f}B\xrightarrow{g}C\in\C^\op$ entonces:
        $$g\circ_\op f:=f\circ g$$
    \end{itemize}
\end{dfn}

Un buen ejercicio para el lector sería comprobar que la categoría dual definida anteriormente siempre satisface los axiomas de una categoría. Solo restaría
comprobar que la regla de composición definida es asociativa y que cada objeto tiene su flecha identidad. 

En general diremos que una propiedad $P$ es \emph{dual} a una propiedad $Q$ si es equivalente que se satisfaga $P$ en una categoría $\C$ a que se satisfaga $Q$
en la respectiva categoría dual $\C^\op$. Por esta razón muchas definiciones vienen en pares, porque para cada propiedad está la propiedad dual, por ejemplo
los monomorfismos y las secciones son duales a los epimorfismos y los retractos, respectivamente. Mostraremos la afirmación de que
los monomorfismos son duales a los epimorfismos, y recomendamos probar de ejercicio la dualidad entre sección y retracto.

\begin{prop}
    Sea $\C$ una categoría. Una flecha $A\xrightarrow{f}B\in\C$ es un monomorfismo (epimorfismo) si y solo si $B\xrightarrow{f}A\in\C^\op$ es un epimorfismo (monomorfismo).
\end{prop}
\begin{proof}
    Pendiente
\end{proof}

Una propiedad universal es...

\begin{dfn}[Objeto inicial y final]
    Sea $\C$ una categoría y $A\in\C$.
    \begin{enumerate}
        \item Decimos que $A$ es un \emph{objeto inicial} si para cualquier objeto $X$ existe exactamente una flecha $A\xrightarrow{!_X}X\in\C$. 
        \item Decimos que $A$ es un \emph{objeto final} si para cualquier objeto $X$ existe exactamente una flecha $X\xrightarrow{!^X}A\in\C$.
    \end{enumerate}
\end{dfn}

Ejemplos... En el ejemplo de grupos mencionar la definición de \emph{objeto cero}

\begin{dfn}[Producto]
    Sea $\C$ una categoría y sean $A,B\in\C$ dos objetos. El \emph{producto} de $A$ con $B$ es un diagrama \begin{tikzcd}
        & P \arrow[ld, "\pi_A"'] \arrow[rd, "\pi_B"] &   \\
      A &                                            & B
      \end{tikzcd}$\in\C$ tal que para cualquier otro diagrama \begin{tikzcd}
        & X \arrow[ld, "\alpha"'] \arrow[rd, "\beta"] &   \\
        A &                                            & B
        \end{tikzcd}$\in\C$ existe una única flecha $X\xrightarrow{(\alpha,\beta)}P\in\C$ tal que el siguiente diagrama conmuta:
        \[\begin{tikzcd}
            & P \arrow[ld, "\pi_A"'] \arrow[rd, "\pi_B"]                                         &   \\
          A &                                                                                    & B \\
            & X \arrow[lu, "\alpha"] \arrow[ru, "\beta"'] \arrow[uu, "{(\alpha,\beta)}", dashed] &  
          \end{tikzcd}\]
\end{dfn}

\begin{dfn}
    [Coproducto]
    Sea $\C$ una categoría y sean $A,B\in\C$ dos objetos. El \emph{coproducto} de $A$ con $B$ es un diagrama \begin{tikzcd}
                & C &                          \\
        A \arrow[ru, "\iota_A"] &   & B \arrow[lu, "\iota_B"']
        \end{tikzcd}$\in\C$ tal que para cualquier otro diagrama \begin{tikzcd}
                & X  &   \\
        A\arrow[ru, "\alpha"'] &    & B\arrow[lu, "\beta"]
        \end{tikzcd}$\in\C$ existe una única flecha $C\xrightarrow{<\alpha,\beta>}X\in\C$ tal que el siguiente diagrama conmuta:
        \[\begin{tikzcd}
                    & C \arrow[dd, "{<\alpha,\beta>}", dashed] &                                              \\
        A \arrow[rd, "\alpha"', no head] \arrow[ru, "\iota_A"] &                                          & B \arrow[ld, "\beta"] \arrow[lu, "\iota_B"'] \\
                    & X                                        &                                             
        \end{tikzcd}\]
\end{dfn}

Ejemplos... En ejemplo de espacios vectoriales hacer notar que un mismo objeto puede ser producto o coproducto dependiendo de los morfismos a los que esté asociado

\begin{dfn}[(Co)Igualador]
    Sea $\C$ una categoría y sean dos flechas $A\overset{f}{\underset{g}{\rightrightarrows}}B\in\C$.
    \begin{enumerate}
        \item El \emph{igualador} de $f$ y $g$ es un diagrama  \begin{tikzcd}
            I \arrow[r, "{e_{f,g}}"] & A \arrow[r, "f", shift left] \arrow[r, "g"', shift right] & B
            \end{tikzcd}$\in\C$ tal que $f\circ e_{f,g}=g\circ e_{f,g}$ y para cualquier otro diagrama \begin{tikzcd}
                X \arrow[r, "h"] & A \arrow[r, "f", shift left] \arrow[r, "g"', shift right] & B
                \end{tikzcd}$\in\C$ tal que $f\circ h=g\circ h$, existe una única flecha $X\xrightarrow{h_{f,g}}I\in\C$ tal que el siguiente diagrama conmuta:
                \[\begin{tikzcd}
                    X \arrow[d, "{h_{f,g}}"', dashed] \arrow[rd, "h"] &   \\
                    I \arrow[r, "{e_{f,g}}"']                         & A
                    \end{tikzcd}\]
        \item El \emph{coigualador} de $f$ y $g$ es un diagrama \begin{tikzcd}
            A \arrow[r, "f", shift left] \arrow[r, "g"', shift right] & B \arrow[r, "{c_{f,g}}"] & CoI
            \end{tikzcd}$\in\C$ tal que $c_{f,g}\circ f=c_{f,g}\circ g$ y para cualquier otro diagrama \begin{tikzcd}
                A \arrow[r, "f", shift left] \arrow[r, "g"', shift right] & B \arrow[r, "h"] & X
                \end{tikzcd}$\in\C$ tal que $h\circ f=h\circ g$, existe una única flecha $CoI\xrightarrow{h^{f,g}}X\in\C$ tal que el siguiente diagrama conmuta:
                \[\begin{tikzcd}
                                    & X                                   \\
                B \arrow[r, "{c_{f,g}}"'] \arrow[ru, "h"] & CoI \arrow[u, "{h^{f,g}}"', dashed]
                \end{tikzcd}\]
    \end{enumerate}
\end{dfn}

\begin{dfn}[(Co)Producto fibrado]
    Sea $\C$ una categoría.
    \begin{enumerate}
        \item Sean dos flechas \begin{tikzcd}
                    & A \arrow[d, "f"] \\
        B \arrow[r, "g"'] & Z               
        \end{tikzcd}$\in\C$. El \emph{producto fibrado} de $f$ y $g$ es un diagrama conmutativo \begin{tikzcd}
            {P_{f,g}} \arrow[r, "\pi_f"] \arrow[d, "\pi_g"'] & A \arrow[d, "f"] \\
            B \arrow[r, "g"']                                & Z               
        \end{tikzcd}$\in\C$ tal que para cualquier otro diagrama conmutativo \begin{tikzcd}
                X \arrow[r, "\alpha"] \arrow[d, "\beta"'] & A \arrow[d, "f"] \\
                B \arrow[r, "g"']                         & Z               
        \end{tikzcd}$\in\C$, existe una única flecha $X\xrightarrow{(\alpha,\beta)_Z}P_{f,g}\in\C$ tal que el siguiente diagrama conmuta:
        \[\begin{tikzcd}
            X \arrow[rrd, "\alpha", bend left] \arrow[rdd, "\beta"', bend right] \arrow[rd, "{(\alpha,\beta)_Z}", dashed] &                                                  &                  \\
                                                                                                                            & {P_{f,g}} \arrow[r, "\pi_f"] \arrow[d, "\pi_g"'] & A \arrow[d, "f"] \\
                                                                                                                            & B \arrow[r, "g"']                                & Z               
        \end{tikzcd}\]
        \item Sean dos flechas \begin{tikzcd}
            & A                                \\
          B & Z \arrow[u, "f"'] \arrow[l, "g"]
          \end{tikzcd}$\in\C$. El \emph{coproducto fibrado} de $f$ y $g$ es un diagrama conmutativo \begin{tikzcd}
            {C_{f,g}}              & A \arrow[l, "\iota_f"']          \\
            B \arrow[u, "\iota_g"] & Z \arrow[u, "f"'] \arrow[l, "g"]
            \end{tikzcd}$\in\C$ tal que para cualquier otro diagrama conmutativo  \begin{tikzcd}
                X                    & A \arrow[l, "\alpha"']           \\
                B \arrow[u, "\beta"] & Z \arrow[u, "f"'] \arrow[l, "g"]
                \end{tikzcd}$\in\C$, existe una única flecha $C_{f,g}\xrightarrow{<\alpha,\beta>_Z}X\in\C$ tal que el siguiente diagrama conmuta:
            \[\begin{tikzcd}
                X &                                                        &                                                            \\
                  & {C_{f,g}} \arrow[lu, "{<\alpha,\beta>_Z}"', dashed]    & A \arrow[l, "\iota_f"'] \arrow[llu, "\alpha"', bend right] \\
                  & B \arrow[u, "\iota_g"] \arrow[luu, "\beta", bend left] & Z \arrow[u, "f"'] \arrow[l, "g"]                          
            \end{tikzcd}\]
                  
    \end{enumerate}
\end{dfn}

Ejemplos de productos, igualadores y productos fibrados, dejar como ejercicio buscar la construcción dual de cada ejemplo dado.

\begin{thm}
    Sea $\C$ una categoría con productos e igualadores. Entonces $\C$ tiene productos fibrados. 
\end{thm}
\begin{proof}
    PENDIENTE
\end{proof}

Ejercicio enunciar y demostrar el teorema dual.
\section{Funtores}

Ahora que entendemos que es una categoría. Usemos el mismo espíritu de la teoría de categorías para estudiar objetos, y pensar en las categorías como objetos matemáticos.
Lo que queremos decir es que para estudiar categorías con ese mismo espíritu, sería necesario definir alguna noción de flechas entre categorías. 
Eso es justamente un funtor.

\begin{dfn}
    Sean $\A$ y $\B$ dos categorías. Definimos un \emph{funtor} $F:\A\to\B$ como una regla de asignación entre $\A$ y $\B$, de tal forma que:
    \begin{enumerate}
        \item A cada objeto $X\in\A$ le corresponde un solo objeto $F(X)\in\B$.
        \item A cada flecha $X\xrightarrow{f}Y\in\A$ le corresponde una sola flecha $F(X)\xrightarrow{F(f)}F(Y)\in\B$ tal que:
        \begin{itemize}
            \item $F(\id_X)=\id_{F(X)}$ para todo objeto $X\in\A$.
            \item $F(g\circ f)=F(g)\circ F(f)$ para toda flecha $Y\xrightarrow{g}Z\in\A$.
        \end{itemize}
    \end{enumerate}
\end{dfn}

Consideremos dos categorías $\A$ y $\B$ y un funtor $F:\A^\op\to\B$. Notemos que la regla de asignación de $F$ también es una regla de asignación entre $\A$ y $\B$,
pues a cada objeto $A\in\Obj(\A)=\Obj(\A^\op)$ le corresponde el objeto $F(A)\in\B$, y cada flecha $A\xrightarrow{f}B\in\A$ es una flecha $B\xrightarrow{f}A\in\A^\op$ que le corresponde
la flecha $F(B)\xrightarrow{F(f)}F(A)\in\B$. De esta manera $F$ casi es un funtor de $A$ a $B$, con la única diferencia de que cambia la dirección de las flechas.
En este sentido podemos decir que es otro tipo de funtor:

\begin{dfn}[Funtor contravariante]
    Sean $\A$ y $\B$ dos categorías. Definimos un \emph{funtor contravariante} $F:\A\to\B$ como una regla de asignación entre $\A$ y $\B$, de tal forma que:
    \begin{enumerate}
        \item A cada objeto $X\in\A$ le corresponde un solo objeto $F(X)\in\B$.
        \item A cada flecha $X\xrightarrow{f}Y\in\A$ le corresponde una sola flecha $F(Y)\xrightarrow{F(f)}F(X)\in\B$ tal que:
        \begin{itemize}
            \item $F(\id_X)=\id_{F(X)}$ para todo objeto $X\in\A$.
            \item $F(g\circ f)=F(f)\circ F(g)$ para toda flecha $Y\xrightarrow{g}Z\in\A$.
        \end{itemize}
    \end{enumerate}
\end{dfn}

Por lo comentado previo a la definición anterior, queda claro que es equivalente tener un funtor $F:\A^\op\to\B$ a tener un funtor contravariante $F:\A\to\B$.
Dejamos como ejercicio al lector hacer la prueba de que también es equivalente tener un funtor $F:\A\to\B^\op$ a tener un funtor contravariante $F:\A\to\B$.

Ejemplos de funtores... entre ellos la identidad. Funtores de olvido Dar definición de funotr fiel y pleno. Funtores "libres", Funtores Homs, Funtor de abiertos, Morfismos de copos, morfismos de grupos. 
Y quiero dar un contra ejemplo que la conclusión es que sacar el centro de un grupo y verlo como un funtor de $\Grp$ a $\Ab$

\begin{prop}
    Sean dos funtores $\A\xrightarrow{F}\B\xrightarrow{G}\C$, entonces la regla de asignación $G\circ F:\A\to\C$ definida como 
    $G\circ F(X\xrightarrow{f}Y)=G(F(X))\xrightarrow{G(F(f))}G(F(Y))$ para toda flecha $X\xrightarrow{f}Y\in\A$, es un funtor.
\end{prop}
\begin{proof}
    PENDIENTE
\end{proof}

La proposición anterior nos dice como está definida la composición de funtores. Dejamos para el lector el ejercico de probar que para cualquier funtor $\A\xrightarrow{F}\B$, se tiene que $F\circ\id_\A=\id_\B\circ F$.
Veamos ahora que la composición es asociativa.

\begin{prop}
    Sean tres funtores $\A\xrightarrow{F}\B\xrightarrow{G}\C\xrightarrow{H}\D$. Entonces $$H\circ(G\circ F)=(H\circ G)\circ F.$$
\end{prop}
\begin{proof}
    PENDIENTE
\end{proof}

Hemos probado que los funtores y su regla de composición satisfacen de cierta forma lo que pedimos para las flechas de una categoría, por lo que es natural preguntarse
¿Es posible hablar de la categoría de categorías?. No como tal, hay algunos detalles técnicos fundacionales por los que no es tán simple hablar de algo tan inmenso. Una manera de
tratar con este problema está en el libro Abstract and Concrete Categories The Joy of Cats escrito por Jiˇr´ıAd´amek, Horst Herrlich y George E. Strecker, donde se define la 
\emph{quasicategoría} $\CAT$ cuyos objetos son todas las categorías y las flechas son los funtores. No es nuestra intensión abordar este tema a fondo, 
solo dar una fuente para que el lector pueda investigar un poco más a fondo el tema. Sin embargo queremos hacer notar que de igual forma podemos
hablar de categorías que satisfacen propiedades universales (incial, final, productos, coproductos, etc...) e isomorfismos de categorías.

\section{Transformaciones naturales}

Al haber introducido este nuevo objeto de funtor, de nuevo es natural pensar si podemos tener flechas entre funtores. Y esto es posible siempre que los funtores
compartan dominio y codominio.

\begin{dfn}
    Sean $\A,\B$ dos categorías y $\A\overset{F}{\underset{G}{\rightrightarrows}}\B$ dos funtores. Una \emph{transformación natural} de $F$ a $G$
    es una colección de flechas $\alpha=\{F(A)\xrightarrow{\alpha_A}G(A) \ |\ A\in\Obj(\A)\}\subseteq\B$ tal que para cada flecha $A\xrightarrow{f}B\in\A$ el siguiente diagrama conmuta
    \[\begin{tikzcd}
        {F(A)} & {G(A)} \\
        {F(B)} & {G(B)}
        \arrow["{\alpha_A}", from=1-1, to=1-2]
        \arrow["{F(f)}"', from=1-1, to=2-1]
        \arrow["{G(f)}", from=1-2, to=2-2]
        \arrow["{\alpha_B}"', from=2-1, to=2-2]
    \end{tikzcd}\]

    Cada flecha $\alpha_A$ se llama componente en $A$ de $\alpha$.
\end{dfn}

En general usaremos la notación $\alpha:F\Rightarrow G$ para referirnos a una transformación natural $\alpha$ de $F$ a $G$. Para la notación de diagrama tenemos, además
de la natural $F\xRightarrow{\alpha}G$, una notación que deja ver también las categorías involucradas entre los funtores:
\[\begin{tikzcd}
	\A && \B
	\arrow[""{name=0, anchor=center, inner sep=0}, "F", shift left=2, curve={height=-6pt}, from=1-1, to=1-3]
	\arrow[""{name=1, anchor=center, inner sep=0}, "G"', shift right=2, curve={height=6pt}, from=1-1, to=1-3]
	\arrow["\alpha"', shorten <=3pt, shorten >=3pt, Rightarrow, from=0, to=1]
\end{tikzcd}\]

Ejemplos... 

Veamos ahora que como es la composición de transformaciones naturales.

\begin{prop}
    Sean \begin{tikzcd}
        \A && \B
        \arrow[""{name=0, anchor=center, inner sep=0}, "F", shift left=2, curve={height=-12pt}, from=1-1, to=1-3]
        \arrow[""{name=1, anchor=center, inner sep=0}, "H"', shift right=2, curve={height=12pt}, from=1-1, to=1-3]
        \arrow[""{name=2, anchor=center, inner sep=0}, "G"{description}, from=1-1, to=1-3]
        \arrow["\alpha"', shorten <=2pt, shorten >=2pt, Rightarrow, from=0, to=2]
        \arrow["\beta"', shorten <=2pt, shorten >=2pt, Rightarrow, from=2, to=1]
    \end{tikzcd} funtores y transformaciones naturales. Entonces la familia de morfismos $\beta\circ\alpha:=\{F(A)\xrightarrow{\beta_A\circ\alpha_A}H(A) \ |\ A\in\Obj(\A)\}\subseteq\B$
    es una transformación natural de $F$ a $H$
\end{prop}
\begin{proof}
    PENDIENTE
\end{proof}

Notemos que las componentes de una composción son exactamente la composición de las componentes. Dejamos al lector verificar que la composición
es además asociativa ya que basta verificar que componente a componente lo es, y como cada componente es una flecha en una categoría, debe ser asociativa.
Todo esto lo mencionamos para definir la categoría de funtores entre dos categorías $[\A,\B]$, donde los objetos son funtores de $\A$ a $\B$ y las flechas
son transformaciones naturales. Más adelante, en la sección del lema de Yoneda, veremos detalles sobre la categoría de funtores $[\A,\Con]$ donde $\A$ es localmente pequeña.

\section{Límites y colímites}

\begin{dfn}[Diagrama]
    Sean $\C,I$ dos categorías, con $I$ una categoría pequeña. Un \emph{diagrama en} $\C$ \emph{de forma} $I$ es simplemente un funtor $D:I\to\C$.   
\end{dfn}

\begin{dfn}
    Sea $D:I\to\C$ un diagrama en $\C$. Un \emph{cono} de $D$ consta de un objeto $X\in\C$, al que llamamos \emph{vértice del cono}, junto a una familia
    de flechas $\{X\xrightarrow{f_i}D(i) \ |\ i\in\Obj(I)\}\subseteq\C$ tales que para cada flecha $i\xrightarrow{u}j\in I$ el siguiente diagrama conmuta
    \[\begin{tikzcd}
        & X \\
        {D(i)} && {D(j)}
        \arrow["{f_i}"', from=1-2, to=2-1]
        \arrow["{f_j}", from=1-2, to=2-3]
        \arrow["{D(u)}"', from=2-1, to=2-3]
    \end{tikzcd}\]
\end{dfn}

\begin{dfn}
    Sea $D:I\to\C$ un diagrama en $\C$. Un \emph{límite} de $D$ es un cono $(L\xrightarrow{l_i}D(i))_{i\in I}$ tal que para cualquier otro cono $(X\xrightarrow{f_i}D(i))_{i\in I}$
    existe una única flecha $X\xrightarrow{h}L\in\C$ tal que el siguiente diagrama conmuta
    \[\begin{tikzcd}
        X && L \\
        & {D(i)}
        \arrow["h", from=1-1, to=1-3]
        \arrow["{f_i}"', from=1-1, to=2-2]
        \arrow["{l_i}", from=1-3, to=2-2]
    \end{tikzcd}\] para todo $i\in I$
\end{dfn}

Si el límite de un diagrama $D$ existe, normalmente denotamos al vértice de su cono límite con $\lm(D)$.

Conceptos duales...

Obervaciones entre las propiedades universales mencionadas antes y (co)límites...

\begin{dfn}
    Decimos que una categoría $\C$ es \emph{completa}, si para cada categoría pequeña $I$, todo diagrama $D:I\to\C$ tiene límite en $\C$
\end{dfn}


\begin{thm}
    Sean $\A$ y $\C$ dos categorías. Si $\C$ es completa, entonces $[\A,\C]$ es completa
\end{thm}
\begin{proof}
    PENDIENTE
\end{proof}

\section{Lema de Yoneda}

Intrudcción sobre el lema de yoneda y enunción y demostración...

\section{Adjunciones}


\section{La adjunción entre $\Frm$ y $\Top$}

Uno de los primeros ejemplos de marcos, y de hecho, la
motivación para la definición, fue que los abiertos de un espacio
topológico $S$ forman un marco $\cal OS$.
En este capítulo desarrollaremos más a fondo la relación entre marcos
y espacios topológicos. Sabemos que la asignación $S\mapsto\cal OS$ es
un funtor contravariante $\cal O\colon\Top\to\Frm$.
Ahora construiremos un funtor de regreso $\pt:\Frm\to\Top$
y probaremos que ambos funtores forman una adjunción.
\[
  \begin{tikzcd}
    \Top \ar[d,shift right=2,"\cal O"'{name=L}]
    \\
    \Frm^\op \ar[u,shift right=2,"\pt"'{name=R}]
    \adj{L}{R}
  \end{tikzcd}
\]

Antes de continuar, vale la pena mencionar que todos nuestros espacios
(de carne y hueso) serán al menos $T_{0}$. Recordemos que un espacio
$T_0$ es un espacio donde cada par de puntos se pueden separar por al
menos un abierto. Más aún:
\begin{thm}\label{tcero}
    La categoría $\Top_0$ de espacios topológicos
    $T_0$  es \emph{reflexiva} en $\Top$.
    Es decir, el funtor de inclusión $\Top_0\to\Top$
    tiene un adjunto izquierdo.
\end{thm}
\begin{proof}
Es un ejercicio sencillo.
\end{proof}
La razón de esta decisión es que, en un espacio que no es $T_0$, hay
pares de puntos que tienen exactamente las mismas vecindades de
abiertos, así que no tienen mucha esperanza de ser caracterizados por
sus marcos de abiertos.
Otro ejercicio sencillo que ayuda a familiarizarse con los espacios
$T_0$ es el siguiente. En cualquier espacio topológico, los puntos
tienen un preorden (es decir, una relación reflexiva y transitiva)
dado por
  \[q\sqsubseteq p\iff \overline{q}\subseteq \overline{p}\]
  Este se llama preorden de especialización.

\section{El espacio de puntos}
Dado un espacio con un punto $\{*\}$, el conjunto de
puntos de $S$ está en biyección con las funciones continuas
$\{*\}\to S$:
\[
  S \simeq \Top(\{*\},S)
\]
donde cada punto $s\in S$ está asociado a la función $*\mapsto
s$.
Así, si $2$ es el marco de dos elementos $2=\{0<1\}$,
cada de estas funciones $s:\{*\}\to S$ induce un morfismo de
marcos $\chi_s:\cal OS\to\cal O\{*\}\simeq 2$ dado como
\[
  \chi_s(u) =
  \begin{cases}
    1 & s\in u \\
    0 & s\not\in u
  \end{cases}
.\]
Así, para cada marco $A$, tiene sentido definir los puntos de $A$
como morfismos $\Frm(A,2)$.
En efecto, más adelante consideraremos esta construcción.
Sin embargo, primero consideraremos otra construcción equivalente:
representaremos cada morfismo $\chi:A\to 2$ con un elemento de
$A$ de manera canónica: el elemento
\[
    p = \Sup\{x\in A\mid \chi(x)=0\}
\]
es el único elemento de $A$ que cumple
\[
    x\leq p \ssi \chi(x) = 0
.\]
En particular, dado que $\chi(1)=1$, tenemos $p\neq 1$.
Por otro lado, para cualesquiera $x,y\in A$ con
$x\inf y\leq p$, tenemos
\[
    \chi(x)\inf\chi(y)=\chi(x\inf y)=0
,\]
así que $\chi(x)=0$ o bien $\chi(x)=0$, pues $\chi$ toma valores
en el marco $2$. es decir: $x\leq p$ o bien $y\leq p$.

\begin{dfn}
  Sea $A\in \Frm$. Un punto o elemento $\inf$-irreducible 
  de A es un elemento $p\in A$ con $p\neq 1$ tal que si 
  $x\inf y\leq p$, entonces $x\leq p$ ó $y\leq p$. 
  Denotamos por $\pt A$ al conjunto de todos los puntos de $A$.
\end{dfn}


\begin{lem}
  Sea $A\in \Frm$.
  \begin{itemize}
      \item Cada máximo de $A$ es $\inf$-irreducible.
      \item Si A es booleano, entonces todo elemento $\inf-$irreducible de A es máximo.
      \item Si A es una cadena, entonces cada elemento propio de A es $\inf-$irreducible.
  \end{itemize}
\end{lem}
\begin{proof}\quad
  \begin{itemize}
      \item Sea $p\in A$ máximo, entonces $p<1$. Si $x\inf y\leq p$ y suponiendo que $x\not\leq p$, entonces $p<x\sup p$ y, por la maximalidad de $p$, tenemos que $p\sup x=1$. Similarmente, $y\not\leq p$ implica $p\sup y=1$. Si $x\not\leq p$ y $y\not\leq p$, se tiene que 
      \[p=p\sup (x\inf y)=(p\sup x)\inf(p\sup y)=1.\]
      Esto es una contradicción ya que $p<1$.
      \item Supongamos que $A$ es booleano. Sean $p\in\pt A$ y $x,y\in A$ con $p<x$ y $y=\neg x$. Tenemos que $x\inf y=0\leq p$, entonces $x\leq p$ ó $y\leq p$ ya que p es $\inf-$irreducible. Además $y\leq p<x$ puesto que $p<x$. En consecuencia, $x\sup y=1=x$, así, $p$ es máximo.
      \item Supongamos que $A$ es una cadena. Para cualesquiera $x,y\in A$, tenemos que $x\leq y$ ó $y\leq x$, es decir, $x\inf y\leq x$ ó $x\inf y\leq y$. Sea $p\in A$ con $p<1$. Si $x\inf y\leq p$, entonces $x\leq p$ ó $y\leq p$.
  \end{itemize}
\end{proof}

Sean $A\in \Frm$ y $a\in A$. Decimos que un punto $p\in \pt A$ está en $U_A(a)\subseteq \pt A$ si, y sólo si $a\not\leq p$.
\begin{lem}
  Sean $A\in \Frm$ y $a,b\in A$.
  \begin{itemize}
      \item $U_A(1)=\pt A$.
      \item $U_A(0)=\emptyset$.
      \item $U_A(a\inf b)=U_A(a)\cap U_A(b)$.
      \item $U_A(\Sup X)=\bigcup \{U_A(x)|x\in X\}$, $\forall X\subseteq A$.
  \end{itemize}
\end{lem}
\begin{proof}
  Sean $A\in \Frm$ y $a,b\in A$.
\begin{itemize}
\item Por definición $U_A(1)\subseteq \pt A$. Sea $p\in \pt A$, entonces $p\neq 1$. Además $1\not\leq p$, por lo que $p\in U_A(1)$. Así, $U_A(1)=\pt A$.
\item Supongamos que $U_A(0)\neq \emptyset$. Sea $p\in U_A(0)$. Por definición, $0\not\leq p$ pero $0\leq a, \forall a\in A$. Por lo tanto, $U_A(0)=\emptyset$.
\item Sea $p\in \pt A$. Tenemos que
\begin{align*}
p\in U_A(a\wedge b)&\iff a\wedge b\not\leq p\\
&\iff a\not\leq p\quad y\quad b\not\leq p\\
&\iff p\in U_A(a)\quad y\quad p\in U_A(b)\\
&\iff p\in U_A(a)\cap U_A(b).
\end{align*}
Por lo que $U_A(a\wedge b)=U_A(a)\cap U_A(b)$.
\item Sea $X\subseteq A$ y notemos que si $X=\emptyset$, entonces ocurre el segundo punto. En caso contrario,
\begin{align*}
p\in U_A(\bigvee X)&\Rightarrow \bigvee X\not\leq p\\
&\Rightarrow \textit{existe }x\in X\textit{ tal que }x\not\leq p\\
&\Rightarrow p\in U_A(x)\\
&\Rightarrow p\in \bigcup \{U_A(x)\mid x\in X\}.
\end{align*}
Además,
\begin{align*}
p\in \bigcup\{U_A(x)\mid x\in X\}&\Rightarrow p\in U_A(x)\textit{ para algún }x\in X\\
&\Rightarrow x\not \leq p\\
&\Rightarrow \bigvee X\not\leq p\\
&\Rightarrow p\in U_A(\bigvee X).
\end{align*}
Por lo tanto, $U_A(\Sup x)=\bigcup \{U_A(x)|x\in X\}$, $\forall X\subseteq A$.
\end{itemize}
\end{proof}
Se sigue que $U_A(A)=\{U_A(a)\mid a\in A\}$ es una topología en $\pt
A$. Al espacio topológico $(\pt A,U_A(A))$ lo llamamos
el \textit{espacio de puntos} de $A$.
Dado que la topología de $\pt A$ es $\cal O\pt A=U_A(A)$,
se sigue que $U_A:A\to\cal O\pt A$ es un morfismo suprayectivo de marcos, al cual llamamos la \textit{reflexión
espacial} de $A$. Si $U_A$ es inyectivo (y, por lo tanto, un
isomorfismo) decimos que el marco $A$ es \textit{espacial}.

\begin{obs}
  \leavevmode
  \begin{enumerate}
    \item (La reflexión espacial como un cociente)
      Como $U_A:A\to\cal O\pt$ es suprayectivo, el marco $\cal O\pt A$ 
      es el cociente de $A$ bajo el núcleo de $U_A$: el núcleo
      $S\in NA$ dado como
      \[
        x\leq S(a) \iff U(x)\subseteq U(a)
      .\]
    \item
      Además, por el lema adecuado, el
      núcleo $S$ de $U_A$ admite la descripción
      \[
        S(a)=\Inf \{p\in \pt A|a\leq p\}
      .\]
    \item
      Dados dos puntos $p,q\in \pt A$, se cumple
      \begin{align*}
        q\sqsubseteq p&\iff \overline{q}\subseteq \overline{p}\\
        &\iff (\forall x\in A)[q\in U(x)\Rightarrow p\in U(x)]\\
        &\iff (\forall x\in A)[x\leq p\Rightarrow x\leq q]\\
        &\iff p\leq q.
      \end{align*}
      Es decir, el preorden de especialización del espacio de puntos es
      el orden opuesto al orden heredado del marco:
      \[
        (\pt A,\sqsubseteq) = (\pt A,\leq)^\op.
      \]
      En particular, ya que su preorden de especialización es un orden
      parcial, esto prueba que el espacio de puntos es $T_0$.
  \end{enumerate}
\end{obs}

\subsection{Funtorialidad y naturalidad}
Queremos ver que la asignación $A\mapsto \pt A$ es un funtor y que
la reflexión espacial $U_A:A\to\cal O\pt A$
es una transformación natural
\[U_\bullet:\id_{\Frm}\to\cal O\pt(\_).\]

Lo primero es verificar que, dado un morfismo de marcos $f:A\to B$,
obtenemos una función continua $\pt f:\pt B\to\pt A$ entre los
espacios de puntos. De hecho, $\pt f$ será la restricción del adjunto
derecho $f_*:B\to A$ a los puntos de $B$, pero hay que verificar que
$f_*$ manda puntos a puntos.

Si $p$ es un punto de $\pt B$, veamos que $f_*p$ es un punto de $A$.
Primero, $f_*p$ no puede ser $1$, ya que en ese caso $1\leq f_\ast(p)$
implicaría $f(1)\leq p$ por la adjunción, pero esto es imposible ya
que $p\neq 1$.
Ahora veamos que $f_*p$ es $\inf$-irreducible. Si $x,y\in A$ son tales
que $x\inf y\leq f_\ast(p)$, por adjunción tenemos $f(x)\inf f(y)\leq p$.
En consecuencia, $f(x)\leq p$ ó $f(y)\leq p$, i.e., $x\leq f_\ast (p)$
ó $y\leq f_\ast (p)$. Por lo tanto, $f_\ast (p)\in \pt A$.

En resumen, dado un morfismo de marcos $f\colon A\to B$, obtenemos una
función $\pt f \colon \pt B\to \pt A$ dada por la restricción de
$f_*:B\to A$.

Observemos que, para todo $p\in \cal \pt B$, tenemos
\begin{align*}
    p\in (\pt f)^{-1} \left(U_A(a)\right)&\iff f_\ast (p)\in U_A(a)\\
    &\iff a\not\leq f_\ast (p)\\
    &\iff f(a)\not\leq p\\
    &\iff p\in U_B\left(f(a)\right).
\end{align*}
Por lo tanto $\pt f\colon \pt B\to \pt A$ es continua.
Es fácil ver que, dados morfismos $k:C\to B$ y $h:B\to A$,
se satisface $(hk)_*=k_*h_*$.
Además, el adjunto derecho de $\id:A\to A$ también es la identidad
de $A$.
De estas observaciones se sigue que la asignación $\pt$
es un funtor (contravariante) $\pt:\Frm\to\Top$.

Además, en el párrafo anterior probamos que
\[
    \cal O(\pt f)(U_A(a)) = U_B(f(a))
\]
para todo $a\in A$.
Es decir: el diagrama
\[
    \begin{tikzcd}
        A \ar[r,"f"] \ar[d,"U_A"'] & B \ar[d,"U_B"] \\
        \cal O\pt A \ar[r,"\cal O\pt f"'] & \cal O\pt B
    \end{tikzcd}
\]
es conmutativo, así que $U_\bullet=(U_A\mid A\in\Frm)$
es una transformación
natural $U_\bullet:\id_\Frm\to\cal O\pt$.

\subsection{El espacio de puntos del marco de abiertos}

¿Qué tanta información acerca de un espacio topológico se
puede recuperar a través de su marco de abiertos?
Comenzaremos preguntándonos
¿cómo se relacionan los puntos de un espacio $S$ con los puntos de
$\pt\cal OS$?
Como dijimos al principio, un punto $s\in S$ se puede ver como una
función continua $\{s\}\to S$, la cual induce un morfismo
$\chi_s:\cal O S \to 2$ como
\begin{equation}
  \chi_s(u) =
  \begin{cases}
    1 & s\in u  \\
    0 & s\not\in u.
  \end{cases}
\end{equation}
Por lo tanto, el $\inf$-irreducible que le corresponde a $s$ es
\begin{equation}
  \Phi_S(s) = \Sup\{u\in\cal OS\mid \chi_s(u)=0\} \in \pt\cal O S
  .\end{equation}
Esto nos da una función $\Phi_S:S\to\pt\cal OS$. \index[Simbolos]{$\Phi_S$}
Esta descripción
se puede simplificar. Notemos que, dados $s\in S$ y $u\in\cal OS$, tenemos
\[
  \chi_s(u) = 0
  \iff
  s\not\in u
  \iff
  s\in u'
  \iff
  \ol s \subseteq u'
  \iff
  u \subseteq {\ol s}'.
\]
Luego,
\begin{equation}
  \Phi_S(s)
  =
  \Sup\{u\in\cal OS\mid u\subseteq {\ol s}'\}
  =
  {\ol s}'
  \in
  \pt\cal OS
.\end{equation}
Ahora, recordemos que un abierto de $\pt\cal OS$ es de la forma
\begin{align*}
  U_{\cal OS}(u)
  &= \{p\in\pt\cal OS \mid u\not\subseteq p\}
.\end{align*}
Tenemos
\begin{align*}
    s\in (\Phi_S)^{-1}(U_{\cal OS}(u))
    &\iff \Phi_S(s) \in U_{\cal OS}(u) \\
    &\iff {\ol s}' \in U_{\cal OS}(u) \\
    &\iff u \nsubseteq {\ol s}' \\
    &\iff s\in u.
\end{align*}
Es decir,
\[
  (\Phi_S)^{-1}(U_{\cal OS}(u)) = u
.\]
En particular, $(\Phi_S)^{-1}$ manda abiertos de $\pt\cal OS$ en
abiertos de $S$, así que la función $\Phi_S:S\to\pt\cal OS$ es continua.
Por último, observemos que, dada una función continua $\psi:S\to T$,
las funciones $\Phi_S:S\to\pt\cal OS$ hacen conmutar el diagrama
\[
    \begin{tikzcd}
        S \ar[r,"\psi"] \ar[d,"\Phi_S"'] & T \ar[d,"\Phi_T"] \\
        \pt\cal OS \ar[r,"\pt\cal O\psi"'] & \pt\cal OT
    \end{tikzcd}
\]
En efecto, para todo $v\in\cal OT$, tenemos
\begin{align*}
    v\subseteq (\pt\cal O\psi)(\Phi_S(s))
    &\iff v\subseteq (\cal O\psi)_*(\Phi_S(s)) \\
    &\iff (\cal O\psi)(v) \subseteq \Phi_S(s) \\
    &\iff \psi^{-1}(v) \subseteq \ol{s}' \\
    &\iff s\nin \psi^{-1}(v) \\
    &\iff \psi(s)\nin v \\
    &\iff v\subseteq \ol{\psi(s)}' \\
    &\iff v\subseteq \Phi_T(\psi(s)),
\end{align*}
por lo cual $(\pt\cal O\psi)(\Phi_S(s))=\Phi_T(\psi(s))$.
Luego, la familia de funciones continuas
\[
    \Phi_\bullet=(\Phi_S:S\to\pt\cal OS\mid S\in \Top)
\]
es una transformación natural
\[
    \Phi_\bullet : \id_\Top\to\pt\cal O
.\]

\section{La adjunción}\label{ssec:adjuncion}

En la primera parte, vimos que todo morfismo de marcos
$f:A\to B$ induce una función continua $\pt f:\pt B\to\pt A$ dada
como la restricción del adjunto derecho $f_*:B\to A$ de $f$ y
probamos que esta asignación es un funtor $\pt:\Frm\to\Top$.
Ahora veremos que $\pt$ y el funtor de abiertos
$\cal O:\Top\to\Frm$ son las mitades de una adjunción contravariante
entre $\Top$ y $\Frm$. En particular, construiremos un isomorfismo
\begin{equation}\label{eqn:adj_frm_top}
    \Frm(A,\cal OS) \simeq \Top(S,\pt A)
\end{equation}
natural en $A$ y en $S$.

\iffalse
\subsubsection{Usando morfismos iniciales}

Veremos que, para cada espacio topológico $S$, la función continua
$\Phi_S:S\to\pt\cal OS$ es el morfismo inicial de $S$ hacia marcos $A$
a través de $\pt$. En efecto, tomemos cualquier función continua
$\phi:S\to \pt A$. Tomando $f$ como la composición
\[
  A\xto{U_A}\cal O\pt A\xto{\cal O\phi}\pt S,
\]
tenemos la siguiente situación:
\[
  \begin{tikzcd}
    S \ar[r,"\Phi_S"] \ar[dr,"\phi"'] & \pt\cal OS \ar[d,"\pt f"] & \cal OS \\
                      & \pt A & A \ar[u,"f"'].
  \end{tikzcd}
\]
Veremos que el triángulo de la izquierda conmuta y que
$f:A\to\cal OS$ es el único morfismo de marcos que, bajo
$\pt$, lo hace conmutar. Primero notemos que
\[
  (\pt f)(\Phi_S(s))
  = \pt f({\ol s}')
  = f_*({\ol s}')
,\]
así que, para todo $u\in A$,
\begin{align*}
  u\leq (\pt f)(\Phi_S(s))
  &\iff u\leq f_*({\ol s}') \\
  &\iff fu \subseteq {\ol s}' \\
  &\iff \cal O\phi(U_A(u)) \subseteq  {\ol s}' \\
  &\iff \phi^{-1}(U_A(u)) \subseteq  {\ol s}' \\
  &\iff \ol s \subseteq \phi^{-1}(U_A(u)') \\
  &\iff s\in \phi^{-1}(U_A(u)') \\
  &\iff \phi(s)\in U_A(u)' \\
  &\iff \phi(s)\in \{p\in\pt A\mid u\leq p\} \\
  &\iff u\leq \phi(s).
\end{align*}
Es decir, $(\pt f^{\sharp})\circ\Phi_S=\phi$.
Ahora, si $f':A\to\cal OS$ es cualquier morfismo que, al aplicarle
$\pt$, hace conmutar el triángulo (es decir, $(\pt
f')\circ\Phi_S=\phi$), tenemos $(\pt f')\circ\Phi_S = (\pt
f')\circ\Phi_S$.

\subsubsection{Otra manera}
\fi

Cuando aprendimos sobre adjunciones,
vimos el caso covariante, en el cual el isomorfismo de
adjunción es equivalente a la existencia de dos transformaciones
naturales que satisfacen las identidades triangulares.

Ahora veremos que, en el caso contravariante,
tenemos el resultado análogo:
las identidades triangulares adecuadas
implican el isomorfismo natural (\ref{eqn:adj_frm_top}).

Recordemos que las transformaciones naturales
$U_\bullet:\id_\Frm\to\cal O\pt$ y
$\Phi_\bullet:\id_\Top\to\pt\cal O$
tienen componentes dadas como
\begin{align*}
    U_A:A&\to \cal O\pt A \\
    a &\mapsto U_A(a) = \{p\in \pt A \mid a\nleq p\}, \\
    \Phi_S:S&\to \pt\cal O S \\
    s &\mapsto \Phi_S(s)=\ol{s}'.
\end{align*}
Primero veremos que se cumplen las identidades triangulares
\[
    \begin{tikzcd}[row sep=15mm]
        & \cal OS \ar[d,"U_{\cal OS}"] \ar[dl,"\id_{\cal OS}"']
        \\
        \cal OS
        & \cal O\pt\cal OS \ar[l,"\cal O\Phi_S"]
    \end{tikzcd}
    \hspace{10mm}
    \begin{tikzcd}[row sep=15mm]
        & \pt A \ar[d,"\Phi_{\pt A}"] \ar[dl,"\id_{\pt A}"']
        \\
        \pt A
        & \pt \cal O\pt A \ar[l,"\pt U_A"]
    \end{tikzcd}
\]
En efecto, usando las equivalencias
\begin{align*}
    u\subseteq \Phi_S(s) &\ssi s\nin u, \\
    x\in U_A(a) &\ssi a\nleq x,
\end{align*}
tenemos
\begin{align*}
    x\in (\cal O\Phi_S)(U_{\cal OS}(u))
    &\iff \Phi_S(x) \in U_{\cal OS}(u) \\
    &\iff u\nleq \Phi_S(x) \\
    &\iff x\in u,
    \\
    a\leq (\pt U_A)(\Phi_{\pt A}(x))
    &\iff U_A(a) \leq \Phi_{\pt A}(x) \\
    &\iff x\nin U_A(a) \\
    &\iff a\leq x.
\end{align*}
Es decir, $(\cal O\Phi_S)(U_{\cal OS}(u))=u$
y $(\pt U_A)(\Phi_{\pt A}(x))=x$, como se quería.

Ahora, afirmamos que las funciones
\begin{align*}
    \Frm(A,\cal OS) &\to \Top(S,\pt A) \\
    f &\mapsto \bar f = (\pt f)\Phi_S,
    \\
    \Frm(A,\cal OS) &\leftarrow \Top(S,\pt A) \\
    (\cal O\phi)U_A = \phi &\mapsto \bar\phi.
\end{align*}
conforman una biyección. En efecto,
la naturalidad de $\Phi_\bullet$, $U_\bullet$ y las identidades
triangulares implican la conmutatividad de los diagramas
\[
    \begin{tikzcd}[row sep=15mm]
        & \cal OS \ar[d,"U_{\cal OS}"] \ar[dl,"\id_{\cal OS}"']
        & A \ar[l,"f"'] \ar[d,"U_A"]
        \\
        \cal OS
        & \cal O\pt\cal OS \ar[l,"\cal O\Phi_S"]
        & \cal O\pt A \ar[l,"\cal O\pt f"]
    \end{tikzcd}
    \hspace{10mm}
    \begin{tikzcd}[row sep=15mm]
        & \pt A \ar[d,"\Phi_{\pt A}"] \ar[dl,"\id_{\pt A}"']
        & S \ar[l,"\phi"'] \ar[d,"\Phi_S"]
        \\
        \pt A
        & \pt \cal O\pt A \ar[l,"\pt U_A"]
        & \pt \cal OS \ar[l,"\pt\cal O\phi"]
    \end{tikzcd}
\]
por lo cual tenemos
\[
    \begin{aligned}
        \bar{\bar f}
        &= \ol{(\pt f)\Phi_S} \\
        &= \cal O((\pt f)\Phi_S)U_A \\
        &= (\cal O\Phi_S)(\cal O\pt f)U_A \\
        &= f,
    \end{aligned}
    \hspace{20mm}
    \begin{aligned}
        \bar{\bar\phi}
        &= \ol{(\cal O\phi)U_A} \\
        &= \pt((\cal O\phi)U_A)\Phi_S \\
        &= (\pt U_A)(\pt\cal O\phi)\Phi_S \\
        &= \phi.
    \end{aligned}
\]
Esto nos da la biyección (\ref{eqn:adj_frm_top}).
De manera explícita, la biyección está dada como
$\Frm(A,\cal OS)\ni f\leftrightarrow \phi\in \Top(S,\pt A)$, donde
\[
    s\in f(a) \ssi a\nleq \phi(s)
\]
para cualesquiera $s\in S$, $a\in A$, puesto que
\begin{align*}
    s\in f(a)
    &\iff f(a)\nleq \Phi_S(s) \\
    &\iff a \nleq (\pt f)(\Phi_S(s))=\bar f(s),
    \\
    a\nleq \phi(s)
    &\iff \phi(s) \in U_A(a) \\
    &\iff s \in (\cal O\phi)(U_A(a)) = \bar\phi(a).
\end{align*}
Finalmente, veamos que la biyección (\ref{eqn:adj_frm_top})
es natural en $A$ y en $S$.
Dado un morfismo de marcos $g:A\to B$, el diagrama
\[
    \begin{tikzcd}
        \Frm(B,\cal OS) \ar[d,"{-}\circ g"'] \ar[r,"f\mapsto\bar f"']
        & \Top(S,\pt B)
        \ar[d,"\pt g\circ{-}"]
        \\
        \Frm(A,\cal OS) \ar[r,"h\mapsto\bar h"']
        & \Top(S,\pt A)
    \end{tikzcd}
\]
es conmutativo:
\begin{align*}
    \ol{fg}
    &= \pt(fg)\Phi_S \\
    &= (\pt g)(\pt f)\Phi_S \\
    &= (\pt g)\bar f.
\end{align*}
Similarmente, dada una función continua $\psi:S\to T$,
el diagrama
\[
    \begin{tikzcd}
        \Frm(A,\cal OT)
        \ar[d,"\cal O\psi\circ{-}"']
        & \Top(T,\pt A) \ar[l,"\bar \phi\mapsfrom \phi"']
        \ar[d,"{-}\circ \psi"]
        \\
        \Frm(A,\cal OS)
        & \Top(S,\pt A) \ar[l,"\bar \xi\mapsfrom \xi"']
    \end{tikzcd}
\]
es conmutativo:
\begin{align*}
    \ol{\phi\psi}
    &= \cal O(\phi\psi)U_A \\
    &= (\cal O\psi)(\cal O\phi)U_A \\
    &= (\cal O\psi)\bar\phi.
\end{align*}

\section{La propiedad universal de las reflexiones}


Por última sección veremos que las biyecciones dadas por la adjunción revelan las propiedades universales del espacio de puntos y el marco de abiertos, mas delante en el siguiente capítulo se observara que son objetos universales en las respectivas categorías de espacios sobrios y marcos espaciales.
La biyección
\begin{align*}
    \Frm(A,\cal OS) &\simeq \Top(S,\pt A) \\
    f &\mapsto \bar f = (\pt f)\Phi_S \\
    (\cal O\phi)U_A = \bar\phi &\mapsfrom \phi.
\end{align*}
se puede leer como sigue:
dado un morfismo $f:A\to\cal OS$, existe una única función continua
$\phi:S\to\pt A$ tal que el diagrama
\[
    \begin{tikzcd}
        A \ar[r,"f"] \ar[d,"U_A"'] & \cal OS \\
        \cal O\pt A \ar[ur,"\cal O\phi"']
    \end{tikzcd}
\]
conmuta.
Similarmente, dada una función continua $\phi:S\to\pt A$, existe
un único morfismo $f:A\to\cal OS$ tal que el diagrama
\[
    \begin{tikzcd}
        S \ar[r,"\phi"] \ar[d,"\Phi_S"'] & \pt A \\
        \pt\cal OS \ar[ur,"\pt f"']
    \end{tikzcd}
\]
conmuta.


\noindent 
\begin{thebibliography}{99}
  \bibitem{D.P.} C. H. Dowker; D. Strauss, \textit{Separation axioms for frames}, Topics in Topology, pp. 223–240. Proc. Colloq., Keszthely, 1972. Colloq. Math. Soc. Janos Bolyai, vol. 8, North-Holland, Amsterdam, 1974.

  \bibitem{Ib.} J. R. Isbell, \textit{Atomless parts of spaces}, Math. Scand. 31 (1972) 5–32.

  \bibitem{P.T.} P. T. Johnstone, \textit{Stone spaces}, Cambridge Studies in Advanced Mathematics, vol. 3, Cambridge University Press, Cambridge, 1982. MR 698074

  \bibitem{J.S.} P. T. Johnstone; S.-H. Sun, \textit{Weak products and Hausdorff locales}, Categorical Algebra and its Applications, pp. 173–193. Lecture Notes in Mathematics, vol. 1348. Springer-Verlag, Berlin, 1988.

  \bibitem{J.M.} J. Monter; A. Zaldívar, \textit{El enfoque locálico de las reflexiones booleanas: un análisis en la categoría de marcos} [tesis de maestría], 2022. Universidad de Guadalajara.

  \bibitem{P.S.}J. Paseka; B. Šmarda, \textit{T2-frames and almost compact frames}, Czechoslovak Math. J. 42 (1992) 297–313.
  
  \bibitem{J.P.} J. Picado and A. Pultr, \textit{Frames and locales: Topology without points}, Frontiers in Mathematics, Springer Basel, 2012.
  
  \bibitem{J.P.2} J. Picado and A. Pultr, \textit{Separation in point-free topology}, Springer, 2021.
  
  \bibitem{Ro.S.} J. Rosický; B. Šmarda, \textit{T1-locales}, Math. Proc. Cambridge Philos. Soc. 98 (1985) 81–86.
  
  \bibitem{R.S.} RA. Sexton, \textit{A point free and point-sensitive analysis of the patch assembly}, The University of Manchester (United Kingdom), 2003.

  \bibitem{R.S.3} RA. Sexton and H. Simmons, \textit{Point-sensitive and point-free patch constructions}, Journal of Pure and Applied Algebra \textbf{207} (2006), no. 2, 433-468.
    
  \bibitem{H.S.3} H. Simmons, \textit{The assembly of a frame}, University of Manchester (2006).

  \bibitem{H.W.} H. Wallman, \textit{Lattices and topological spaces}, Ann. Math. 39 (1938) 112–126.
  
  \bibitem{A.Z.} A. Zaldívar, \textit{Introducción a la teoría de marcos} [notas curso], 2024. Universidad de Guadalajara.
\end{thebibliography}

\end{document}



\end{document}

\documentclass[10pt]{amsart}
\setlength{\textwidth}{12cm}
\setlength{\textheight}{17.5cm}
%%%%%%%%%%%%%%%%%%%%%%%%%%%%%%%%%%%%%%%%%%%%%%%%%%%%%%%%%%%%%%%%%%%%%%%%%%%%%%%%%%%%%%%%%%%%%%%%%%%%%%%%%%%%%%%%%%%%%%%%%%%%

\usepackage[latin1]{inputenc}
\usepackage{latexsym}
\usepackage{amsfonts}

%%%%%%%%%%%%%%%%%%%%%%%%%%%%%%%%%%%%%%%%%%%%%%%%%%%%%%%%%%%%%%%%%%%%%%%%%%%%%%%%%%%%%%%%%%%%%%%%%%%
\usepackage{graphicx}
\usepackage{amsmath}


%  COMPILE EL TRABAJO DOS VECES PARA QUITAR ALGUNOS WARNINGS 
