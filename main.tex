
\documentclass{comunicaciones}

\usepackage[utopia,sfscaled]{mathdesign}
\usepackage[scaled]{helvet}

\usepackage[utf8]{inputenc}
\usepackage[T1]{fontenc}

\usepackage{microtype}
\usepackage[spanish,mexico,es-noindentfirst]{babel}

\usepackage[paperwidth=170mm,paperheight=230mm,total={125mm,170mm},top=29mm,left=23mm,includeheadfoot]{geometry}

\newcounter{FirstPage}
\newcommand{\Primera}[1]{\setcounter{FirstPage}{#1}}

\usepackage{mathtools}
\usepackage{enumerate}
\usepackage{float}

\renewcommand{\abstractname}{Resumen}
\renewcommand{\keywordsname}{Palabras Claves}
\renewcommand{\figurename}{Figura}
\renewcommand{\tablename}{Tabla}
\renewcommand{\refname}{Bibliografía}

\usepackage[mathscr]{eucal}
\usepackage{enumerate}
\usepackage{times}
\usepackage{tikz}
\usepackage{graphicx}
\usepackage{tikz-cd}\usetikzlibrary{decorations.pathmorphing}
%\usepackage{amsmath,amssymb,latexsym,amscd}   
\usetikzlibrary{babel}
\usepackage{hyperref}
\hypersetup{colorlinks=true,linkcolor=blue,citecolor=brown,linktocpage=true,pagebackref=true,hyperindex=true}
\usepackage{amsthm}
\usepackage{graphicx}
\usepackage[all,cmtip]{xy}
\usepackage{fancyhdr}
\usepackage{mathalfa}
\usepackage{mathrsfs}
\usepackage[sans]{dsfont}
\usepackage{upgreek}

\newcommand{\Frm}{\mathrm{Frm}}
\newcommand{\Pos}{\mathrm{Pos}}
\newcommand{\Dpos}{\mathrm{Dpos}}
\newcommand{\cbd}{\mathrm{cbd}}
\newcommand{\Cbd}{\mathrm{Cbd}}
\newcommand{\CBD}{\mathrm{CBD}}
\newcommand{\Obj}{\mathrm{Obj}}
\newcommand{\Hom}{\mathrm{Hom}}
\newcommand{\Loc}{\mathrm{Loc}}
\newcommand{\CBA}{\mathrm{CBA}}
\newcommand{\Ord}{\mathrm{Ord}}
\newcommand{\Top}{\mathrm{Top}}
\newcommand{\id}{\mathrm{id}}
\newcommand{\Id}{\mathrm{Id}}
\newcommand{\ID}{\mathrm{ID}}
\newcommand{\tp}{\mathrm{tp}}
\newcommand{\Tp}{\mathrm{Tp}}
\newcommand{\TP}{\mathrm{TP}}
\newcommand{\op}{\mathrm{op}}
\newcommand{\pt}{\mathrm{pt}}
\newcommand{\cl}{\mathrm{cl}}
\newcommand{\inte}{\mathrm{int}}
\newcommand{\ob}{\mathrm{ob}}
\newcommand{\ins}{\mathrm{ins}}
\newcommand{\coht}{\mathrm{coht}}
\newcommand{\dep}{\mathrm{dp}}
\newcommand{\sob}{\mathrm{sob}}
\newcommand{\Sob}{\mathrm{Sob}}
\newcommand{\Haus}{\mathrm{Haus}}
\newcommand{\DLat}{\mathrm{DLat}}


\theoremstyle{plain}

\newtheorem*{thm*}{Teorema}
\newtheorem{thm}{\protect\theoremname}[section]
  \theoremstyle{remark}
  \newtheorem{obs}[thm]{\protect\remarkname}
  \theoremstyle{remark}
  \newtheorem{ej}[thm]{\protect\examplename}
    \theoremstyle{plain}
  \newtheorem{subej}[thm]{\protect\subexamplename}
  \theoremstyle{plain}
  \newtheorem{cor}[thm]{\protect\corollaryname}
  \theoremstyle{plain}
  \newtheorem{lem}[thm]{\protect\lemmaname}
  \theoremstyle{plain}
  \newtheorem{prop}[thm]{\protect\propositionname}
    \theoremstyle{definition}
\newtheorem{dfn}[thm]{\protect\definitionname}
\theoremstyle{plain}
\newtheorem*{dfn*}{Definición}
% \theoremstyle{definition}
% \newtheorem{algorithm}[theorem]{Algoritmo}
% \newtheorem{axiom}[theorem]{Suposición}
% \newtheorem{case}[theorem]{Caso}
% \newtheorem{claim}[theorem]{Ayuda}
% \newtheorem{conclusion}[theorem]{Conclusión}
% \newtheorem{condition}[theorem]{Condición}
% \newtheorem{conjecture}[theorem]{Conjetura}

% \newtheorem{criterion}[theorem]{Criterio}
% \newtheorem{definition}[theorem]{Definición}
% \newtheorem{example}[theorem]{Ejemplo}
% \newtheorem{exercise}[theorem]{Ejercicio}

% \newtheorem{notation}[theorem]{Notación}
% \newtheorem{problem}[theorem]{Problema}

% \newtheorem{remark}[theorem]{Observación}
% \newtheorem{solution}[theorem]{Solución}
% \newtheorem{summary}[theorem]{Summary}
% \numberwithin{equation}{section}



% \newcommand{\av}{\mbox{{\bf Av}}\,}
% \newcommand{\be}{\mbox{{\bf E}}}
% \newcommand{\gik}{g_{i,k}}
% \newcommand{\gaik}{\gamma_{i,k}}
% \newcommand{\sik}{\sigma_{i,k}}
% \newcommand{\hz}{\hat Z}
% \newcommand{\nut}{\nu_t}
% \newcommand{\ou}{[0,1]}
% \newcommand{\rud}{R_{1,2}}
% \newcommand{\sii}{\sigma_i}
% \newcommand{\siN}{\sigma_N}
% \newcommand{\siu}{\sigma_{i_1}}
% \newcommand{\sip}{\sigma_{i_p}}
% \newcommand{\var}{\mbox{Var}}
% \newcommand{\skm}{\sum_{k\le M}}
% \newcommand{\sln}{\sum_{l\le n}}
% \newcommand{\sli}{\sum_{i\le n}}
% \newcommand{\slp}{\sum_{1\le l<l'\le n}}
% \newcommand{\snn}{\sum_{i\le N}}
% \newcommand{\ssn}{\Sigma_N}
% %\newcommand{\1}{{\bf 1}}
% \newcommand{\la}{\lambda}

% %%%%%%%%%%%%%%%%%%%%%%%%%%%%%%%%%%%%%%%%%%%%%%%%%%%%%%%%%%%%%%%%%%%
% %%%%%%%%%%% calligraphic %% %%%%%%%%%%%%%%%%%%%%%%%%%%%%%%%%%%%%%%%
% %%%%%%%%%%%%%%%%%%%%%%%%%%%%%%%%%%%%%%%%%%%%%%%%%%%%%%%%%%%%%%%%%%%
% \newcommand{\ca}{{\cal A}}
% \newcommand{\cb}{{\cal B}}
% \newcommand{\cc}{{\cal C}}
% \newcommand{\cd}{{\cal D}}
% \newcommand{\ce}{{\cal E}}
% \newcommand{\cf}{{\cal F}}
% \newcommand{\cg}{{\cal G}}
% \newcommand{\ch}{{\cal H}}
% \newcommand{\cj}{{\cal J}}
% \newcommand{\cl}{{\cal L}}
% \newcommand{\cm}{{\cal M}}
% \newcommand{\cn}{{\cal N}}
% \newcommand{\co}{{\cal O}}
% \newcommand{\cp}{{\cal P}}
% \newcommand{\ccr}{{\cal R}}
% \newcommand{\cs}{{\cal S}}
% \newcommand{\ct}{{\cal T}}
% \newcommand{\cu}{{\cal U}}

% %%%%%%%%%%%%%%%%%%%%%%%%%%%%%%%%%%%%%%%%%%%%%%%%%%%%%%%%%%%%%%%%%%%
% %%%%%%%%%%%%%%% greek %%%%%%%%%%%%%%%%%%%%%%%%%%%%%%%%%%%%%%%%%%%%%
% %%%%%%%%%%%%%%%%%%%%%%%%%%%%%%%%%%%%%%%%%%%%%%%%%%%%%%%%%%%%%%%%%%%
% \newcommand{\al}{\alpha}
% \newcommand{\ga}{\gamma}
% \newcommand{\ep}{\varepsilon}
\newcommand{\si}{\sigma}
% \newcommand{\vp}{\varphi}

% \newcommand{\laa}{\Lambda}

% %%%%%%%%%%%%%%%%%%%%%%%%%%%%%%%%%%%%%%%%%%%%%%%%%%%%%%%%%%%%%%%%%%%
% %%%%%%%%%%%%% mathbb %%%%%%%%%%%%%%%%%%%%%%%%%%%%%%%%%%%%%%%%%%%%%%
% %%%%%%%%%%%%%%%%%%%%%%%%%%%%%%%%%%%%%%%%%%%%%%%%%%%%%%%%%%%%%%%%%%%
% \newcommand{\D}{{\mathbb D}}
\newcommand{\E}{{\mathbf E}}
% \newcommand{\F}{{\mathbb F}}
% \newcommand{\N}{{\mathbb N}}
% \newcommand{\Q}{{\mathbb Q}}
% \newcommand{\R}{{\mathbb R}}
% \newcommand{\Z}{{\mathbb Z}}


% \newcommand{\lla}{\left\langle}
% \newcommand{\rra}{\right\rangle}
% \newcommand{\lcl}{\left\{}
% \newcommand{\rcl}{\right\}}
% \newcommand{\lp}{\left(}
% \newcommand{\rp}{\right)}
% \newcommand{\lc}{\left[}
% \newcommand{\rc}{\right]}
% \newcommand{\lln}{\left|}
% \newcommand{\rrn}{\right|}
% \newcommand{\rat}{\right\rangle_t}
% \newcommand{\ram}{\right\rangle_{t,-}}
% \newcommand{\raz}{\right\rangle^{\circ}}

\providecommand{\corollaryname}{Corolario}
\providecommand{\examplename}{Ejemplo}
\providecommand{\lemmaname}{Lema}
\providecommand{\propositionname}{Proposition}
\providecommand{\remarkname}{Observación}
\providecommand{\theoremname}{Teorema}
\providecommand{\subexamplename}{Subexample}
\providecommand{\definitionname}{Definición}


\def\tresp#1_#2^#3{\mathrel {\mathop{\kern 0pt#1}\limits_{#2}^{#3}}}

\DeclareMathOperator{\senh}{senh\,}


\issueinfo{52}{Comunicaciones}{}{2023}
\Primera{5}
\PII{\ }

\begin{document}

\author[Vázquez A., Monter J. C., Medina M., Zaldívar L. A.]{Alejandro Vázquez Aceves, Juan Carlos Monter Cortés, Mauricio Medina Bárcena y Luis Ángel Zaldívar Corichi}

\address{\parbox{\textwidth}{
Centro Universitario de Ciencias Exactas e Ingenieria\\
Universidad de Guadalajara\\
Blvd Gral. Marcelino García Barragán 1421, Olímpica, 44430 Guadalajara, Jalisco}}

\email{\lowercase{\texttt{luis.zaldivar@academicos.udg.mx}}}
\email{\lowercase{\texttt{mauricio.medina@academicos.udg.mx}}}
\email{\lowercase{\texttt{juan.monter2902@alumnos.udg.mx}}}
\email{\lowercase{\texttt{alejandro.vazquez5702@alumnos.udg.mx}}}

\title[Teoría de categorías]{Una introducción a la teoría de categorías}

\begin{abstract} 
La teoría de categorías es una rama de las matemáticas que estudia la esencia de estructuras matemáticas y relaciones entre ellas. 
Se basa en la idea de que muchas estructuras matemáticas pueden ser vistas como objetos y las relaciones entre ellas como morfismos (o flechas). 
Esta perspectiva permite un enfoque más abstracto y generalizado de las matemáticas, facilitando la comparación y el estudio de diferentes áreas de 
La Matemática.
\vskip .3cm

\noindent {\sc Abstract.}  Category theory is a branch of mathematics that explores mathematical structures and the relationships between them. It is built on the idea that many of these structures can be seen as objects, 
with the connections between them represented as morphisms (or arrows). This point of view offers a more abstract and unified approach, allowing for deeper insight into the inner workings of Mathematics as a whole.
\end{abstract}

\subjclass[2000]{18-01, 18A05} % Puedes cambiar estas según corresponda

\keywords{Categorías, funtores, transformaciones naturales, límites, adjunciones.}

\maketitle

\newpage
\section*{Introducción}\label{Introduccion}

\noindent 

Por más de 2 milenios se ha estudiado indirectamente una relación muy profunda que hay entre el \textquotedblleft contexto\textquotedblright  espacial y el \textquotedblleft contexto\textquotedblright  algebraico. Uno de los avances más importantes en esta relación la hizo René Descartes, ya que gracias a la
idea del plano cartesioano, se evidenció como las ecuaciones (objetos algebraicos) se relacionan con curvas en el plano (objetos espaciales). Con la maduración de las matemáticas y un cambio de paradigma que vino al rededor del siglo XIX, se empezó a entender muchos objetos matemáticas como objetos con cierta estructura, y que el estudio de las estructuras y sus clasificaciones da mucha información matemática de estos objetos. Estas ideas dieron forma a diversas ramas de las matemáticas, como la topología, la teoría de grupos, anillos, campos, espacios vectoriales, y muchas más, pero la protagonista de esta historia y fruto de varias de las antes mencionadas, es la topología algebraica; esta es una rama que básicamente utiliza estructuras algebráicas para estudiar y clasificar estructuras topológicas. 

A mediados del siglo XX, Samuel Eilenberg y Saunders Mac Lane, en un ejercicio del cálculo de una cohomología; un término técnico de topología algebraica que no es necesario aclarar, básicamente estaban trabajando con funtores que asignan a cada espacio topológico, una estructura algebraica; notaron que había maneras \textquotedblleft naturales de pasar\textquotedblright  de un funtor a otro. Esto llevó a la definición formal de transformación natural. Así, en su artículo de 1945, “General Theory of Natural Equivalences”, introdujeron por primera vez el lenguaje de categorías para poder expresar todo esto de forma precisa y general.

En este lenguaje de categorías, hay un nuevo cambio de paradigma. En general muchas definiciones dependen de un conjunto y de hablar de los elementos de ese conjunto, y darle estructura a ese conjunto, y todo eso, da como resultado un objeto matemático que puede ser estudiado mirando sus elementos; en categorías, no es posible mirar un objeto aisladamente, es necesario entenderlo como parte de un \textquotedblleft contexto\textquotedblright  donde sus propiedas las estudiamos a partir de como se \textquotedblleft relaciona\textquotedblright con los demás objetos (incluido él mismo) de dicho \textquotedblleft contexto\textquotedblright . Esta noción de \textquotedblleft contexto\textquotedblright  viene formalizado por lo que es una categoría, que como veremos más a detalle, básicamente consta de dar una colección de objetos, dar una colección de flechas (las \textquotedblleft relaciones\textquotedblright  entre los objetos) , y definir una regla de composición de flechas (algo así como \textquotedblleft pegar relaciones\textquotedblright ) que satisfaga un axioma de asociatividad y un axioma de identidades (véase la definición \ref{Definicion categoria} para detalles). Con este espíritu de entender los objetos a partir de flechas hacia otros objetos de \textquotedblleft una misma naturaleza\textquotedblright , se busca definir el concepto de funtor, el cúal es básicamente una flecha entre categorías, que respeta la composición de flechas. Recursivamente, pensemos si es posible tener
flechas entre funtores, y en efecto lo es,  justo estas son las transformaciones naturales.

Las nociones antes mencionadas se detallan en las primeras 3 secciones de estas notas. La sección 4 busca definir lo que es el límite de un diagrama, y profundizar en un concepto categórico llamado \textquotedblleft universalidad\textquotedblright . Este concepto se aborda en la sección 1, pero no es hasta que vemos el concepto de límite que se puede dar a entender que toda propiedad universal es el límite de algún diagrama. Además en esta sección exploramos la noción de categoría completa, la cual es simplemente una categoría donde todo diagrama tiene límite, y probamos que toda categoría de funtores cuya categoría codominio es completa, es una categoría completa.

En la sección 5  hablamos del lema de Yoneda. Este lema nos da información acerca de la cateogoría de funtores que van de una categoría localmente pequeña a la categoría de conjuntos, y tiene una consecuencia importante que se llama encaje de Yoneda, que básicamente nos dice que toda categoría localmente pequeña está encajada en la categoría de funtores que van de dicha categoría a la categoría de conjuntos, y a sabiendas de que la categoría de conjuntos es una categoría completa, esto nos dice básicamente que toda categoría localmente pequeña está encajada en una categoría completa.

La sección 6 introduce el concepto de adjunción, el cúal es uno de las más importantes en la teoría de categorías. Pasa que aunque podemos hablar de isomorfismos de categorías, esto muchas veces es complicado de que se de en las ramas de las matemáticas conocidas, lo que si es común son las adjunciones, y estás se pueden interpretar como una manera de debilitar la noción de isomorfismo, de hecho llamamos equivalencia de categorías cuando una adjunción satisface unas propiedades particulares. Esta idea de adjunción es tan importante que de manera intuitiva, esta relación entre el \textquotedblleft contexto\textquotedblright  espacial y el \textquotedblleft contexto\textquotedblright  algebraico, que mencionamos al inicio, lo podemos ver como una adjunción; que es justo de lo que habla la sección 7 de estas notas, una adjunción entre la categoría de espacios topológicos y la categoría de marcos. 

\newpage
\tableofcontents

\newpage
\section{Conceptos básicos}\label{Conceptos basicos}
\begin{dfn}[Categoría]\label{Definicion categoria}
 Una \emph{categoría} $\mathcal{C}$ consiste de los siguientes datos:
 \begin{enumerate}
        \item Una colección de \emph{objetos}, que denotaremos por $\Obj(\mathcal{C})$.
        \item Una colección de \emph{flechas} ó \emph{morfismos} entre objetos. Cada flecha $f$ tiene un objeto
        de salida $\Dom(f)\in\Obj(\C)$ \footnote{Aunque la colección de objetos 
        no forme un conjunto, usamos la notación conjuntista de elemento, para decir
        que el dominio de $f$ es uno de los objetos determinados en el punto 1},
        que llamamos \emph{dominio de} $f$, y un objeto de llegada 
        $\Cod(f)\in\Obj(\C)$, que llamamos \emph{codominio de} $f$. Toda la información
        que carga una flecha se condensa en la notación $f:\Dom(f)\to\Cod(f)$\footnote{
        Aunque la notación sugiere que $f$ es una función, en general no lo es.}. 

        A la colección de flechas de la categoría la denotaremos por $\Fle(\C)$. Y a la colección 
        de flechas con dominio $A$ y codominio $B$ la denotaremos por $\C(A,B)$.
        \item Una regla de composición de flechas. Lo que significa que a cada par de flechas $f,g\in\Fle(\C)$, tales que $\Cod(f)=\Dom(g)$, 
        le asigna una flecha $g\circ f:\Dom(f)\to \Cod(g)$, llamada composición de $g$ con $f$, y es tal que: 

                
        \begin{itemize}
            \item Para cualesquiera objetos $A,B,C,D\in\Obj(\C)$, y flechas $f:A\to B,g:B\to C$ y $h:C\to D$, se tiene que:
            $$h\circ(g\circ f) = (h\circ g)\circ f.$$
            
            \item Para todo objeto $A\in\Obj(\C)$, existe $\id_A\in\C( A,A)$ tal que cualquier objeto $B\in\Obj(\C)$ y flechas $g\in\C(A,B)$, $h\in\C(B,A)$ satisfacen:
            $$g\circ \id_A=g\text{ y }\id_A\circ h=h.$$
        \end{itemize}
 \end{enumerate}
\end{dfn}

Para simplificar la notación, cuando nos refiramos a un objeto o flecha de una categoría $\C$, escribiremos $A\in\C$ o $f\in\C$; 
y en caso de que sea ambigua usaremos la notación $A\in\Obj(\C)$ o $f\in\Fle(\C)$. También al hablar de una flecha en una categoría, podríamos usar
una notación de diagrama como sigue $$A\xrightarrow{f}B\in\C$$
lo que significa que la flecha $f$ está en la categoría $\C$ y tiene dominio $A$ y codominio $B$. De forma similar, podemos denotar que todas
las flechas de un diagrama estén en una categoría $\C$, y al mismo tiempo decir sus dominios y codominios como sea conveniente. Como por ejemplo
$$\begin{tikzcd}
    A\arrow{r}{f} & B\arrow{r}{g} & C
\end{tikzcd}\in\C, 
\begin{tikzcd}
    A \arrow[d, "f"'] \arrow[r, "g"] & B \\
    C                                &  
\end{tikzcd}\in\C, 
\begin{tikzcd}
    A \arrow[r, "g"', shift right] \arrow[r, "f", shift left] & B
\end{tikzcd}\in\C$$

Hay que decir que las flechas en $\C(A,A)$ reciben el nombre
de \emph{endomorfismos}, por lo que denotaremos a tal colección como $\End(A)$, y de haber alguna duda de la categoría en discusión, usaremos la 
notación $\End_\C(A)$.

Por último hay que mencionar que en un diagrama como el siguiente
\[\begin{tikzcd}
    A \arrow[d, "f"'] \arrow[r, "g"] & B \arrow[ld, "h"] \\
    C                                &                  
    \end{tikzcd}\]
donde es posible componer la flecha $h$ con $g$ y tener una flecha con mismo dominio y codominio que $f$, es equivalente decir
que $h\circ g=f$ a que el \emph{diagrama conmuta}. En general decir que un diagrama conmuta es equivalente a decir que cualesquiera dos
\textquotedblleft caminos\textquotedblright que tengan mismos dominios y codominios, respectivamente, son iguales. De esta manera a través de un diagrama
conmutativo, podemos expresar varias ecuaciones de composiciones de flechas. Por ejemplo, que el siguiente diagrama conmute
\[\begin{tikzcd}
    A \arrow[rd, "h"'] \arrow[r, "f"] & C \arrow[r, "g"]                  & D \\
                                      & B \arrow[u, "d"] \arrow[ru, "e"'] &  
\end{tikzcd}\]
condensa varias ecuaciones, como $f=d\circ h$, $e=g\circ d$, $e\circ h=g\circ f$, y algunas otras redundantes.

\begin{dfn}[Isomorfismo]
    Sea $\C$ una categoría y una flecha $A\xrightarrow{f}B\in\C$. Decimos que $f$ es un \emph{isomorfismo} si existe una flecha 
    $B\xrightarrow{g}A\in\C$ tal que $$g\circ f=\id_A \text{  y  } f\circ g=\id_B$$
    
    En este caso, decimos que $A$ y $B$ son objetos \emph{isomorfos} en $\C$ y denotamos $A\cong_\C B$.
\end{dfn}

En general si no hay ambigüedad de la categoría en la que dos objetos son isomorfos, entonces podemos omitir especificarla, y en la notación
simplemente escribir $A\cong B$.
Como curiosidad acerca de diagramas conmutativos, sería bueno hacer notar que una flecha $A\xrightarrow{f}B\in\C$ es un isomorfismo si y solo si
existe una flecha $B\xrightarrow{g}A\in\C$ tal que el siguiente diagrama
\[\begin{tikzcd}
    A \arrow[rd, "f"'] \arrow[r, "\id_A"] & A \arrow[r, "f"]                      & B \\
                                          & B \arrow[u, "g"] \arrow[ru, "\id_B"'] &  
\end{tikzcd}\]
conmuta.

\begin{prop}
    Sea $\C$ una categoría y $A\xrightarrow{f}B\in\C$ un isomorfismo, entonces existe una única flecha $B\xrightarrow{g}A\in\C$ tal que
    $g\circ f=\id_A$ y $f\circ g=\id_B$.
\end{prop}
\begin{proof}
    Supongamos que existen dos flechas $B \overset{g_1}{\underset{g_2}{\rightrightarrows}}A\in\C$ tales que los diagramas
    $$\begin{tikzcd}
        A \arrow[rd, "f"'] \arrow[r, "\id_A"] & A \arrow[r, "f"]                      & B \\
                                              & B \arrow[u, "g_1"] \arrow[ru, "\id_B"'] &  
    \end{tikzcd} \text{ y } \begin{tikzcd}
        A \arrow[rd, "f"'] \arrow[r, "\id_A"] & A \arrow[r, "f"]                      & B \\
                                              & B \arrow[u, "g_2"] \arrow[ru, "\id_B"'] &  
    \end{tikzcd} $$
    conmutan. Entonces
    $$g_1=g_1\circ\id_B=g_1\circ(f\circ g_2)=(g_1\circ f)\circ g_2=\id_A\circ g_2=g_2.$$
\end{proof}

De lo anterior queda claro que el inverso de un isomorfismo $f$ es único, por lo que lo de notaremos como $f^{-1}$. 
\begin{lem}\label{propiedades de isos}
    \begin{enumerate} \text{\\}
        \item Para todo objeto $A\in\C$, $\id_A$ es un isomorfismo.
        \item Si $f\in\C$ es un isomorfismo, entonces $f^{-1}$ es un isomorfismo.
        \item Si $A\xrightarrow{f}B\xrightarrow{g}C\in\C$ son isomorfismos, entonces $g\circ f$ es un isomorfismo.
    \end{enumerate}
\end{lem}
\begin{proof}\text{\\}
    \begin{enumerate}
        \item Es consecuencia directa de que $\id_A\circ\id_A=\id_A$.
        \item Por definición de isomorfismo, existe $B\xrightarrow{f^{-1}}A$ tal que $f\circ f^{-1}=\id_B$ y $f^{-1}\circ f=\id_A$.
        \item Por definición de isomorfismo, existen $B\xrightarrow{f^{-1}}A$ y $C\xrightarrow{g^{-1}}B$ tales que $f\circ f^{-1}=\id_B$, $f^{-1}\circ f=\id_A$, $g\circ g^{-1}=\id_C$ y $g^{-1}\circ g=\id_B$. Entonces
        $$(g\circ f)\circ(f^{-1}\circ g^{-1})=g\circ(f\circ f^{-1})\circ g^{-1}=g\circ\id_B\circ g^{-1}=g\circ g^{-1}=\id_C$$
        y
        $$(f^{-1}\circ g^{-1})\circ(g\circ f)=f^{-1}\circ(g^{-1}\circ g)\circ f=f^{-1}\circ\id_B\circ f=f^{-1}\circ f=\id_A.$$
    \end{enumerate}
\end{proof}

Consideremos un objeto $A$ en una categoría arbitraria $\C$. Si la colección $\End(A)$ forma un conjunto, entonces este conjunto forma un monoide
con la composición de flechas de $\C$. Cuando un endomorfismo es un isomorfismo, lo llamamos \emph{automorfismo} y por lo tanto podemos definir la subcolección
$$\Aut(A):=\{f\in\End(A) \ |\ f \text{ es un isomorfismo}\}$$

\begin{thm}
    Sea $\C$ una categoría y $A\in\C$ un objeto tal que $\Aut(A)$ es un conjunto. Entonces $\Aut(A)$ es un grupo con la composición de flechas de $\C$.
\end{thm}
\begin{proof}
    Es consecuencia directa del lema \ref{propiedades de isos}
\end{proof}

\begin{dfn}[Monomorfismos y epimorfismos]
    Sea $\C$ una categoría y $A\xrightarrow{f}B\in\C$ una flecha. 
    \begin{enumerate}
        \item Decimos que $f$ es un \emph{monomorfismo} si para cualquier objeto $X$ y flechas $X\overset{\alpha}{\underset{\beta}{\rightrightarrows}}A$
              tales que $f\circ\alpha=f\circ\beta$, entonces $\alpha=\beta$. 
        \item Decimos que $f$ es un \emph{epimorfismo} si para cualquier objeto $Y$ y flechas $B\overset{\alpha}{\underset{\beta}{\rightrightarrows}}Y$
              tales que $\alpha\circ f=\beta\circ f$, entonces $\alpha=\beta$.
    \end{enumerate}
\end{dfn}
Hay que hacer énfasis en que básicamente una flecha es monomorfismo (epimorfismo) si es cancelable por la izquierda (derecha). Sin embargo esto no implica
que exista una flecha tal que la composición (en alguna dirección) sea igual a la identidad, pero puede existir, y por eso introducimos la siguiente definición
\begin{dfn}[Secciones y retractos]
    Sea $\C$ una categoría y $A\xrightarrow{f}B\in\C$ una flecha. 
    \begin{enumerate}
        \item Decimos que $f$ es una \emph{sección} si existe $B\xrightarrow{g}A\in\C$ tal que $g\circ f=\id_A$. 
        \item Decimos que $f$ es un \emph{retracto} si existe $B\xrightarrow{g}A\in\C$ tal que $f\circ g=\id_B$.
    \end{enumerate}
\end{dfn}

Observemos que toda sección (retracto) es un monomorfismo (epimorfismo). Y notemos que, para un flecha en una categoría, aunque no es suficiente ser monomorfismo
y epimorfismo para ser isomorfismo, si es suficiente que sea sección y retracto (Demostración como ejercicio para el lector).

Ejemplos de categorías, en cada ejemplo decir que son Isos, monos, epis, secciones y retractos...

\begin{ej}[Categorías discretas]
    Consideremos una colección arbitraria $A$, esta colección puede representar una categoría, considerando a cada elemento de $A$ como un objeto, podemos
    simplemente decir que las únicas flechas son las identidades de cada objeto, y por lo tanto la regla de composición es trivial.
    Cualquier categoría $\C$ que satisfaga que hay una biyección entre su colección de objetos $\Obj(\C)$ y su colección de flechas $\Fle(\C)$, le llamamos
    \emph{categoría discreta}. En este tipo de categorías, cada flecha es un isomorfismo, y por lo tanto también es un monomorfismo, epimorfismo, sección y retracto.
\end{ej}

\begin{ej}[Categorías Concretas]\label{Categorías concretas}
    Aún no podemos definir formalmente lo que es una categoría concreta, pero esencialmente son categorías donde sus objetos pueden ser vistos como conjuntos
    con estructuras adicionales, y sus flechas como funciones entre estos conjuntos que preservan las estructuras.
    \begin{itemize}
        \item El ejemplo más sencillo de categoría concreta es la categoría de conjuntos $\Con$, donde los objetos son conjuntos y las flechas son funciones.
        No es complicado verificar que los isomorfismos son funciones biyectivas. Esta categoría satisface que para una función es equivalente ser monomorfismo a
        ser sección a ser inyectiva; y es equivalente ser epimorfismo a ser retracto a ser sobreyectiva.
        \item La categoría de grupos $\Grp$ es otra categoría concreta, donde los objetos son grupos y las flechas son homomorfismos de grupos. en esta categoría
        los isomorfismos son precisamente los homomorfismos biyectivos. Se cumple que es equivalente ser monomorfismo a ser inyectiva; y es equivalente ser
        epimorfismo a ser sobreyectiva. Sin embargo, no es equivalente ser sección a ser inyectiva, ni ser retracto a ser sobreyectiva.
        \item La categoría de espacios topológicos $\Top$ es otra categoría concreta, donde los objetos son espacios topológicos y las flechas son funciones
        continuas. En esta categoría, los isomorfismos son los homeomorfismos, los monomorfismos son las funciones inyectivas continuas, y los epimorfismos
        son las funciones sobreyectivas continuas. Sabemos que en los dos ejemplos anteriores, que una flecha sea monomorfismo y epimorfismo es equivalente
        a que sea un isomorfismo, pero en esta categoría no pasa eso, ya que no es suficiente tener una función biyectiva y continua para que sea un 
        homeomorfismo. De nuevo en esta categoría monomorfismos y secciones no son equivalentes, ni epimorfismos y retractos. 
        \item Dado un campo $K$, tenemos la categoría $\Vect_K$ de espacios vectoriales sobre $K$, donde los objetos son espacios vectoriales con escalares
        en $K$ y las flechas son transformaciones lineales. En esta categoría, los isomorfismos son precisamente las transformaciones lineales biyectivas. 
        De nuevo es equivalente ser monomorfismo a ser inyectiva, ser epimorfismo a ser sobreyectiva, y ser isomorfismo a ser monomorfismo y epimorfismo, pero
        no es equivalente ser sección a ser inyectiva, ni ser retracto a ser sobreyectiva.
        \item La categoría $\Ring$ de anillos con identidad, donde los objetos son anillos con identidad y las flechas son homomorfismos de anillos, funciones
        que preservan las operaciones del anillo, la identidad y el cero. En esta categoría, los isomorfismos son los homomorfismos biyectivos, de nuevo es
        equivalente ser monomorfismo a ser inyectiva, pero no todos los epimorfismos son sobreyectivos, y por lo tanto tampoco es equivalente ser isomorfismo
        a ser monomorfismo y epimorfismo. Sugerimos como ejercicio al lector probar que la inclusión de $\Z$ en $\Q$ es un epimorfismo en $\Ring$. Y con las secciones y
        retractos tenemos el mismo caso que en los ejemplos anteriores (con excepción de $\Con$).
        \item La categoría $\Frm$ de marcos, donde los objetos, llamados marcos, son conjuntos parcialmente ordenados que tienen elemento tope, elemento fondo, supremos arbitrarios,
        ínfimos finitos y estos dos últimos son distributivos entre sí; y las flechas son funciones monótonas que preservan el tope, el fondo, los supremos arbitrarios
        y los ínfimos finitos. Las caracterizaciones de isomorfismos, monomorfismos y epimorfismos son básicamente las mismas que en $\Con$, pero de nuevo,
        como en otras categorías, las secciones y retractos no son equivalentes a los monomorfismos y epimorfismos, respectivamente.
    \end{itemize}
\end{ej}

\begin{ej}[Monoides]\label{Monoides}
    Consideremos un monoide $M$, es decir, un conjunto con una operación binaria asociativa y un elemento neutro. 
    Podemos verlo como una categoría que tiene un solo objeto, y cada flecha, que es un endomorfismo del único objeto, está representada por un elemento
    del monoide. La regla de composición es la operación del monoide, lo que da la asociatividad, y el elemento neutro del monoide es la flecha identidad
    del único objeto. En este tipo de categorías, los isomorfismos son precisamente los elementos invertibles del monoide, las secciones y retractos son los
    elementos que tienen inverso a la izquierda y derecha respectivamente, y los monomorfismos y epimorfismos son los elementos cancelables por la izquierda
    y derecha, respectivamente. Veamos algunas categorías de este tipo:
    \begin{itemize}
        \item El monoide de los naturales $\N$ con la multiplicación como operación binaria. Esta categoría tiene un isomorfismo, el $1$, y todos
        los demás elementos son monomorfismos y epimorfismos simultáneamente, pero ninguno es una sección o un retracto. De hecho en este caso particular, como
        el monoide es conmutativo, tanto las nociones de monomorfismo y epimorfismo son equivalentes, como las nociones de sección y retracto.
        \item El monoide de los enteros $\Z$ con la suma. Esta categoría solo tiene isomorfismos, y sabemos que es un grupo, entonces en particular 
        los grupos son categorías con un único objeto donde todos los morfismos son isomorfismos.
    \end{itemize}
\end{ej}

\begin{ej}[Categorías delgadas]\label{categoria delgada}
    Una categoría $\C$ es \emph{delgada} si para cualquier par de objetos $A,B\in\Obj(\C)$, la colección de flechas $\C(A,B)$ tiene a lo más un elemento.
    Notemos que esto es equivalente a tener una colección arbitraria $A$, y tener un preorden sobre $A$, es decir, una relación binaria $\leq$ sobre $A$ 
    reflexiva y transitiva. Entonces podemos ver al preorden $(A,\leq)$ como una categoría que tiene como objetos los elementos de $A$ y simplemente decimos que hay
    una única flecha de $a$ a $b$ si y solo si $a\leq b$; la regla de composición está dada por la transitividad de la relación, que es asociativa trivialmente 
    (pues a lo mucho hay una única flecha entre dos objetos), y la identidad de cada objeto viene dada por la reflexividad de la relación. En este tipo de categorías
    todas las flechas son monomorfismos y epimorfismos, pero solo son secciones y retractos si son isomorfismos.
\end{ej}

Ahora queremos abordar dos conceptos importantes, el de \textquotedblleft universalidad\textquotedblright y el de \textquotedblleft dualidad\textquotedblright . Empezaremos por el segundo.

\subsection{Dualidad}
Notemos que esencialmente la diferencia entre los conceptos de monomorfismo y epimorfismo, o de sección y retracto, solo está 
en la dirección de las flechas. En esta última observación está la clave para entender el concepto de dualidad. Antes de dar una definición más formal 
definamos lo que es la categorías dual de una categoría. 

\begin{dfn}[Categoría opuesta o dual]
    Sea $\C$ una categoría. Denotaremos como $\C^\op$ (o $\C^\ast$) a la categoría \emph{opuesta o dual} de $\C$, la cual está determinada de la siguiente manera:
    \begin{itemize}
        \item Los objetos de $\C^\op$ son los mismos que los de $\C$. $\Obj(\C^\op)=\Obj(\C)$.
        \item Los morfismos de $\C^\op$ son \textquotedblleft casi\textquotedblright  los mismos que $\C$, en el sentido de que para cualesquiera
        objetos $A,B\in\C^\op$ definimos $\C^\op(A,B):=\C(B,A)$. Se puede decir simplemente que cada flecha en $\C$ toma la \textquotedblleft dirección opuesta\textquotedblright .
        \item La regla de composición la definimos en base a la regla de composición de $\C$ de tal manera que dadas dos flechas $A\xrightarrow{f}B\xrightarrow{g}C\in\C^\op$ entonces:
        $$g\circ_\op f:=f\circ g$$
    \end{itemize}
\end{dfn}

Un buen ejercicio para el lector sería comprobar que la categoría dual definida anteriormente siempre satisface los axiomas de una categoría. Solo restaría
comprobar que la regla de composición definida es asociativa y que cada objeto tiene su flecha identidad. 

En general diremos que una propiedad $P$ es \emph{dual} a una propiedad $Q$ si es equivalente que se satisfaga $P$ en una categoría $\C$ a que se satisfaga $Q$
en la respectiva categoría dual $\C^\op$. Por esta razón muchas definiciones vienen en pares, porque para cada propiedad está la propiedad dual, por ejemplo
los monomorfismos y las secciones son duales a los epimorfismos y los retractos, respectivamente. Mostraremos la afirmación de que
los monomorfismos son duales a los epimorfismos, y recomendamos probar de ejercicio la dualidad entre sección y retracto.

\begin{prop}
    Sea $\C$ una categoría. Una flecha $A\xrightarrow{f}B\in\C$ es un monomorfismo si y solo si $B\xrightarrow{f}A\in\C^\op$ es un epimorfismo.
\end{prop}
\begin{proof}
    Supongamos que $f$ es un monomorfismo en $\C$, y sean dos flechas $A\overset{\alpha}{\underset{\beta}{\rightrightarrows}}X\in\C^\op$ tales que
    $\alpha\circ^\op f=\beta\circ^\op f$ en $\C^\op$. Entonces, por la definición de la composición de flechas en $\C^\op$, tenemos que
    $f\circ\alpha=\beta\circ f$ en $\C$. Como $f$ es un monomorfismo, esto implica que $\alpha=\beta$. Por lo tanto, $f$ es un epimorfismo en $\C^\op$.

    Ahora supongamos que $f$ es un epimorfismo en $\C^\op$, y sean dos flechas $X\overset{\alpha}{\underset{\beta}{\rightrightarrows}}A\in\C$ tales que
    $f\circ\alpha=f\circ\beta$ en $\C$. Entonces, por la definición de la composición de flechas en $\C^\op$, tenemos que
    $\alpha\circ^\op f=\beta\circ^\op f$ en $\C^\op$. Como $f$ es un epimorfismo en $\C^\op$, esto implica que $\alpha=\beta$. Por lo tanto, $f$ es un monomorfismo en $\C$.
\end{proof}

\subsection{Universalidad}

Pensemos en una categoría $\C$, y consideremos un objeto $U\in\C$ que satisface unas propiedades con respecto a algunas flechas en la categoría. La universalidad de $U$ se refiere a que esas propiedades que satisfece con respecto a algunas flechas, son \textquotedblleft esenciales\textquotedblright  
para \textquotedblleft describir\textquotedblright de manera única cualesquiera otras propiedades, sobre las mismas flechas, que pueda llegar a tener otro objeto de la categoría. En tal caso decimos que también $U$ satisface una \emph{propiedad universal}. Veamos algunos ejemplos de propeidades universales.

\begin{dfn}[Objeto inicial y final]
    Sea $\C$ una categoría y $A\in\C$.
    \begin{enumerate}
        \item Decimos que $A$ es un \emph{objeto inicial} si para cualquier objeto $X$ existe exactamente una flecha $A\xrightarrow{!_X}X\in\C$. 
        \item Decimos que $A$ es un \emph{objeto final} si para cualquier objeto $X$ existe exactamente una flecha $X\xrightarrow{!^X}A\in\C$.
    \end{enumerate}
\end{dfn}

\begin{ej}
    \begin{itemize}
        \item En la categoría de conjuntos $\Con$, el conjunto vacío es un objeto inicial, y cualquier conjunto unipuntual es un objeto final.
        \item En la categoría de grupos $\Grp$, el grupo trivial es un objeto incial y final simultáneamente. En general esto puede pasar, y cuando un objeto
        es inicial y final simultáneamente, lo llamamos \emph{objeto cero}.
        \item Consideremos la categoría delgada $(\Z,\ |\ )$, de los enteros con el preorden de divisibilidad, como se describe en el 
        ejemplo \ref{categoria delgada}. En esta categoría $1$ y $-1$ son objetos iniciales; y el $0$ es un objeto final. 
    \end{itemize}
\end{ej}

Como vemos estas propiedades universales descritas, no son sobre ninguna flecha, de hecho podemos decir que es sobre un conjunto vacío de flechas. En las sección
de límites veremos que esto tiene mucho sentido y que de hecho cualquier propiedad universal puede ser vista como un límite o un colímite. Antes de mirar otras
propiedades universales, queremos recordar que muchas veces para hacer referencia a un objeto junto con algunas flechas, muchas veces usamos la notación
de diagrama; con esta idea, de cierta forma podemos decir que las propiedades universales son propiedades de los diagramas que tienen un objeto
\textquotedblleft central\textquotedblright .

\begin{dfn}[Producto]
    Sea $\C$ una categoría y sean $A,B\in\C$ dos objetos. El \emph{producto} de $A$ con $B$ es un diagrama \begin{tikzcd}
        & P \arrow[ld, "\pi_A"'] \arrow[rd, "\pi_B"] &   \\
      A &                                            & B
      \end{tikzcd}$\in\C$ tal que para cualquier otro diagrama \begin{tikzcd}
        & X \arrow[ld, "\alpha"'] \arrow[rd, "\beta"] &   \\
        A &                                            & B
        \end{tikzcd}$\in\C$ existe una única flecha $X\xrightarrow{(\alpha,\beta)}P\in\C$ tal que el siguiente diagrama conmuta:
        \[\begin{tikzcd}
            & P \arrow[ld, "\pi_A"'] \arrow[rd, "\pi_B"]                                         &   \\
          A &                                                                                    & B \\
            & X \arrow[lu, "\alpha"] \arrow[ru, "\beta"'] \arrow[uu, "{(\alpha,\beta)}", dashed] &  
          \end{tikzcd}\]
\end{dfn}

\begin{dfn}[Coproducto]
    Sea $\C$ una categoría y sean $A,B\in\C$ dos objetos. El \emph{coproducto} de $A$ con $B$ es un diagrama \begin{tikzcd}
                & C &                          \\
        A \arrow[ru, "\iota_A"] &   & B \arrow[lu, "\iota_B"']
        \end{tikzcd}$\in\C$ tal que para cualquier otro diagrama \begin{tikzcd}
                & X  &   \\
        A\arrow[ru, "\alpha"'] &    & B\arrow[lu, "\beta"]
        \end{tikzcd}$\in\C$ existe una única flecha $C\xrightarrow{<\alpha,\beta>}X\in\C$ tal que el siguiente diagrama conmuta:
        \[\begin{tikzcd}
                    & C \arrow[dd, "{<\alpha,\beta>}", dashed] &                                              \\
        A \arrow[rd, "\alpha"', no head] \arrow[ru, "\iota_A"] &                                          & B \arrow[ld, "\beta"] \arrow[lu, "\iota_B"'] \\
                    & X                                        &                                             
        \end{tikzcd}\]
\end{dfn}

\begin{ej}[(Co)Productos]
    \begin{itemize}
        \item En $\Con$, el producto de $A$ y $B$ es el producto cartesiano $A\times B$, junto a las proyecciones 
        $\pi_1:A\times B\to A$ y $\pi_2:A\times B\to B$, definidas como $\pi_1(a,b)=a$ y $\pi_2(a,b)=b$. Y el coproducto es la unión disjunta 
        $A\sqcup B=(A\times\{1\})\cup(B\times\{2\})$ junto a las inclusiones $\iota_1:A\to A\sqcup B$ y $\iota_2:B\to A\sqcup B$, definidas como 
        $\iota_1(a)=(a,1)$ y $\iota_2(b)=(b,2)$. 
        \item Consideremos un campo $K$ y la categoría $\Vect_K$ de espacios vectoriales sobre $K$. Veamos como las flechas asociadas al objeto que decimos
        que satisface la propiedad universal, son realmente importantes. En esta categoría, el espacio vectorial $V\times W$ junto a los morfismos
        $\pi_V:V\times W\to V$ y $\pi_W:V\times W\to W$ (definidos como en el punto anterior) son un producto en la categoría $\Vect_K$; y si consideramos 
        los morfismos $\iota_V:V\to V\times W$ y $\iota_W:W\to V\times W$, definidos como $\iota_V(v)=(v,0)$ y $\iota_W(w)=(0,w)$, ahora es un coproducto. 
        \item Consideremos de nuevo la categoría delgada $(\Z,\ |\ )$. En esta categoría el producto de dos objetos $a,b\in\Z$ es el máximo común divisor
        $mcd(a,b)$, junto a los únicos morfismos que hay entre $mcd(a,b)$ y $a$, o $b$. Y el coproducto de $a$ y $b$ es el mínimo común múltiplo $mcm(a,b)$, 
        de nuevo junto a los únicos morfismos que hay entre $a$, o $b$, y $mcm(a,b)$.    
    \end{itemize}
\end{ej}

En general no queremos detenernos en muchos detalles sobre cada propiedad universal. Como ya notaron antes, es posible que varios objetos con diversos morfismos
pueden satisfacer una misma propiedad universal, y esto no es problemático, ya que se puede probar que si dos objetos, junto a sus respectivas familias de morfismos,
satisfacen una misma propiedad universal, estos objetos deben ser isomorfos; esta prueba lo haremos de manera más general en la sección de límites, donde veremos
que todas las propiedades mencionadas en esta sección son un caso particular de límites de un diagrama. Antes de continuar, queremos mencionar que en general
denotaremos el producto de dos objetos $A$ y $B$ como $A\times B$ y al coproducto como $A+B$. Además diremos que una categoría tiene (co)productos binarios
si para cada pareja de objetos existe su (co)producto en la categoría.

\begin{dfn}[(Co)Igualador]
    Sea $\C$ una categoría y sean dos flechas $A\overset{f}{\underset{g}{\rightrightarrows}}B\in\C$.
    \begin{enumerate}
        \item El \emph{igualador} de $f$ y $g$ es un diagrama  \begin{tikzcd}
            I \arrow[r, "{e_{f,g}}"] & A \arrow[r, "f", shift left] \arrow[r, "g"', shift right] & B
            \end{tikzcd}$\in\C$ tal que $f\circ e_{f,g}=g\circ e_{f,g}$ y para cualquier otro diagrama \begin{tikzcd}
                X \arrow[r, "h"] & A \arrow[r, "f", shift left] \arrow[r, "g"', shift right] & B
                \end{tikzcd}$\in\C$ tal que $f\circ h=g\circ h$, existe una única flecha $X\xrightarrow{h_{f,g}}I\in\C$ tal que el siguiente diagrama conmuta:
                \[\begin{tikzcd}
                    X \arrow[d, "{h_{f,g}}"', dashed] \arrow[rd, "h"] &   \\
                    I \arrow[r, "{e_{f,g}}"']                         & A
                    \end{tikzcd}\]
        \item El \emph{coigualador} de $f$ y $g$ es un diagrama \begin{tikzcd}
            A \arrow[r, "f", shift left] \arrow[r, "g"', shift right] & B \arrow[r, "{c_{f,g}}"] & CoI
            \end{tikzcd}$\in\C$ tal que $c_{f,g}\circ f=c_{f,g}\circ g$ y para cualquier otro diagrama \begin{tikzcd}
                A \arrow[r, "f", shift left] \arrow[r, "g"', shift right] & B \arrow[r, "h"] & X
                \end{tikzcd}$\in\C$ tal que $h\circ f=h\circ g$, existe una única flecha $CoI\xrightarrow{h^{f,g}}X\in\C$ tal que el siguiente diagrama conmuta:
                \[\begin{tikzcd}
                                    & X                                   \\
                B \arrow[r, "{c_{f,g}}"'] \arrow[ru, "h"] & CoI \arrow[u, "{h^{f,g}}"', dashed]
                \end{tikzcd}\]
    \end{enumerate}
\end{dfn}

\begin{dfn}[(Co)Producto fibrado]
    Sea $\C$ una categoría.
    \begin{enumerate}
        \item Sean dos flechas \begin{tikzcd}
                    & A \arrow[d, "f"] \\
        B \arrow[r, "g"'] & Z               
        \end{tikzcd}$\in\C$. El \emph{producto fibrado} de $f$ y $g$ es un diagrama conmutativo \begin{tikzcd}
            {P_{f,g}} \arrow[r, "\pi_f"] \arrow[d, "\pi_g"'] & A \arrow[d, "f"] \\
            B \arrow[r, "g"']                                & Z               
        \end{tikzcd}$\in\C$ tal que para cualquier otro diagrama conmutativo \begin{tikzcd}
                X \arrow[r, "\alpha"] \arrow[d, "\beta"'] & A \arrow[d, "f"] \\
                B \arrow[r, "g"']                         & Z               
        \end{tikzcd}$\in\C$, existe una única flecha $X\xrightarrow{(\alpha,\beta)_Z}P_{f,g}\in\C$ tal que el siguiente diagrama conmuta:
        \[\begin{tikzcd}
            X \arrow[rrd, "\alpha", bend left] \arrow[rdd, "\beta"', bend right] \arrow[rd, "{(\alpha,\beta)_Z}", dashed] &                                                  &                  \\
                                                                                                                            & {P_{f,g}} \arrow[r, "\pi_f"] \arrow[d, "\pi_g"'] & A \arrow[d, "f"] \\
                                                                                                                            & B \arrow[r, "g"']                                & Z               
        \end{tikzcd}\]
        \item Sean dos flechas \begin{tikzcd}
            & A                                \\
          B & Z \arrow[u, "f"'] \arrow[l, "g"]
          \end{tikzcd}$\in\C$. El \emph{coproducto fibrado} de $f$ y $g$ es un diagrama conmutativo \begin{tikzcd}
            {C_{f,g}}              & A \arrow[l, "\iota_f"']          \\
            B \arrow[u, "\iota_g"] & Z \arrow[u, "f"'] \arrow[l, "g"]
            \end{tikzcd}$\in\C$ tal que para cualquier otro diagrama conmutativo  \begin{tikzcd}
                X                    & A \arrow[l, "\alpha"']           \\
                B \arrow[u, "\beta"] & Z \arrow[u, "f"'] \arrow[l, "g"]
                \end{tikzcd}$\in\C$, existe una única flecha $C_{f,g}\xrightarrow{<\alpha,\beta>_Z}X\in\C$ tal que el siguiente diagrama conmuta:
            \[\begin{tikzcd}
                X &                                                        &                                                            \\
                  & {C_{f,g}} \arrow[lu, "{<\alpha,\beta>_Z}"', dashed]    & A \arrow[l, "\iota_f"'] \arrow[llu, "\alpha"', bend right] \\
                  & B \arrow[u, "\iota_g"] \arrow[luu, "\beta", bend left] & Z \arrow[u, "f"'] \arrow[l, "g"]                          
            \end{tikzcd}\]
                  
    \end{enumerate}
\end{dfn}

\begin{ej}[Igualadores y Productos fibrados]
    \begin{itemize}
    \item En $\Con$ el igualador de dos funciones $f,g:A\to B$ es el conjunto $I=\{a\in A \ | \ f(a)=g(a)\}$ junto a la inclusión $\iota:I\to A$.
    \item En $\Con$ el producto fibrado de dos funciones $f:A\to Z$ y $g:B\to Z$ es el conjunto $P_{f,g}=\{(a,b)\in A\times B \ | \ f(a)=g(b)\}$ junto
    a las proyecciones del producto $A\times B$ restringidas a $P_{f,g}$.
    \item En $\Grp$ el kernel de un homomorfismo $f:G\to H$ puede ser visto, tanto como un igualador de $f$ con el homomorfismo trivial; como el coproducto 
    de $f$ con el único homomorfismo del grupo trivial a $H$.
    \end{itemize}
\end{ej}

No haremos ejemplos de coigualadores y coproductos fibrados porque estos son un poco 
complicados de construir, en el sentido de que se requiere definir una relación de equivalencia para definir el cociente que terminará siendo el \textquotedblleft coobjeto\textquotedblright  
correspondiente. En términos categóricos, simplemente son conceptos duales, y en general, de ahora en adelante, buscaremos solo dar ejemplos de una dualidad de cada concepto, y dejar algún ejercicio
para la otra dualidad. Recomendamos como ejercicio comprobar que en la categoría de grupos abelianos $\Ab$, el cokernel de un homomorfismo es un coproducto fibrado.

Ahora queremos presentar un pequeño teorema que relaciona varias propiedades universales que hemos mencionado, solo antes queremos decir que así como decimos
que una cateogoría tiene (co)productos binarios, podemos decir que una categoría tiene (co)igualadores si para cada par de flechas de la forma adecuada,
tienen (co)igualador en la categoría; también decimos que tiene (co)productos fibrados si para cada par de flechas con la forma adecuada, tienen (co)producto
fibrado en la categoría.

\begin{thm}
    Sea $\C$ una categoría con productos e igualadores. Entonces $\C$ tiene productos fibrados.
\end{thm}
\begin{proof}
    Sean dos flechas \begin{tikzcd}
                    & A \arrow[d, "f"] \\
        B \arrow[r, "g"'] & Z               
        \end{tikzcd}$\in\C$. Como $\C$ tiene productos tenemos que existe $A\times B\in\C$ junto a las respectivas proyecciones. Entonces tenemos el par de flechas 
        $A\times B\overset{f\circ\pi_A}{\underset{g\circ\pi_B}{\rightrightarrows}}Z\in\C$, y como $\C$ tiene igualadores, entonces existe $I\in\C$ junto a una flecha 
        $e:I\to A\times B$ tal que $f\circ\pi_A\circ e=g\circ\pi_B\circ e$. Comprobemos que el diagrama conmutativo
        \[\begin{tikzcd}
            {I} \arrow[r, "\pi_A\circ e"] \arrow[d, "\pi_B\circ e"'] & A \arrow[d, "f"] \\
            B \arrow[r, "g"']                                & Z               
        \end{tikzcd}\]
        es un producto fibrado. Para esto sea $X\in\C$ y sean dos flechas $X\xrightarrow{\alpha}A$ y $X\xrightarrow{\beta}B$ tales que $f\circ\alpha=g\circ\beta$. Por la propiedad universal
        del producto, existe una única flecha $X\xrightarrow{(\alpha,\beta)}A\times B\in\C$ tal que el siguiente diagrama conmuta:
        \[\begin{tikzcd}
            & A\times B \arrow[ld, "\pi_A"'] \arrow[rd, "\pi_B"]                                         &   \\
          A &                                                                                    & B \\
            & X \arrow[lu, "\alpha"] \arrow[ru, "\beta"'] \arrow[uu, "{(\alpha,\beta)}", dashed] &  
          \end{tikzcd}\]
        Entonces tenemos que $f\circ\pi_A\circ(\alpha,\beta)=g\circ\pi_B\circ(\alpha,\beta)$, es decir, $(\alpha,\beta)$ iguala las flechas $A\times B\overset{f\circ\pi_A}{\underset{g\circ\pi_B}{\rightrightarrows}}Z$
        , y por la propiedad universal del igualador, existe una única flecha $X\xrightarrow{h}I\in\C$ tal que $e\circ h=(\alpha,\beta)$ y por lo tanto el siguiente diagrama conmuta:
        \[\begin{tikzcd}
            X \arrow[rrd, "\alpha", bend left] \arrow[rdd, "\beta"', bend right] \arrow[rd, "{h}", dashed] &                                                  &                  \\
                                                                                                                            & {I} \arrow[r, "\pi_A\circ e"] \arrow[d, "\pi_B\circ e"'] & A \arrow[d, "f"] \\
                                                                                                                            & B \arrow[r, "g"']                                & Z               
        \end{tikzcd}\]
        Y por la unicidad de $h$, entonces \begin{tikzcd}
            {I} \arrow[r, "\pi_A\circ e"] \arrow[d, "\pi_B\circ e"'] & A \arrow[d, "f"] \\
            B \arrow[r, "g"']                                & Z               
        \end{tikzcd} es un producto fibrado de $f$ y $g$ en $\C$.
\end{proof}

Dejamos como ejercicio al lector probar la versión dual de este teorema; aconsejamos intentar enunciarlo, pero de igual forma queda enunciado en la parte de ejercicios sugeridos.

\subsection{Ejercicios sugeridos de esta sección}
\begin{ex}[Secciones, Retractos e isomorfismos]
    Sea $\C$ una categoría y $A\xrightarrow{f}B\in\C$ una flecha. $f$ es un isomorfismo si y solo si $f$ es una sección y un retracto.
\end{ex}

\begin{ex}[Homomorfismo de anillos que es monomorfismo y epimorfismo y no es biyectivo]
    Demuestra que para cualquier anillo $R\in\Ring$ y culaes quiera homomorfismos de anillos $\Q\overset{\alpha}{\underset{\beta}{\rightrightarrows}}R\in\Ring$. Si $\alpha(z)=\beta(z)$ para todo $z\in\Z\subseteq\Q$, 
    entonces $\alpha=\beta$.
\end{ex}

\begin{ex}[Categoría opuesta]
    Dada una categoría $\C$, demuestra que la regla de composición definida para $\C^\op$ es asociativa y que cada objeto tiene su flecha identidad.
\end{ex}

\begin{ex}[Dualidad entre sección y retracto]
    Sea $\C$ una categoría. Demuestra que una flecha $A\xrightarrow{f}B\in\C$ es una sección si y solo si $B\xrightarrow{f}A\in\C^\op$ es un retracto.
\end{ex}

\begin{ex}[Cokernel de un homomorfismo]
    Demuestre que para un homomorfismo de grupos $G\xrightarrow{f}H\in\Ab$ entre grupos abelianos, y considerando el grupo cociente $\Cokr(f):=H/\im(f)$. Entonces para cualquier grupo abeliano $K$ y
    cualesquiera homomorfismos de grupos $H\xrightarrow{g}K$ y $\{e\}\xrightarrow{0}K$ tal que $g\circ f=0$, entonces existe un único homomorfismo $\Cokr(f)\xrightarrow{\alpha}K$ tal que 
    el siguiente diagrama conmuta:
    \[\begin{tikzcd}
                K &                                                        &                                                            \\
                  & {\Cokr(f)} \arrow[lu, "{\alpha}"', dashed]    & H \arrow[l, "\pi"'] \arrow[llu, "g"', bend right] \\
                  & \{e\} \arrow[u, "0"] \arrow[luu, "0", bend left] & G \arrow[u, "f"'] \arrow[l, "0"]                          
    \end{tikzcd}\]
    donde $\pi:H\to \Cokr(f)$ es la proyección canónica, que manda cada $h\in H$ a su clase lateral $h\im(f)\in \Cokr(f)$; $0$ es el homomorfismo trivial que manda todo elemento de su dominio
    al elemento neutro de su codominio.
\end{ex}

\begin{ex}[Teorema dual]
    Demuestre que si $\C$ es una categoría con coproductos y coigualadores, entonces $\C$ tiene coproductos fibrados.
\end{ex}

\section{Funtores}

Ahora que entendemos que es una categoría. Usemos el mismo espíritu de la teoría de categorías y pensemos en las categorías como objetos matemáticos.
Lo que queremos decir es que para estudiar categorías con ese mismo espíritu, sería necesario definir alguna noción de flechas entre categorías. 
Eso es justamente un funtor.

\begin{dfn}
    Sean $\A$ y $\B$ dos categorías. Definimos un \emph{funtor} $F:\A\to\B$ como una regla de asignación entre $\A$ y $\B$, de tal forma que:
    \begin{enumerate}
        \item A cada objeto $X\in\A$ le corresponde un solo objeto $F(X)\in\B$.
        \item A cada flecha $X\xrightarrow{f}Y\in\A$ le corresponde una sola flecha $F(X)\xrightarrow{F(f)}F(Y)\in\B$ tal que:
        \begin{itemize}
            \item $F(\id_X)=\id_{F(X)}$ para todo objeto $X\in\A$.
            \item $F(g\circ f)=F(g)\circ F(f)$ para toda flecha $Y\xrightarrow{g}Z\in\A$.
        \end{itemize}
    \end{enumerate}
\end{dfn}

Consideremos dos categorías $\A$ y $\B$ y un funtor $F:\A^\op\to\B$. Notemos que la regla de asignación de $F$ también es una regla de asignación entre $\A$ y $\B$,
pues a cada objeto $A\in\Obj(\A)=\Obj(\A^\op)$ le corresponde el objeto $F(A)\in\B$, y cada flecha $A\xrightarrow{f}B\in\A$ es una flecha $B\xrightarrow{f}A\in\A^\op$ que le corresponde
la flecha $F(B)\xrightarrow{F(f)}F(A)\in\B$. De esta manera $F$ casi es un funtor de $A$ a $B$, con la única diferencia de que cambia la dirección de las flechas.
En este sentido podemos decir que es otro tipo de funtor:

\begin{dfn}[Funtor contravariante]
    Sean $\A$ y $\B$ dos categorías. Definimos un \emph{funtor contravariante} $F:\A\to\B$ como una regla de asignación entre $\A$ y $\B$, de tal forma que:
    \begin{enumerate}
        \item A cada objeto $X\in\A$ le corresponde un solo objeto $F(X)\in\B$.
        \item A cada flecha $X\xrightarrow{f}Y\in\A$ le corresponde una sola flecha $F(Y)\xrightarrow{F(f)}F(X)\in\B$ tal que:
        \begin{itemize}
            \item $F(\id_X)=\id_{F(X)}$ para todo objeto $X\in\A$.
            \item $F(g\circ f)=F(f)\circ F(g)$ para toda flecha $Y\xrightarrow{g}Z\in\A$.
        \end{itemize}
    \end{enumerate}
\end{dfn}

Por lo comentado previo a la definición anterior, queda claro que es equivalente tener un funtor $F:\A^\op\to\B$ a tener un funtor contravariante $F:\A\to\B$.
Dejamos como ejercicio al lector hacer la prueba de que también es equivalente tener un funtor $F:\A\to\B^\op$ a tener un funtor contravariante $F:\A\to\B$.
Solo queremos aclarar que un funtor que no es contravariante, también se le llama \emph{funtor covariante}, En general no usaremos ese término,
y asumieremos que un funtor es covariante a menos que se especifique lo contrario, es decir, que se diga que es un funtor contravariante; finalmente tenemos
una manera de mirar equivalentemente funtores contravariantes y covariantes.

Recordando que en el ejemplo \ref{Categorías concretas} mencionamos algunas categorías concretas, pero no definimos formalmente que era una categoría concreta.
Antes de esa definición, necesitamos definir un tipo de funtor, entonces veamos algunos cuantos.

\begin{dfn}[Tipos de funtores]
    Sea $F:\A\to\B$ un funtor entre las categorías $\A$ y $\B$. Decimos que:
    \begin{itemize}
        \item $F$ es \emph{fiel} si para cuales quiera dos flechas $X\overset{f}{\underset{g}{\rightrightarrows}}Y\in\A$, si $F(f)=F(g)$ entonces $f=g$.
        \item $F$ es \emph{pleno} si para cualquier par de objetos $X,Y\in\A$, y cualquier flecha $F(X)\xrightarrow{g}F(Y)\in\B$, existe una flecha 
        $X\xrightarrow{f}Y\in\A$ tal que $F(f)=g$.
        \item $F$ es esencialmente sobreyectivo si para cualquier objeto $Y\in\B$, existe un objeto $X\in\A$ tal que $F(X)\cong Y$.
    \end{itemize}
\end{dfn}

Con esto podemos decir formalmente que una categoría $C$ se dice \emph{concreta} si existe un funtor $U:\C\to\Con$ fiel; a este funtor se le llama 
\emph{funtor de olvido}. Veamos ya algunos ejemplos de funtores.

\begin{ej}[Funtores que \textquotedblleft no hacen nada\textquotedblright]
    \begin{itemize}
        \item El funtor identidad $\id_\C:\C\to\C$ para cualquier categoría $\C$. Por supuesto este funtos asigna a cada flecha $X\xrightarrow{f}Y\in\C$ 
        la misma flecha $X\xrightarrow{f}Y\in\C$. Es claro que satisface las condiciones de un funtor.
        \item De las categorías concretas que mencionamos en el ejemplo \ref{Categorías concretas}, cada una tiene su funtor de olvido, que simplemente 
        asigna a cada objeto (grupo, anillo, espacio topológico, vectorial, etc.) su conjunto subyacente, y cada a morfismo (homomorfismo de grupos, anillos,
        funciones continuas, etc.) la función que va entre los conjuntos subyacentes. Es claro que estos funtores son fieles, que son justo lo que las hace
        categorías concretas. En general podemos decir que un funtor que olvida alguna estructura se podría llamar funtor que olvida; algunos funtores
        que olvidan podrían ser: de $\Grp$ a $\Mon$ (Un grupo es un monoide, y solo se olvida de que es grupo) , de $\Ring$ a $\Ab$ (Olvida el producto del
        anillo), de $\Ab$ a $\Grp$ (olvida que es un grupo abeliano), de $\Ring$ a $\Mon$ (olvida la operación aditiva del anillo), etc.
    \end{itemize}
\end{ej}

\begin{ej}[Funtores que \textquotedblleft dan estructura\textquotedblright]
    Ahora veamos algunos ejemplos de funtores que asignan un objeto con estructura a un conjunto.
    \begin{itemize}
        \item El funtor $F:\Con\to\Grp$ que asigna a cada conjunto $X\in\Con$ el grupo libre generado por $X$. $F(X):=\{\text{Cadenas finitas de elementos de }X\}$,
        es decir, los elementos de $F(X)$ se ven de la forma $x_1x_2x_3\ldots x_n$ donde $x_i\in X$ y $n\geq 0$, si $n=0$ se forma la cadena vacía. Y a cada
        función (flecha) $f:X\to Y\in\Con$ le asigna el homomorfismo de grupos $F(f):F(X)\to F(Y)$ que manda cada cadena $x_1x_2\ldots x_n\in F(X)$ a la cadena
        $f(x_1)f(x_2)\ldots f(x_n)\in F(Y)$.
        \item Para el caso de dar estructura de espacio topológico, tenemos dos funtores $I,D:\Con\to\Top$, $D$ asigna a cada conjunto $X\in\Con$ el espacio 
        topológico discreto $(X,\mathcal{P}(X))$, donde $\mathcal{P}(X)$ es el conjunto potencia de $X$; e $I$ asigna a cada conjunto $X\in\Con$ el espacio
        topológico indiscreto $(X,\{\emptyset,X\})$. Ambos funtores son la identidad en la asignación de flechas, pues cualquier función entre dos espacios 
        (in)discretos es una función continua.
        \item Consideremos un campo $K$, el funtor $K^{()}:\Con\to\Vect_K$, que asigna a cada conjunto $X\in\Con$ el espacio vectorial $K^{(X)}$ de
        funciones $f:X\to K$ tales que $f(x)\not=0$ para un número finito de $x\in X$. Para hacer la asignación de flechas, es necesario mencionar la inyección
        $X$ en $K^{(X)}$, dada por las funciones de Dirichlet de $\delta_x:X\to K$ que dan la identidad del campo $1$ cuando mapean a $x$ y cero en otro caso.
        De esta manera a cada función $f:X\to Y\in\Con$ induce una función entre las bases de $K^{(X)}$ y $K^{(Y)}$, y por el teorema de extensión lineal,
        esta función induce una única transformación lineal que será la asignación de flechas del funtor $K^{()}$.
    \end{itemize}
\end{ej}

Si el lector está familiarizado con teoría de grupos y espacios vectoriales, recomendamos comprobar que que los funtores $F:\Con\to\Grp$ y $K^{()}:\Con\to\Vect_K$ 
satisfacen las condiciones de conservar identidades y abrir composiciones de flechas.

\begin{ej}[Funtores en base a objetos]\label{funtores en base a objetos}
    Consideremos una categoría $\C$ y un objeto $A\in\C$.
    \begin{itemize}
        \item Supongamos que $A$ tiene producto con cualquier otro objeto de $\C$, entonces podemos definir un funtor $A\times-:\C\to\C$ que asigna a cada
        objeto $X\in\C$ el producto $A\times X\in\C$, y a cada flecha $X\xrightarrow{f}Y\in\C$ le asigna la flecha $\widetilde{f}:=(\pi_A,f\circ\pi_X):A\times X\to A\times Y\in\C$ dada
        por la propiedad universal del producto. Similarmente, si $A$ tiene coproducto con cualquier otro objeto de $\C$, tenemos el funtor $A+-:\C\to\C$.
        Sería un buen ejercicio para el lector comprobar que en efecto la asignación en flechas de estos dos funtores, $A\times-:\C\to\C$ y 
        $A+-:\C\to\C$, satisfacen las condiciones de un funtor.
        \item Decimos que una categoría $\C$ es \emph{localmente pequeña} si para cualesquiera dos objetos $A,B\in\C$, la colección $\C(A,B)$, de flechas
        entre $A$ y $B$, es un conjunto. En este caso, tenemos el funtor $\C^A:\C\to\Con$ que asigna a cada objeto $X\in\C$ el conjunto $\C(A,X)$, y a cada 
        flecha $X\xrightarrow{f}Y\in\C$ le asigna la función $f^A:\C(A,X)\to\C(A,Y)$ que manda cada flecha $A\xrightarrow{g}X\in\C$ a la flecha 
        $A\xrightarrow{f\circ g}Y\in\C$. De igual forma, podemos definir el funtor contravariante $\C_A:\C\to\Con$ que asigna a cada objeto $X\in\C$ el conjunto $\C(X,A)$, 
        y a cada flecha $X\xrightarrow{f}Y\in\C$ le asigna la función $f_A:\C(Y,A)\to\C(X,A)$ que manda cada flecha $Y\xrightarrow{g}A\in\C$ a la flecha
        $X\xrightarrow{g\circ f}A\in\C$. A estos funtores se les llama \emph{funtores hom}, uno es covariante y otro contravariante.
        \item Un ejemplo particular en la categoría de conjuntos es el funtor contravariante $\Con_2:\Con\to\Con$, donde $2=\{0,1\}$. Por otro lado, veamos que 
        $\mathcal{P}:\Con\to\Con$ es un funtor contravariante mandando cada conjunto $X$ a su conjunto potencia $\mathcal{P}(X)$, y cada función 
        $X\xrightarrow{f}Y\in\C$ a la función $\mathcal{P}(Y)\xrightarrow{f^{-1}}\mathcal{P}(X)$ que asigna a cada subconjunto de $Y$ su preimagen bajo $f$.
        Es bien sabido que hay una biyección entre $\Con(X,2)$ y $\mathcal{P}(X)$, dada por la función característica de cada subconjunto de $X$; 
        $X\supseteq A\longleftrightarrow \chi_A\in \Con(X,2)$, donde $\chi_A$ mapea $x\in X$ a $1\in 2$ si y solo si $x\in A$.
        Como ejercicio pruebe que para cualquier función $X\xrightarrow{f}Y\in\Con$ y subconjunto $A\subseteq Y$ se tiene que $\chi_A\circ f=\chi_{f^{-1}(A)}$.
    \end{itemize}
\end{ej}

\begin{ej}[Morfismos que son funtores]
    \begin{itemize}
        \item Como mencionamos en el ejemplo \ref{Monoides}, cada monoide se puede ver como una categoría, donde cada elemento del monoide es una flecha de un
        único objeto. De esta manera los homomorfismos de monoides son esencialmente funtores que en objetos la asignación es trivial, y en flechas es justo la 
        regla del homomorfismo, que satisface justo lo que pedimos a un funtor.
        \item Como mencionamos en el ejemplo \ref{categoria delgada}, un conjunto parcialmente ordenado puede ser visto como categoría, entonces una función
        monótona es un funtor. La asignación en objetos es la regla de la función, y la asignación en flechas básicamente es trivial (pues entre cada objeto a lo
        mucho hay una flecha) y es coherente gracias a que la función es monótona, pues así garantiza que cuando tomas una flecha en el dominio, si existe una
        que asignarle en el codominio.
    \end{itemize}
\end{ej}

\begin{ej}\label{Funtor de abiertos}
    \begin{itemize}
        \item Recordando las categorías $\Top$ y $\Frm$ mencionadas en el ejemplo \ref{Categorías concretas}, tenemos un funtor contravariante $\mathcal O:\Top\to\Frm$ que 
        le asigna a cada espacio topológico $X$ la familia de sus conjuntos abiertos $\mathcal O(X)$, que en efecto es un marco con el orden de subconjunto, tiene 
        elemento tope ($X$ es abierto), tiene fondo ($\emptyset$ es abierto), tiene supremos arbitrarios (La unión de abiertos es abierta), tiene ínfimos finitos
        (la intersección finita de abiertos es abierta), y estos distribuyen. Por otro lado a cada función continua $f:X\to Y\in\Top$ le asigna la preimagen
        $f^{-1}:\mathcal O(Y)\to\mathcal O(X)$, la cual está bien definida pues una función continua por definición la preimagen de abiertos es abierta, y como la preimagen
        se comporta bien con las uniones e intersecciones, entonces es un morfismo de marcos. Que preserva identidades y composiciones (contravariantemente) son
        consecuencia de las propiedades de la preimagen.
        \item Consideremos un anillo con identidad $R\in\Ring$ y fijemos un número natural $n\in\N$, podemos asignarle el monoide $M_n(R)$ de matrices $n\times n$ 
        con entradas en $R$, con la operación de multiplicación de matrices. Y a cada homomorfismo de anillos $f:R\to S\in\Ring$ se le asigna el homomorfismo 
        de monoides $M_n(f):M_n(R)\to M_n(S)$ que manda a cada matriz con entradas en $R$ a la matriz con entradas en $S$ dada por la imagen de $f$ entrada 
        por entrada.
    \end{itemize}
\end{ej}

Con todo lo anterior se puede pensar que si tienes una categoría $\C$ y tienes una regla para asignar a cada objeto en $\C$ un objeto en otra categoría, digamos $\D$,
es posible encontrar una manera de asignar flechas de tal manera que esta asignación se convierta en un funtor. Pues eso no siempre es posible, pero cuando si lo es,
decimos que la asignación es funtorial. Veamos un ejemplo de una asignación que no es funtorial.

\begin{ej}[Una asignación no funtorial]
    Sabemos que a cada grupo $G\in\Grp$ le podemos asignar su centro que es un grupo abeliano $Z(G)\in\Ab$. Supongamos que esta asignación es funtorial, es decir,
    a cada homomorfismo de grupos $G\xrightarrow{\phi}H\in\Grp$ se le asigna el homomorfismo $Z(G)\xrightarrow{Z(\phi)}Z(H)\in\Ab$ de forma que satisface los axiomas
    de funtor, manda identidades a identidades, y respeta composiciones. Asumiendo esto, consideremos los grupos $\Z_2$, el grupo cíclico de dos elementos; $S_3$,
    el grupo de permutaciones de un conjunto de 3 elementos; y $A_3$ el grupo alternante, que es un subgrupo normal de $S_3$. Ahora consideremos los siguientes homomorfismos:
    \[\Z_2\xrightarrow{f}S_3\xrightarrow{\pi}S_3/A_3\xrightarrow{g}\Z_2\]
    donde $f$ mapea el $0\in\Z_2$ a la permutación $(1)\in S_3$ y el $1\in\Z_2$ a la permutación $(12)\in S_3$; $\pi$ es la proyección canónica a las 
    clases laterales; y $g$ mapea la clase $\overline{(1)}\in S_3/A_3$ al $0\in\Z_2$ y la clase $\overline{(12)}\in S_3/A_3$ al $1\in\Z_2$.
    
    Denotemos con $h$ a la composición de $g\circ\pi$, entonces tenemos que $\id_{\Z_2}=h\circ f$. Como asumimos que el centro es funtorial, tenemos que
    $Z(\id_{\Z_2})=Z(h\circ f)=Z(h)\circ Z(f)$. Ahora, considerando que $Z(S_3)=\{(1)\}$, es decir
    es el grupo trivial, entonces $Z(h)$ y $Z(f)$ no tienen otra opción que ser los homomorfismos triviales, lo que nos da una contradicción, pues significaría
    por un lado que $Z(id_{\Z_2})=\id_{\Z_2}$ (porque un funtor preserva identidades y $\Z_2$ es igual a su centro, por ser abeliano), y por otro lado $Z(\id_{\Z_2})$
    debe ser el homomorfismo trivial (al ser la composición de dos homomorfismos triviales); lo que significa que asignar el centro a un grupo no puede ser funtorial.
\end{ej}

Veamos algunas propiedades de los funtores.

\subsection{Propiedades básicas}
Veamos primero que los funtores se pueden componer

\begin{prop}
    Sean dos funtores $\A\xrightarrow{F}\B\xrightarrow{G}\C$, entonces la regla de asignación $G\circ F:\A\to\C$ definida como 
    $G\circ F(X\xrightarrow{f}Y)=G(F(X))\xrightarrow{G(F(f))}G(F(Y))$ para toda flecha $X\xrightarrow{f}Y\in\A$, es un funtor.
\end{prop}
\begin{proof}
    Sea una flecha $f:A\to B\in \A$, entonces $F(f):F(A)\to F(B)\in\B$ y $G(F(f)):G(F(A))\to G(F(B))\in\C$.
    Luego como $F$ y $G$ son funtores
    \begin{equation*}
    G\circ F(\id_A) =G(F(\id_A)) =G(\id_{F(A)}) =\id_{G(F(A))}=\id_{G\circ F(A)}.
    \end{equation*}

    Ahora sea $g:B\to C\in\A$, entonces
    \begin{eqnarray*}
    G\circ F(g\circ f) &=&G(F(g\circ f))=G(F(g)\circ F(f))=G(F(g))\circ G(F(f)) \\
    &=&(G\circ F(g))\circ(G\circ F(f))
    \end{eqnarray*}
    lo que prueba que $G\circ F$ es un funtor.
\end{proof}

La proposición anterior nos dice como está definida la composición de funtores. Dejamos para el lector el ejercico de probar que para cualquier funtor $\A\xrightarrow{F}\B$, se tiene que $F\circ\id_\A=\id_\B\circ F=F$.
Veamos ahora que la composición es asociativa.

\begin{prop}
    Sean tres funtores $\A\xrightarrow{F}\B\xrightarrow{G}\C\xrightarrow{H}\D$. Entonces $$H\circ(G\circ F)=(H\circ G)\circ F.$$
\end{prop}
\begin{proof}
    Simplemente veamos que dado $\bigstar\in\A$ (objeto o flecha) se tiene
    que
    \begin{eqnarray*}
    (H\circ G) \circ F(\bigstar)  &=&
    H\circ G(F(\bigstar) ) =H(G(F(\bigstar)))  \\
    &=&H(G\circ F(\bigstar))  \\
    &=&H\circ(G\circ F)(\bigstar) 
    \end{eqnarray*}
    lo que prueba la igualdad.
\end{proof}

Hemos probado que los funtores y su regla de composición satisfacen de cierta forma lo que pedimos para las flechas de una categoría, por lo que es natural preguntarse
¿Es posible hablar de la categoría de categorías?. No como tal, hay algunos detalles técnicos fundacionales por los que no es tán simple hablar de algo tan inmenso. Una manera de
tratar con este problema está en el libro Abstract and Concrete Categories The Joy of Cats escrito por Jiˇr´ıAd´amek, Horst Herrlich y George E. Strecker, donde se define la 
\emph{quasicategoría} $\CAT$ cuyos objetos son todas las categorías y las flechas son los funtores. No es nuestra intensión abordar este tema a fondo, 
solo dar una fuente para que el lector pueda investigar un poco más a fondo el tema. Sin embargo queremos hacer notar que de igual forma podemos
hablar de categorías que satisfacen propiedades universales (incial, final, productos, coproductos, etc...) e isomorfismos de categorías.


\subsection{Ejercicios sugeridos de esta sección}

\begin{ex}[Variante de funtor contravariante]
    Demuestre que es equivalente tener un funtor $F:\A\to\B^\op$ a tener un funtor contravariante $F:\A\to\B$.
\end{ex}

\begin{ex}[Funtores libres]
    Verifique que los funtores $F:\Con\to\Grp$ y $K^{()}:\Con\to\Vect_K$, del ejemplo \ref{funtores libres}, preservan identidades y composición.
\end{ex}

\begin{ex}
    Dada una categoría $\C$ con productos y coproductos. Las asignaciones $A\times-:\C\to\C$ y $A+-:\C\to\C$ son funtoriales. 
\end{ex}

\begin{ex}
    Sea una función $X\xrightarrow{f}Y\in\Con$ y un subconjunto $A\subseteq Y$, entonces
    \[\chi_A\circ f=\chi_{f^{-1}(A)}\]
    donde cada $chi_\cdot$ son las funciones características mencionadas en el ejemplo \ref{funtores en base a objetos}.
\end{ex}

\begin{ex}
    Demuestre que para cualquier funtor $\A\xrightarrow{F}\B$, se tiene que $F\circ\id_\A=\id_\B\circ F=F$.
\end{ex}

\section{Transformaciones naturales}




\section{Límites y colímites}

\begin{dfn}[Diagrama]
    Sean $\C,I$ dos categorías, con $I$ una categoría pequeña. Un \emph{diagrama en} $\C$ \emph{de forma} $I$ es simplemente un funtor $D:I\to\C$.   
\end{dfn}

\begin{dfn}
    Sea $D:I\to\C$ un diagrama en $\C$. Un \emph{cono} de $D$ consta de un objeto $X\in\C$, al que llamamos \emph{vértice del cono}, junto a una familia
    de flechas $\{X\xrightarrow{f_i}D(i) \ |\ i\in\Obj(I)\}\subseteq\C$ tales que para cada flecha $i\xrightarrow{u}j\in I$ el siguiente diagrama conmuta
    \[\begin{tikzcd}
        & X \\
        {D(i)} && {D(j)}
        \arrow["{f_i}"', from=1-2, to=2-1]
        \arrow["{f_j}", from=1-2, to=2-3]
        \arrow["{D(u)}"', from=2-1, to=2-3]
    \end{tikzcd}\]
\end{dfn}

\begin{dfn}
    Sea $D:I\to\C$ un diagrama en $\C$. Un \emph{límite} de $D$ es un cono $(L\xrightarrow{l_i}D(i))_{i\in I}$ tal que para cualquier otro cono $(X\xrightarrow{f_i}D(i))_{i\in I}$
    existe una única flecha $X\xrightarrow{h}L\in\C$ tal que el siguiente diagrama conmuta
    \[\begin{tikzcd}
        X && L \\
        & {D(i)}
        \arrow["h", from=1-1, to=1-3]
        \arrow["{f_i}"', from=1-1, to=2-2]
        \arrow["{l_i}", from=1-3, to=2-2]
    \end{tikzcd}\] para todo $i\in I$
\end{dfn}

Si el límite de un diagrama $D$ existe, normalmente denotamos al vértice de su cono límite con $\lm(D)$.

Conceptos duales...

Obervaciones entre las propiedades universales mencionadas antes y (co)límites...

\begin{dfn}
    Decimos que una categoría $\C$ es \emph{completa}, si para cada categoría pequeña $I$, todo diagrama $D:I\to\C$ tiene límite en $\C$
\end{dfn}


\begin{thm}
    Sean $\A$ y $\C$ dos categorías. Si $\C$ es completa, entonces $[\A,\C]$ es completa
\end{thm}
\begin{proof}
    PENDIENTE
\end{proof}

\section{Lema de Yoneda}

Introducción sobre el lema de Yoneda, enunciación y demostración...

\section{Adjunciones}

En general, dadas dos categorías y dos funtores entre ellas, es muy difícil saber a priori cuando estos funtores forman una equivalencia, por eso se buscó las condiciones mínimas que se pueden exigir para tratar a dichas categorías 'como si fueran equivalentes'. De esta idea surgió el concepto de adjunción.
\begin{dfn}
	Dadas dos categorías $\mathcal{C,D}$ y dos funtores $F:\mathcal{C\to D}$, $G:\mathcal{C\to D}$, diremos que $F$ es el adjunto izquierdo de $G$ (o que $G$ es el adjunto derecho de $F$), denotado $F\dashv G$, si existe un isomorfismo natural $\Hom_\mathcal{D}(F(A),B)\simeq \Hom_\mathcal{C}(A,G(B))$ para cualesquiera objetos $A$ en $\mathcal{D}$ y $B$ en $\mathcal{C}$, es decir, para cualesquiera morfismos $f:A'\to A$
	en $\mathcal C$ y $g:B\to B'$ en $\mathcal D$,
	el siguiente diagrama es conmutativo:
	\[
	\begin{tikzcd}
		\Hom_\mathcal{D}(FA,B)
		\ar[d,"g\circ-\circ Ff"']
		\ar[r,"\sim",shift left]
		& \Hom_\mathcal{C}(A,GB) 
		\ar[d,"Gg\circ-\circ f"]
		\ar[l,"\sim",shift left] \\
		\Hom_\mathcal{D}(FA',B')
		\ar[r,"\sim",shift left]
		& \Hom_\mathcal{D}(A',GB')
		\ar[l,"\sim",shift left]
	\end{tikzcd}
	\]
\end{dfn}
La definición anterior quiere decir que hay un iso natural entre los funtores.

\begin{obs}\label{adjcomp}
	Supongamos que tenemos $F\dashv G$ funtores adjuntos entre $\mathcal{C}$ y $\mathcal{D}$, dadas flechas $(FA\xrightarrow p B)\in\mathcal{D}$ y $(A\xrightarrow q GB)\in\mathcal{C}$ les corresponden flechas únicas $(A\xrightarrow{\Bar{p}}GB)$	y $(FA\xrightarrow{\Bar{q}}B)$,
	respectivamente, bajo el isomorfismo de adjunción. Claramente $\Bar{\Bar{p}}=p$ y $\Bar{\Bar{q}}=q$. As\'i:
	\begin{enumerate}
		\item
		Tomando $f=\id_A:A\to A$ y cualquier $g:B\to B'$ en $\mathcal{C}$, $(Gg)\bar p=\ol{gp}$.
		\item
		Tomado $g=\id_B:B\to B$ y cualquier $f:A\to A'$ en $\mathcal{D}$, $\bar q(Ff)=\ol{qf}$.
	\end{enumerate}
\end{obs}

Ahora, notemos que si $G: \mathcal{D} \longrightarrow  \mathcal{C}$ es un funtor y suponemos que  $F,F': \mathcal{C} \longrightarrow  \mathcal{D}$ son adjuntos izquierdo de $G$, entonces 
\begin{center}
	$\Hom_\mathcal{D}(F(A),B)\cong \Hom_\mathcal{C}(A,G(B)) \cong \Hom_\mathcal{D}(F'(A),B)$.
\end{center}
natural en $B \in \mathcal{D}$. Esto implica que hay un isomorfismo 
\[\Hom_\mathcal{D}(F(A),F'(A))\simeq \Hom_\mathcal{D}(F'(A),F'(A)).\]
Definimos la siguiente transformaci\'on natural $F\xRightarrow{\alpha}F'$ como $\alpha_A=\overline{\id_{F'(A)}}$ para cada objeto $A$ de $\mathcal{C}$. Se deja al lector checar que $\alpha$ es un isomorfismo. Por lo tanto cualquier adjunto izquierdo es único salvo isomorfismos y dualmente se demuestra que los adjuntos derechos también son único salvo isomorfismos.

\begin{ej}
	\begin{itemize}
		\item En álgebra surge mucho el fen\'omeno de ``generar'' y ``olvidar'', en otras palabras, a partir de un conjunto podemos generar una estructura algebraica y dada una estructura algebraica podemos olvidar alguna operaci\'on. Para ejemplificar consideremos la categoría de espacios vectoriales $\Vect_k$ sobre un campo $k$ y la categoría de conjuntos $\Con$. Definimos los siguientes funtores $F:\Con\to\Vect_k$ y $U:\Vect_k\to\Con$ de la siguiente manera:
		Dado un $k$-espacio vectorial $V$, $U(V)$ es solo el conjunto subyacente $V$ y cualquier transfomaci\'on lineal $g:V\to W$, $U(g)$ la considera solo como funci\'on. Por otro lado, el funtor $F$ toma un conjunto $X$ y $F(X)$ es el $k$ espacio vectorial cuya base es $X$, o de otra manera, la suma directa indicada en $X$ de copias de $k$. Dada una funci\'on $f:X\to Y$, $F(f)$ es la transformaci\'on lineal tal que 
		\[(k_x)_{x\in X}\mapsto (k_y)_{y\in Y}=\begin{cases}
			k_x \; \text{si }\; y=f(x) \\
			0 \;\text{ en otro caso.}
		\end{cases}\]
		El lector puede revisar que efect\'ivamente $F$ y $U$ son funtores. M\'as a\'un, $F\dashv U$. Para mostrar esto, tomamos $X\in\Con$, $V\in\Vect_k$, y una flecha $g\in\Hom_{\Vect_k}(F(X),V)$. Definimos la funci\'on $\Bar{g}\in\Hom_{\Con}(X,U(V))$, como $\Bar{g}(x)=g(e_x)$ para cada $x\in X$, donde 
		\[e_y:=(k_x)_{x\in X}=\begin{cases}
			1 \;\text{si } x=y \\
			0 \;\text{ en otro caso.} 
		\end{cases}\]
		Ahora sea $f\in\Hom_{\Con}(X,U(V))$ y definimos $\Bar{f}:F(X)\to V$ como 
		
		\[\Bar{f}((k_x)_{x\in X})=\sum_{x\in X} k_xf(x).\]
		
		Por lo tanto si tomamos doble barra, tenemos que 
		\[\Bar{\Bar{g}}((k_x)_{x\in X})=\sum_{x\in X} k_x\Bar{g}(x)=\sum_{x\in X} k_xg(e_x)=g\left(\sum_{x\in X}k_xe_x\right)=g((k_x)_{x\in X}).\]
		
		Por otro lado,
		
		\[\Bar{\Bar{f}}(x)=\Bar{f}(e_x)=f(x).\]
		Esto nos da el isomorfismo de adjunci\'on entre $F$ y $G$. Se le deja lector verificar que en efecto este isomorfismo es natural.
		
		
		\item Un an\'alisis similar se puede hacer con cualquier otra estructura algebraica como grupos, anillos, campos, etc\'etera.
		
		\item
		Tomando los grupos abelianos $\Z-\mathrm{Mod}$, existe una adjunción con $\Grp$, $F\dashv U$, donde $U$ es el functor inclusión (olvida que es abeliano), y $F$ asigna a cada grupo $G$ su respectiva 'abelianización', dada por $G/[G,G]$,
		donde $[G,G]=\langle xyx^{-1}y^{-1}\mid x,y\in G\rangle$
		es el subgrupo conmutador.
		
		\item Dado un conjunto $B$, tenemos el funtor producto 
		\begin{equation*}
			{-}\times B:\Con\to \Con
		\end{equation*}
		es decir, el funtor que a cada conjunto $A$ le asocia el producto cartesiano $A\times B$. Este funtor, tiene un adjunto derecho 
		\begin{equation*}
			({-})^B:\Con\to \Con
		\end{equation*}
		llamado exponenciaci\'on, que a cada conjunto $A$, le asocia su conjunto de funciones $C^B:=\Hom_{\Con}(B,C)$. El isomorfismo de adjunci\'on se construye de la siguiente manera: Tomemos una funci\'on $f\in\Hom_{\Con}(A\times B, C)$. Definimos la siguiente funci\'on $\Bar{f}\in \Hom_{\Con}(A,C^B)$ como: para cada $a\in A$, tenemos una funci\'on $\Bar{f}(a):B\to C$ definida como $\Bar{f}(a)(b)=f(a,b)$. Rec\'iprocamente, si $g\in\Hom_{\Con}(A,C^B)$, definimos $\Bar{g}:A\times B\to C$ como $\Bar{g}(a,b)=g(a)(b)$. Directamente vemos que 
		\[\Bar{\Bar{f}}(a,b)=\Bar{f}(a)(b)=f(a,b)\]
		y 
		\[\Bar{\Bar{g}}(a)(b)=\Bar{g}(a,b)=g(a)(b).\]
	\end{itemize}
\end{ej}

\bigskip

Sea
\begin{center}
	\begin{tikzcd}
		\mathcal{C} \arrow[dd, "F"', bend right] \\
		\dashv \\
		\mathcal{D} \arrow[uu, "G"', bend right]
	\end{tikzcd}
\end{center}
una adjunci\'on con isomorfismo natural $\Hom_\mathcal{D}(FA,B)\simeq\Hom_\mathcal{C}(A,GB)$. Usando este isomorfismo natural, para cada objeto $A$ y para cada $B$ tenemos dos mofismos distinguidos $\eta_A$ y $\varepsilon_B$ definidos como:
\begin{align*}
	\overline{\left(F(A)\overset{1_{FA}}{\longrightarrow}F(A)\right)} & = \left(A\underset{\eta_A}{\longrightarrow}GFA\right)\\
	\left(FGB\underset{\varepsilon_B}{\longrightarrow}B\right) & = \overline{\left(G(B)\overset{1_{GB}}{\longrightarrow}G(B)\right)}
\end{align*}
y que son naturales en $A$ y en $B$ respectivamente. Por lo tanto tenemos transformaciones naturales
\begin{align*}
	\eta_\bullet: 1_\mathcal{A}\longrightarrow GF \\
	\varepsilon_\bullet: FG\longrightarrow 1_\mathcal{B}
\end{align*}
que son llamadas la \textbf{unidad} y la \textbf{co-unidad} de la adjunción.

\begin{lem}\label{lemmatriangulo}
	Sea $F\dashv G$ una adjunci\'on entre las categor\'ias $\mathcal{C}$ y $\mathcal{D}$. Dados $f\in\Hom_\mathcal{D}(F(A),B)$ y $g\in \Hom_\mathcal{C}(A,G(B))$ entonces los siguientes diagramas conmutan
	\[
	\begin{tikzcd}
		F(A) \arrow[r, "F(g)"] \arrow[rd, "\Bar{g}"'] & FG(B) \arrow[d, "\varepsilon_B"] & A \arrow[r, "\eta_A"] \arrow[rd, "\Bar{f}"'] & GF(A) \arrow[d, "G(f)"] \\
		& B & & G(B).                            
	\end{tikzcd}
	\]
\end{lem}

\begin{proof}
	Para el primer diagrama tenemos que $\varepsilon_{B}=\overline{\id_{GB}}$ por definici\'on. As\'i que por la Observaci\'on \ref{adjcomp}.2. se tiene que
	\[\varepsilon_{B}\circ F(g)=\overline{\id_{GB}}\circ F(g)=\overline{\id_{GB}\circ g}=\Bar{g}.\]
	
	Para el segundo diagrama, $\eta_A=\overline{\id_{FA}}$. Por la Observaci\'on \ref{adjcomp}.1. se tiene que
	\[G(f)\circ \eta_A=G(f)\circ\overline{\id_{FA}}=\overline{f\circ \id_{F(A)}}=\Bar{f}.\]
\end{proof}


\begin{prop}\label{identidadestraingulares}
	Dada un adjunci\'on $F\dashv G$, los siguientes diagramas de funtores conmutan
	
	
	\[
	\begin{tikzcd}
		F \arrow[r, "F\eta"] \arrow[rd, "1_F"'] & FGF \arrow[d, "\varepsilon F"] & G \arrow[r, "\eta G"] \arrow[rd, "1_G"'] & GFG \arrow[d, "G\varepsilon"] \\
		& F & & G.                            
	\end{tikzcd}
	\]
\end{prop}

\begin{proof}
	La componente de la transformaci\'on natural $\varepsilon F\circ F\eta$ es la composici\'on 
	\[\varepsilon_{FA}\circ F(\eta_A):F(A)\to FGF(A)\to F(A),\]
	y hay que demostrar que cada una de estas componentes es la identidad en $A$. Usando el primer diagrama del Lema \ref{lemmatriangulo} para el morfismo $\eta_A:A\to FG(A)$, se tiene que $\varepsilon_{FA}\circ F(\eta_A)=\overline{\eta_A}=\overline{\overline{\id_A}}=\id_A$.
	
	La conmutatividad del otro diagrama se prueba de manera an\'aloga.
\end{proof}

\subsection[Monadas una pequeña introducción]{Monadas una pequeña introducción}\label{SECMON}


Esta mini sección tiene el objetivo de introducir otra faceta de las adjunciones, empecemos con una situación de abstracta sin sentido.


Dado un funtor $T\colon \EuScript{C}\rightarrow\EuScript{C}$ quisiéramos (por que no) factorizarlo es decir, queremos encontrar una categoría $\D$ y funtores $F\colon \C\rightarrow\D$,
y $G\colon \D\rightarrow \C$ tal que: \[T=GF\;\;\;\; F\dashv G.\]

Esta factorización nos da una \emph{presentación universal de } $T$ donde universal quiere decir \emph{libre}
para hallar tal descomposición universal necesitamos desarrollar un poco mas de teoría, primero algunos ejemplos de posibles funtores a factorizar:

\begin{itemize}
\item[(1)]
Sea $M\colon \Con\rightarrow\Con$ el funtor que asigna a cada conjunto $A$ su conjunto de todas las posibles palabras que se pueden formar con $A$, es decir que tiene como alfabeto
a $A$, $M$ es flechas esta dado en \emph{generadores} es decir, dad cualquier función $f\colon A\rightarrow B$ sea $Mf\colon MA\rightarrow MB$ dada por \[Mf(a_{1}\ldots a_{n})=f(a_{1})\ldots f(a_{n}).\]

Casos interesantes para este $M$ es cuando $M$ es un tipo de monoide.


\item[(2)]
Sea $\Pos$ la categoría de conjuntos parcialmente ordenados, denotemos por 
$\mathcal{P}_{fin}(A,\leq)=\{\text{ subconjuntos finitos de } A\}$ entonces para cualesquiera $C,D\in\mathcal{P}_{fin}(A,\leq)$

denotemos:

\begin{enumerate}
\item[(i)] $C\preceq D$ si y solo si $\downarrow C\subseteq\downarrow D$.
\item[(ii)] $C\preceq D$ si y solo si $\uparrow C\supseteq\uparrow D$.
\item[(iii)]  $C\preceq D$ si y solo si $\downarrow C\subseteq\downarrow D$ y $\uparrow C\supseteq\uparrow D$.
\end{enumerate}
Esto da lugar a tres funtores $\mathcal{P}_{fin}(\_)\colon\Pos\rightarrow\Pos$ con cada una de las decoraciones anteriores, es decir con cada uno de los ordenes parciales.

\item[(3)]
Denotemos por $\D\C\Pos$ la categoría de \emph{conjuntos parcialmente ordenados completamente dirigidos}, es decir, $A\in  \D\Pos$ si:

\begin{enumerate}
    \item[(i)] $A$ es un conjunto parcialmente ordenado.
    \item[(ii)] Cada subconjunto $D\subseteq A$ \emph{dirigido} tiene supremo en $A$. 
\end{enumerate}
Fabricamos el siguiente funtor, $(\_)_{\perp}\colon\D\C\Pos\rightarrow \D\C\Pos $ donde el "elemento" $\perp$ esta fijo, entonces este funtor manda a cada copo directamente completo digamos,

$D$ a este mismo solo que se adjunta $\perp$ como elemento menor, la acción en flechas es la obvia, $f\colon A\rightarrow B$ entonces \[\begin{cases} f_{\perp}(a)
    a & a\in D\hookrightarrow D_{\perp} \\
    \perp & a=\perp
  \end{cases}\]
  \item[(4)]
  \begin{enumerate}
 \item[(a)] Consideremos una de nuestras categorías favoritas, $Top$ la categoría de espacios topológicos junto con funciones continuas,
fijemos nuestra atención en la subcategoría reflexiva de espacios topológicos $T_{0}$, denotemos por $Top_{0}$ a esta subcategoría plena, 
 Para cada $S\in Top_{0}$ sea $\mathcal{C}S$ los cerrados de $S$ y consideramos los siguientes conjuntos \[\lozenge U=\{X\in\mathcal{C}S\mid X\cap U\neq\emptyset\}\]
sea \[\Omega S^{\lozenge }=\langle\lozenge U\mid U\in\Omega S \rangle \] la topología generada por estos conjuntos, entonces tenemos el funtor $(\_)^{\lozenge}\colon Top_{0}\rightarrow Top_{0}$
que manda a cada espacio a $(\mathcal{C}S, \Omega S^{\lozenge })$ y en cada función continua, digamos $f\colon S\rightarrow T$ sea 
\[f^{\lozenge}\colon(\mathcal{C}S, \Omega S^{\lozenge })\rightarrow (\mathcal{C}T, \Omega T^{\lozenge })\] la función continua dada por \[f^{\lozenge}(X)=\bar{f[X]}\] Este ejemplos tiene algo que ver
con el hiperespacio de Vietoris y la monada diamante potencia.
\item[(b)]
En esta misma digamos, vertiente denotemos por $\EuScript{S}Top$ la categoría de espacios sobrios localmente compactos junto con funciones continuas como nuestras flechas,
sea $\mathcal{K}S=\{\text{ compactos de } S\}$ topologizamos este conjunto con: 
\[\square  U=\{X\in\mathcal{C}S\mid X\subseteq U\}\] y tomamos la topología generada por estos; 
\[\Omega S^{\square }=\langle\square U\mid U\in\Omega S\rangle \]
de manera análoga a la anterior tenemos un funtor $(\_)^{\square}\colon \EuScript{S}Top\rightarrow \EuScript{S}Top$ solo que en flechas se calcula como \[f^{\square}(X)=f[X]\]
  \end{enumerate}
\end{itemize}

\subsection[Mon]{Monadas} 
De acuerdo con nuestros problemas sin sentido, es decir, la factorización que queremos construir para $T\colon\C\rightarrow\C$, necesitamos que exista una adjunción $\D$ y funtores $F\colon \C\rightarrow\D$,
y $G\colon \D\rightarrow \C$ tal que: \[T=GF\;\;\;\; F\dashv G.\], entonces tenemos que asegurar que existan
  flechas \[\eta\colon id\_{\C}\rightarrow GF\text{ y }\;\; \epsilon\colon FG\rightarrow id_{\D}\] de tal manera que para todo $A\in\C$ y para todo $D\in\D$ los triángulos:
  \[
    \begin{tikzcd}[column sep=large, row sep=large]
      F \arrow[r, "F\eta"] \arrow[dr, "1_F"'] & FGF \arrow[d, "\varepsilon F"] \\
      & F
    \end{tikzcd}
    \qquad \text{y} \qquad
    \begin{tikzcd}[column sep=large, row sep=large]
      G \arrow[r, "\eta G"] \arrow[dr, "1_G"'] & GFG \arrow[d, "G\varepsilon"] \\
      & G
    \end{tikzcd}
    \]
    conmutan.

    En nuestro supuesto si tenemos que $T=GF$ entonces podemos asumir que tenemos una transformación natural $\eta\colon id_{\C}\rightarrow T$ ahora no podemos hablar de la transformación natural
    $\epsilon$ ya que no tenemos forma de hablar de $FG$ si solo tenemos la información $T=GF$, pero podemos hacer lo siguiente:

    \[G\epsilon F\colon GFGF=TT\rightarrow GF=T\] que se puede derivar directamente del triangulo  
    
    \begin{tikzcd}[column sep=large, row sep=large]
        FA \arrow[r, "F\eta_{A}"] \arrow[dr, "1_FA"'] & FGFA \arrow[d, "\varepsilon F_{A}"] \\
        & FA 
      \end{tikzcd}
y así podemos exigir la existencia de una transformación natural \[\mu\colon T^{2}\rightarrow T\]

Entonces para tener una definición adecuada de lo que es la factorización necesitamos tener una terna $(T,\eta,\mu)$. Es instructivo para el lector (neurótico) calcular $\eta$ y $\mu$ de nuestros ejemplos clave.

Sin embargo necesitamos aun mas información para tener una definición adecuada de lo que será nuestra monada.

Usando la supuesta adjunción de nuestro endofuntor $T\colon\C\rightarrow\C$ aplicando $G$ al triangulo con cateto   
\begin{tikzcd}[column sep=large, row sep=large]
    FA \arrow[r, "F\eta_{A}"] & FGFA  \\ 
\end{tikzcd}
obtenemos
    

\[\begin{tikzcd}[column sep=large, row sep=large]
   TA= GFA \arrow[r, "T\eta_{A}=GF\eta_{A}"] \arrow[dr, "1_GFA"'] & T^{2}A=GFGFA \arrow[d, "\mu_{A}=G\varepsilon F_{A}"] \\
    & GFA=TA 
  \end{tikzcd}
\]
  Y si en el triangulo para $G$ tomamos la componente con $D=FA$ uno obtiene:

\[\begin{tikzcd}[column sep=large, row sep=large]
   TA=GFA \arrow[r, "\eta_{TA}=\eta GFA"] \arrow[dr, "1_GFA"'] & T^{2}A=GFGFA \arrow[d, "\mu_{A}=G\varepsilon_{FA}"] \\
    & TA=GFA
  \end{tikzcd}\]
es decir, \[id_{TA}=\mu_{A}T\eta_{A}\text{ y } id_{TA}=\mu_{A}\eta_{TA}.\] es decir tenemos un tipo de \emph{conmutatividad} entre $T\eta_{A}$ y $\eta_{TA}$ y estas identidades
son las mas cercanas a nuestro caso original (recuerden que empezamos con una factorización chida).

Pero aun no tenemos todo lo que necesitamos, algo nos esta faltando, tenemos dos maneras naturales de ir desde $T^{3}A$ a $TA$, como el siguiente cuadro indica:

\[
\begin{tikzcd}[column sep=large, row sep=large]
  T^3A=GFGFGFA \arrow[r, "T\mu_{A}=GFG\epsilon_{FA}"] \arrow[d, "\mu TA=G\epsilon_FGFA"'] & T^2A \arrow[d, "\mu_{A}=G\epsilon_{FA}"] \\
  T^2A=GFGFA \arrow[r, "\mu_{A}=G\epsilon_{FA}"'] & TA=GFA
\end{tikzcd}
\]
este debemos demandar que conmute para cada $A$.

Todas las consideraciones anteriores siempre y cuando exista tal factorización codifican la siguiente definición:

\begin{dfn}\label{mona}
Una \emph{monada} $(T, \eta, \mu)$ en una categoría $\C$ consiste de los siguientes datos:\begin{itemize}
\item[(i)] Un funtor $T\colon \C\rightarrow\C$.
\item[(ii)] transformaciones naturales \[\eta\colon id_{\C}\rightarrow T\text{ y } \mu\colon T^{2}\rightarrow T.\] tales que:\[\mu_{A} T\eta_{A}=id_{TA}=\mu_{A}\eta_{TA}\text{ y } \mu_{A}T\mu_{A}=\mu_{A}\mu_{TA}.\]  
\end{itemize}


\end{dfn}
\begin{itemize}


\item[(i)]   Un ejemplo de lo anterior es considerar un conjunto parcialmente ordenado $A$ pensado como categoría tenemos que una monada sobre $A$ es un operador cerradura.
\item[(ii)] Algo mas elaborado es considerar para una categoría $\C$ digamos esencialmente pequeña, su categoría de endofuntores $[\C,\C]$, esta categoría es estrictamente monoidal cerrada ya que tiene un tensor, este
 dado por la composición de funtores, una monada para $\C$ no es nada mas que un monoide respecto a este tensor.  

\end{itemize}

Todas las observaciones anteriores dan lugar a la siguiente:

\begin{prop}\label{adjmon}
Si $(F,G,\eta,\epsilon)$ son datos de una adjunción con $F\colon \C\rightarrow\D$ entonces la terna $(GF, \eta,G\epsilon F)$ define una monada en $\EuScript{C}$.
\end{prop}   

Resulta (o debería de ser) que esta definición es suficiente para tener la factorización dada, para realmente justificar esto necesitamos un poco mas de trabajo.

\subsection{Ternas de Kleisli}\label{KLEI}

La definición de monada que dimos \ref{mona} se derivo de la definición de adjunción dada por las identidad triangulares, pero como sabemos existen tres formas clásicas de definir adjunción veamos que sucede con la dada por flechas universales.
requiere la existencia de un funtor $G\colon\D\rightarrow \C$ y una asignación (no se requiere funtorialidad) $F$ que solo actúe en objetos de $\C$ a objetos de $\D$, pero esto esta subyugado al hecho de que para cada objeto $A\in\C$ y cada flecha
$\eta_{A}\colon A\rightarrow GFA$ tal que para todo \[f\colon A\rightarrow GD\] con $D$ cualquier objeto de $\D$ exista una única flecha \[f^{+}\colon FA\rightarrow D\] de tal manera que \[f=Gf^{+}\eta_{A}.\], es decir tenemos
una función \[(\_)^{+}\colon\C(A,GD)\rightarrow\D(FA,D).\] en un dibujo:

\[
\begin{tikzcd}[column sep=large, row sep=large]
    A \arrow[r, "\eta_{A}", name=eta] \arrow[dr, "\forall f"', name=forallf] & GFA \arrow[d, dashed, "Gf^{+}", name=gfplus] \\
    & GD
\end{tikzcd}
\]
Esto en $\C$

  \[
\begin{tikzcd}
  FA \arrow[r, dashed, "\exists! f^+"] & D
\end{tikzcd}
\]
Esto otro en $\D$.

Como antes queremos dado $T$ encontrar $F$ y $G$ de tal manera que hagan la gracia de arriba y factoricen a $T$, la ventaja de esta definición es que no siendo $F$ un funtor a priori, podemos entonces empezar $T$ mandando objetos en objetos.

, y así se debe de pedir, que para cada objeto $A$ en $\C$ exista una flecha $\eta_{A}\colon A\rightarrow TA$. En este caso el operador de flechas $(\_)^{+}$ esta descrito como sigue: Si \[f\colon A\rightarrow GFB\] entonces
 \[Gf^{+}\colon GFA\rightarrow GFB\] y aquí se demanda que para toda \[f\colon A\rightarrow TB,\] exista una flecha \[f^{*}\colon TA\rightarrow TB\], En otras palabras el operador esta definido :
 \[\C(A,TB)\rightarrow\C(TA,TB)\] ahora desglosemos las ecuaciones que ligan la receta de $(T,\eta,(\_)^{*})$.
 Imitando un poco la universalidad de la flecha que nos da la definición de adjunción tenemos que pedir:
 \[f^{*}\eta_{A}=f.\] 
 Y entonces si $\eta_{A}\colon A\rightarrow GFA$ y por la propiedad universal debemos tener :\[Gid_{FA}\circ\eta_{A}=\eta_{A}=G\eta^{+}_{A}\circ\eta_{A}\]
  y por unicidad entonces obtenemos \[\eta^{+}_{A}=id_{FA}\] con esto debemos exigir a nuestra terna que \[\eta^{*}_{A}=id_{FA}.\]
  Y por último que sucede con la propiedad universal en este caso, veamos, sea \[f\colon A\rightarrow GFB\text{ y } g\colon B\rightarrow GFC\] entonces 
  \[\begin{tikzcd}[column sep=large, row sep=large]
    A \arrow[r, "f^+"] \arrow[d, "\eta_A"'] & GFB \arrow[d, "Gg^+"] \\
    GFA \arrow[r, "Gf^+"'] & GFC
    \end{tikzcd}\]


Que por unicidad debemos de tener \[(Gg^{+}\circ f)^{+}=g^{+}\circ f^{+}\] y así \[G(Gg^{+}\circ f)^{+}=Gg^{+}\circ Gf^{+}.\] Entonces requerimos para nuestro operador $(\_)^{*}$ lo análogo, es decir, \[(g^{*}\circ f)^{*}=g^{*}\circ f^{*}\]
con esto cocinado tenemos nuestra definición:

\begin{dfn}\label{Kleislitriple}
Una \emph{monada de Kleisli} en una categoría $\C$ consiste de los siguientes datos:\begin{itemize}
\item[(i)] Un operador $T$ de objetos de $\C$ en objetos de $\C$.
\item[(ii)] Para cada $A$ una flecha distinguida $\eta_{A}$.
\item[(iii)] Para todo $A,B\in\C$ una función $(\_)^{*}\colon\C(A,TB)\rightarrow\C(TA,TB)$ sujeto a las siguientes ecuaciones \[f^{*}\circ\eta_{A}=f\;\;\; \eta_{A}^{*}=id_{TA}\;\;\;\; (g^{*}\circ f)^{*}=g^{*}\circ f^{*}\]  

\end{itemize}
Y así hablaremos de la monada de Kleisli como la terna $(T,\eta,(\_)^{*})$.
\end{dfn}
Y claro tenemos la proposición correspondiente con la definición de adjunción mediante flechas universales.
\begin{prop}\label{Kleisiarrow}
Si $(F,G,\eta,(\_)^{+})$ describe una adjunción en $\C$ con $F$ una asignación de objetos de $\C$ en objetos de $\D$ entonces $(GF,\eta,G(\_)^{+})$ define una monada de Kleisli en $\C$. 
\end{prop}    
El lector perspicaz notara que debe de haber una relación entre la definición \ref{mona} y la definición \ref{Kleislitriple}, pues lo esperado sucede:

\begin{prop}\label{expected}
Las nociones de monada y monada de Kleisli son equivalentes.
\end{prop}    

\begin{proof}
Empecemos con una monada de Kleisli digamos $(T,\eta,(\_)^{*})$ en $\EuScript{C}$, entonces veamos como hacer que $T$ sea un funtor de carne y hueso, tomemos $f\colon A\rightarrow B$ en $\EuScript{C}$
y sea su correspondiente $Tf\colon Ta\rightarrow TB$ dado: \[Tf=(\eta_{B}\circ f)^{*}\] esto lo hace funtorial ya que para todo $A\in\EuScript{C}$ se tiene: 
\[Tid_{A}=(\eta_{A}\circ id_{A})^{*}=\eta_{A}^{*}=id_{TA}.\]

y para flechas \[\begin{tikzcd}[column sep=large, row sep=large]
    A \arrow[r, "f"]  & B \arrow[r, "g"] & C
    \end{tikzcd}\]

tenemos que: 

\[
    \begin{aligned}
        Tg\circ Tf
        &= (\eta_{C}\circ g)^{*}\circ (\eta_{B}\circ f)^{*} \\
        &= ((\eta_{C}\circ g)^{*}\circ\eta_{B}\circ f)^{*}\\
        &= (\eta_{C}\circ g\circ f)^{*} \\
        &=T(g\circ f),
    \end{aligned}\]
Ahora en efecto $\eta$ es una transformación natural, es decir para cualquier flecha $f\colon A\rightarrow B$ en $\EuScript{C}$ tenemos:

\[\eta_{B}\circ f=Tf\circ\eta_{A}.\]

Calculamos: \[Tf\circ\eta_{A}=(\eta_{B}\circ f)^{*}\circ\eta_{A}=\eta_{B}\circ f.\]
es decir es natural.

Ahora nos falta calcular el otro dato de monada, la multiplicación, $\mu\colon T^{2}\rightarrow T$ esta la damos como: \[\mu_{A}= id_{TA}^{*}\colon T^{2}A\rightarrow TA.\]
Y esto en efecto es natural, es decir para cualquier $f\colon A\rightarrow B$ tenemos que: 


\[
    \begin{aligned}
        \mu_{B}\circ T^{2}f
        &= id_{TB}^{*}\circ (\eta_{TB}\circ Tf)^{*} \\
        &= (id_{TB}^{*}\circ\eta_{TB}\circ Tf)^{*}\\
        &= (id_{TB}\circ Tf)^{*} \\
        &=(Tf)^{*}\\
        &=(\eta_{B}\circ f)^{**}\\
        &=((\eta_{B}\circ f)^{*}\circ id_{TA})^{*}\\
        &=(\eta_{B}\circ f)^{*}\circ id_{TA}^{*}\\
        &= Tf\circ\mu_{A},
    \end{aligned}\]
Solo resta calcular las tres ecuaciones, entonces calculamos: \[\mu_{A}\circ\eta_{TA}=id^{*}_{TA}=id_{TA}.\] y 
\[\mu_{A}\circ T\eta_{A}=id^{*}_{TA}\circ(\eta_{TA}\circ\eta_{A})^{*}=(id^{*}_{TA}\circ\eta_{TA}\circ\eta_{A})^{*}=(id_{TA}\circ\eta_{A})^{*}=\eta_{A}^{*}=id_{TA}.\]
Finalmente el cuadrado adecuado conmuta:
\[
    \begin{aligned}
        \mu_{A}\circ T\mu_{A}
        &= id^{*}_{TA}\circ(\eta_{TA}\circ id^{*}_{TA})^{*} \\
        &= (id^{*}_{TA}\circ\eta_{TA}\circ id^{*}_{TA})^{*}\\
        &= (id_{TA}\circ id_{TA}^{*})^{*} \\
        &=id_{TA}^{**}\\
        &=(id^{*}_{TA}\circ id^{*}_{T^{2}A})^{*}\\
        &=id^{*}_{TA}\circ id_{T^{2}A}^{*}\\
        &=\mu_{A}\circ\mu_{TA},
        \end{aligned}\]

Y así $(T,\eta,\mu)$ define una monada en $\EuScript{C}$.

Recíprocamente, si comenzamos con los datos de una monada $(T,\eta,\mu)$ en $\EuScript{C}$., claro casi tautológicamente tenemos que $T$ y $\eta$ satisfacen las condiciones de la definición
\ref{Kleislitriple}, resta ver que el operador de conjuntos de funciones $(\_)^{*}$ cumple lo requerido, sea entonces $f\colon A\rightarrow TB$ en $\EuScript{C}$ pongamos $f^{*}$ que sea la flecha

\[\begin{tikzcd}[column sep=large, row sep=large]
    TA \arrow[r, "Tf"]  & T^{2}B \arrow[r, "\mu_{B}"] & TB
    \end{tikzcd}\]
 resta ver que se cumplen las ecuaciones de Kleisli, para esto tomemos cualquier $f\colon A\rightarrow TB$, entonces \[f^{*}\circ\eta_{A}=\mu_{B}\circ Tf\circ\eta_{A}=\mu_{B}\circ\eta_{TB} f=id_{TB}\circ f=f.\]

 de aquí \[\eta_{A}^{*}=\mu_{A}\circ T\eta_{A}=id_{TA}.\]
como se pedía.
Ahora para $f$ como antes tenemos con $g\in\EuScript{C}(A,TB)$, entonces calculamos usando la funtorialidad de $T$: 


\[
    \begin{aligned}
        g^{*}\circ f^{*}
        &= \mu_{C}\circ Tg\circ\mu_{B}\circ Tf \\
        &= \mu_{C}\circ\mu_{TC}\circ T^{2}g\circ Tf\\
        &= \mu_{C}\circ T\mu_{C}\circ T^{*}g\circ Tf \\
        &=\mu_{C}\circ T(\mu_{C}\circ Tg\circ f)\\
        &=(g^{*}\circ f)^{*},
        \end{aligned}\]
y así los datos $(T,\eta,(\_)^{*})$ define una monada de Kleisli.

Ahora en efecto si denotamos por $\mathcal{M}(\EuScript{C})=\{\text{ monadas en }\EuScript{C}\}$ y por $\mathcal{K}\mathcal{M}(\EuScript{C})=\{\text{ monadas de Kleisli en }\EuScript{C}\}$
Tenemos entonces dos asignaciones \[\mathcal{M}(\EuScript{C})\rightarrow \mathcal{K}\mathcal{M}(\EuScript{C})\] y \[\mathcal{K}\mathcal{M}(\EuScript{C})\rightarrow \mathcal{M}(\EuScript{C})\] dadas por 
$(T,\eta,(\_)^{*})\leadsto (T,\eta,\mu)$ y viceversa forman una biyección, en efecto si empezamos con una de Kleisli $(T,\eta,(\_)^{*})\leadsto (T,\eta,\mu)$ y su monada construida como antes,  entonces para cualquier flecha $f\colon A\rightarrow TB$ y entonces \[\mu_{B}\circ Tf=id^{*}_{TA}\circ(\eta_{B}\circ f)^{*}=(id^{*}_{TA}\circ\eta_{B}\circ f)^{*}=(id_{TA}\circ f)^{*}=f^{*}.\]
y así obtenemos la monada de Kleisli original. 
Y viceversa si empezamos con una monada usual $(T,\eta,\mu)$ con su Kleisli y de allí obtenemos la monada dada por esta entonces con los cálculos que tenemos hechos \[id^{*}_{TA}=\mu_{TA}\circ Tid_{TA}=\mu_{TA}\circ id_{T^{2}A}=\mu_{TA}.\]
y así recuperamos la monada con la que empezamos.

\end{proof}   

Esto esta bien, ya tener dos opciones para cocinar la factorización que estamos buscando, pero aún así necesitamos una solución \emph{universal}.

\subsection{Solución de Kleisli}\label{SOLKLEI}

La solución que describiremos aquí es debida a Kleisli en su celebra artículo Every standard construction is induced by a pair of adjoint functors (habrá que ponerlo en la biblio)
Esta construcción empieza con una adjunción $F\dashv G$ donde ya esta tenemos a la categoría $\EuScript{D}$ que soluciona el problema de factorización, es decir tenemos:

\[\EuScript{D}(FA,FB)\cong\EuScript{C}(A,GFB)\cong\EuScript{C}(A,TB).\]
 \begin{dfn}\label{cate de Kleisli}
La \emph{categoría de Kleisli} $\EuScript{C}_{T}$ de una monada de Kleisli $(T,\eta,(\_)^{*})$ en una categoría $\EuScript{C}$ consiste de los siguientes datos:
\begin{itemize}
\item[(i)] Los objetos son los mismos que los de $\EuScript{C}$.
\item[(ii)] Las flechas $A\rightarrow B$ están dadas por $A\rightarrow TB$;
\item[(iii)] Las identidades para cada $A$ están dadas por $\eta_{A}$.
\item[(iv)] La composición esta dada, $f\colon A\rightarrow B$  y $g\colon B\rightarrow C$ en $\EuScript{C}_{T}$ : \[\begin{tikzcd}[column sep=large, row sep=large]
    A \arrow[r, "f"]  & TB \arrow[r, "g^{*}"] & TC
    \end{tikzcd}\]
\end{itemize}
Formalmente deberíamos distinguir entre las composición en $\EuScript{C}$ y la de $\EuScript{C}_{T}$ para esto digamos que el símbolo $g\bullet f$ representa la composición en $\EuScript{C}_{T}$.

También nos falta verificar que en efecto los datos de \ref{SOLKLEI} constituyen una categoría, para esto si $f\colon A\rightarrow B$ en $\EuScript{C}_{T}$ tenemos que la composición con la identidad esta dada como \[f^{*}\circ\eta_{A}=f.\]

Y pos-componiendo (si es que dicha palabra existe) con la identidad $B$ tenemos: \[\eta^{*}_{B}\circ f=id_{B}\circ f=f.\]
Por otro lado sea $f$ como antes y tomemos dos flechas que se pueden componer en $\EuScript{C}_{T}$ digamos $g\colon B\rightarrow C$ y $h\colon C\rightarrow D$, entonces \[h\bullet (g\bullet f)=h^{*}\circ(g^{*}\circ f)=(h^{*}\circ g^{*})\circ f=(h^{*}\circ g)^{*}\circ f=(h\bullet g)\bullet f.\]
\end{dfn}

Y por lo tanto:

\begin{prop}\label{catkleisliesunacat}
Con la notación de la definición \ref{SOLKLEI} se tiene que $\EuScript{C}_{T}$ es una categoría.
\end{prop}

Existe una comparación directa entre objetos de $\EuScript{C}$ y $\EuScript{C}_{T}$, $F_{T}\colon Obj\EuScript{C}\rightarrow Obj\EuScript{C}_{T}$ esta asignación no hace nada perse, también podemos comparar con un funtor
$G_{T}\colon\EuScript{C}_{T}\rightarrow\EuScript{C}$ como sigue, en cada objeto $C\in\EuScript{C}_{T}$ entonces $G_{T}(C)=TC $ y en cada morfismo $f\colon A\rightarrow B$ en $\EuScript{C}_{T}$
lo mandamos a $f^{*}\colon TA\rightarrow TB$ en $\EuScript{C}$, esto en efecto define un funtor ya que en identidades tenemos que $\eta_{A}$ lo manda a $\eta_{A}^{*}=id_{TA}$, si $f\colon A\rightarrow B$ y $g\colon b\rightarrow C$ en $\EuScript{C}_{T}$.
entonces: \[G_{Tg}\circ G_{Tf}=g^{*}\circ f^{*}=(g^{*}\circ f)^{*}=G_{T}(g\bullet f).\] y así $G_{T}$ preserva las composiciones y por lo tanto es un funtor.
Ahora claramente tenemos: \[T=G_{T}F_{T}\] en objetos, queremos estatar una propiedad universal para esta construcción.
Para este fin, tomemos un $f\colon A\rightarrow G_{T}B$ con $A\in\EuScript{C}$ y $B\in\EuScript{C}_{T}$. necesitamos construir una flecha única $f^{+}\colon F_{T}A\rightarrow B$ en $\EuScript{C}_{T}$ tal que:
\[G_{T}f^{+}\circ\eta_{A}=f.\] esto es justamente lo que hace nuestra construcción hace que $f\colon A\rightarrow G_{T}B=TB$ es un morfismo en la categoría $\EuScript{D}$, y así,
$G_{T}f^{+}=f^{*}$ y calculando \[G_{T}f^{+}\circ\eta_{A}=f^{*}\circ\eta_{A}=f\] como se requería. Ahora para la unicidad, si $g\colon F_{T}A\rightarrow B$ en $\EuScript{C}_{T}$ 
Es un ejercicio que el lector calcule la categoría de Kleisli de los ejemplos que hemos dejado al principio de esta sección.

Vale la pena en este punto (supongo) poner un poco de contexto aún mas abstracto sin-sentido, sea $(T,\eta,\mu)$ una monada en una categoría $\mathscr{C}$.
denotemos por $\mathrm{Ad}\mathscr{C}$ a la categoría de todas las adjunciones
\[\begin{tikzcd}
	{(F,G,\eta,\epsilon)\colon\mathscr{C}} & {\mathscr{A}}
	\arrow[harpoon, from=1-1, to=1-2]
\end{tikzcd}\]
tales que definen a $(T,\eta,\mu)$ y las flechas de $\mathrm{Ad}\mathscr{C}$ están dadas como sigue, si 
\[\begin{tikzcd}
	{(F,G,\eta,\epsilon)\colon\mathscr{C}} & {\mathscr{A}}
	\arrow[harpoon, from=1-1, to=1-2]
\end{tikzcd}\]
y
\[\begin{tikzcd}
	{(F',G',\eta',\epsilon')\colon\mathscr{C}'} & {\mathscr{A}'}
	\arrow[harpoon, from=1-1, to=1-2]
\end{tikzcd}\]
son adjunciones sobre $\mathscr{C}$, un morfismo entre estas \[(F,G,\eta,\epsilon)\rightarrow(F',G',\eta',\epsilon')\] 
es un par de funtores $K\colon\mathscr{A}\rightarrow\mathscr{A}'$ y $L\colon\mathscr{C}\rightarrow\mathscr{C}$ tal que
los siguientes diagramas conmutan:
\[
\begin{tikzcd}[column sep=large, row sep=large]
A \arrow[r, "G"] \arrow[d, "K"'] & X \arrow[d, "L"] \\
A' \arrow[r, "G'"'] & X'
\end{tikzcd}
\]
\[
\begin{tikzcd}[column sep=large, row sep=large]
A \arrow[r, "F"] \arrow[d, "K"'] & X \arrow[d, "L"] \\
A' \arrow[r, "F'"'] & X'
\end{tikzcd}\]

Hemos así probado dos hechos destacables:

\begin{thm}(Comparación de Kleisli)\label{kleisli es final}
Dada una adjunción $(F,G,\eta,\epsilon)$ en una categoría $\EuScript{C}$, y sea $T=(GF, \eta,G\epsilon F)$ la monada que esta define,
entonces existe un único funtor $L\colon\mathscr{C}_{T}\rightarrow\EuScript{A}$ con $GL=G_{T}$ y $LF_{T}=F$.
\end{thm}


\begin{thm}\label{kleisli es final2}
Dada una monada $(T,\eta,\mu)$ en una categoría $\EuScript{C}$, 
\end{thm}


En la práctica ya que estamos hablando de objetos en la categoría de adjunciones de una categoría agradable, podríamos preguntarnos si tal categoría tiene objeto final.

\subsection{La solución de Eilenberg y Moore}\label{EMSOL}

En efecto construyamos el objeto final en la categoría de posibles adjunciones que factoricen una monada dada.


\section{La adjunción entre $\Frm$ y $\Top$}

Esta sección está dedica a desglosar varias de las cosas que presentamos en estas notas aterrizándolas en dos categorías especificas ($\Frm$ y $\Top$).

Uno de los primeros ejemplos de marcos, y de hecho, la
motivación para la definición, fue que los abiertos de un espacio
topológico $S$ forman un marco $\mathcal{O}S$.
Aquí desarrollaremos más a fondo la relación entre marcos
y espacios topológicos. No es complicado verificar que la asignación $S\mapsto\mathcal{O}S$ es
un funtor contra variante $\mathcal{O}\colon\Top\to\Frm$. La asignación en objetos está dado por la topología del espacio y la asignación en flechas a través de la preimagen de las funciones continuas (estas resultan ser un morfismo de marcos por las propiedades de la preimagen). Para detalles más específicos consultar \cite{P.T.} o \cite{J.P.}.\\

Ahora la tarea es construir un funtor de regreso $\pt\colon\Frm\to\Top$
y probaremos que ambos funtores forman una adjunción.
\[
  \begin{tikzcd}
    \Top \ar[d,shift right=2,"\Cal O"'{name=L}]
    \\
    \Frm^\op \ar[u,shift right=2,"\pt"'{name=R}]
    \adj{L}{R}
  \end{tikzcd}
\]

Antes de continuar, vale la pena mencionar que todos nuestros espacios
(de carne y hueso) serán al menos $T_{0}$. Recordemos que un espacio
$T_0$ es un espacio donde cada par de puntos se pueden separar por al
menos un abierto. Más aún:
\begin{thm}\label{tcero}
    La categoría $\Top_0$ de espacios topológicos
    $T_0$  es \emph{reflexiva} en $\Top$.
    Es decir, el funtor de inclusión $\Top_0\to\Top$
    tiene un adjunto izquierdo.
\end{thm}
\begin{proof}
Es un ejercicio sencillo.
\end{proof}
La razón de esta decisión es que, en un espacio que no es $T_0$, hay
pares de puntos que tienen exactamente las mismas vecindades de
abiertos, así que no tienen mucha esperanza de ser caracterizados por
sus marcos de abiertos.
Otro ejercicio sencillo que ayuda a familiarizarse con los espacios
$T_0$ es el siguiente. En cualquier espacio topológico, los puntos
tienen un preorden (es decir, una relación reflexiva y transitiva)
dado por
  \[q\sqsubseteq p\iff \overline{q}\subseteq \overline{p}\]
  Este se llama \emph{preorden de especialización}.

\subsection{El espacio de puntos}
Dado un espacio con un punto $\{*\}$, el conjunto de
puntos de $S$ está en biyección con las funciones continuas
$\{*\}\to S$:
\[
  S \simeq \Top(\{*\},S)
\]
donde cada punto $s\in S$ está asociado a la función $*\mapsto
s$.
Así, si $2$ es el marco de dos elementos $2=\{0<1\}$,
cada de estas funciones $s:\{*\}\to S$ induce un morfismo de
marcos $\chi_s:\mathcal{O}S\to\mathcal{O}\{*\}\simeq 2$ dado como
\[
  \chi_s(u) =
  \begin{cases}
    1, & s\in u \\
    0, & s\notin u
  \end{cases}
.\]
Así, para cada marco $A$, tiene sentido definir los puntos de $A$
como morfismos $\Frm(A,2)$.
En efecto, más adelante consideraremos esta construcción.
Sin embargo, primero consideraremos otra construcción equivalente:
representaremos cada morfismo $\chi:A\to 2$ con un elemento de
$A$ de manera canónica: el elemento
\[
    p = \bigvee\{x\in A\mid \chi(x)=0\}
\]
es el único elemento de $A$ que cumple
\[
    x\leq p \ssi \chi(x) = 0
.\]
En particular, dado que $\chi(1)=1$, tenemos $p\neq 1$.
Por otro lado, para cualesquiera $x,y\in A$ con
$x\inf y\leq p$, tenemos
\[
    \chi(x)\inf\chi(y)=\chi(x\inf y)=0
,\]
así que $\chi(x)=0$ o bien $\chi(x)=0$, pues $\chi$ toma valores
en el marco $2$. es decir: $x\leq p$ o bien $y\leq p$.

\begin{dfn}
  Sea $A\in \Frm$. Un punto o elemento $\inf$-irreducible 
  de A es un elemento $p\in A$ con $p\neq 1$ tal que si 
  $x\inf y\leq p$, entonces $x\leq p$ o $y\leq p$. 
  Denotamos por $\pt A$ al conjunto de todos los puntos de $A$.
\end{dfn}

\begin{lem}
  Sea $A\in \Frm$.
  \begin{itemize}
      \item Cada máximo de $A$ es $\inf$-irreducible.
      \item Si A es booleano, entonces todo elemento $\inf-$irreducible de A es máximo.
      \item Si A es una cadena, entonces cada elemento propio de A es $\inf-$irreducible.
  \end{itemize}
\end{lem}
\begin{proof}\quad
  \begin{itemize}
      \item Sea $p\in A$ máximo, entonces $p<1$. Si $x\inf y\leq p$ y suponiendo que $x\not\leq p$, entonces $p<x\sup p$ y, por la maximalidad de $p$, tenemos que $p\sup x=1$. Similarmente, $y\not\leq p$ implica $p\sup y=1$. Si $x\not\leq p$ y $y\not\leq p$, se tiene que 
      \[p=p\sup (x\inf y)=(p\sup x)\inf(p\sup y)=1.\]
      Esto es una contradicción ya que $p<1$.
      \item Supongamos que $A$ es booleano. Sean $p\in\pt A$ y $x,y\in A$ con $p<x$ y $y=\neg x$. Tenemos que $x\inf y=0\leq p$, entonces $x\leq p$ ó $y\leq p$ ya que p es $\inf-$irreducible. Además $y\leq p<x$ puesto que $p<x$. En consecuencia, $x\sup y=1=x$, así, $p$ es máximo.
      \item Supongamos que $A$ es una cadena. Para cualesquiera $x,y\in A$, tenemos que $x\leq y$ ó $y\leq x$, es decir, $x\inf y\leq x$ ó $x\inf y\leq y$. Sea $p\in A$ con $p<1$. Si $x\inf y\leq p$, entonces $x\leq p$ ó $y\leq p$.
  \end{itemize}
\end{proof}

Sean $A\in \Frm$ y $a\in A$. Decimos que un punto $p\in \pt A$ está en $U_A(a)\subseteq \pt A$ si, y sólo si $a\not\leq p$.
\begin{lem}
  Sean $A\in \Frm$ y $a,b\in A$.
  \begin{itemize}
      \item $U_A(1)=\pt A$.
      \item $U_A(0)=\emptyset$.
      \item $U_A(a\inf b)=U_A(a)\cap U_A(b)$.
      \item $U_A(\Sup X)=\bigcup \{U_A(x)|x\in X\}$, $\forall X\subseteq A$.
  \end{itemize}
\end{lem}
\begin{proof}
  Sean $A\in \Frm$ y $a,b\in A$.
\begin{itemize}
\item Por definición $U_A(1)\subseteq \pt A$. Sea $p\in \pt A$, entonces $p\neq 1$. Además $1\not\leq p$, por lo que $p\in U_A(1)$. Así, $U_A(1)=\pt A$.
\item Supongamos que $U_A(0)\neq \emptyset$. Sea $p\in U_A(0)$. Por definición, $0\not\leq p$ pero $0\leq a, \forall a\in A$. Por lo tanto, $U_A(0)=\emptyset$.
\item Sea $p\in \pt A$. Tenemos que
\begin{align*}
p\in U_A(a\wedge b)&\iff a\wedge b\not\leq p\\
&\iff a\not\leq p\quad y\quad b\not\leq p\\
&\iff p\in U_A(a)\quad y\quad p\in U_A(b)\\
&\iff p\in U_A(a)\cap U_A(b).
\end{align*}
Por lo que $U_A(a\wedge b)=U_A(a)\cap U_A(b)$.
\item Sea $X\subseteq A$ y notemos que si $X=\emptyset$, entonces ocurre el segundo punto. En caso contrario,
\begin{align*}
p\in U_A(\bigvee X)&\Rightarrow \bigvee X\not\leq p\\
&\Rightarrow \textit{existe }x\in X\textit{ tal que }x\not\leq p\\
&\Rightarrow p\in U_A(x)\\
&\Rightarrow p\in \bigcup \{U_A(x)\mid x\in X\}.
\end{align*}
Además,
\begin{align*}
p\in \bigcup\{U_A(x)\mid x\in X\}&\Rightarrow p\in U_A(x)\textit{ para algún }x\in X\\
&\Rightarrow x\not \leq p\\
&\Rightarrow \bigvee X\not\leq p\\
&\Rightarrow p\in U_A(\bigvee X).
\end{align*}
Por lo tanto, $U_A(\Sup x)=\bigcup \{U_A(x)|x\in X\}$, $\forall X\subseteq A$.
\end{itemize}
\end{proof}
Se sigue que $U_A(A)=\{U_A(a)\mid a\in A\}$ es una topología en $\pt
A$. Al espacio topológico $(\pt A,U_A(A))$ lo llamamos
el \textit{espacio de puntos} de $A$.
Dado que la topología de $\pt A$ es $\Cal O\pt A=U_A(A)$,
se sigue que $U_A:A\to\Cal O\pt A$ es un morfismo suprayectivo de marcos, al cual llamamos la \textit{reflexión
espacial} de $A$. Si $U_A$ es inyectivo (y, por lo tanto, un
isomorfismo) decimos que el marco $A$ es \textit{espacial}.

\begin{obs}
  \leavevmode
  \begin{enumerate}
    \item (La reflexión espacial como un cociente)
      Como $U_A:A\to\Cal O\pt$ es suprayectivo, el marco $\Cal O\pt A$ 
      es el cociente de $A$ (por definición de cocientes en $\Frm$) bajo el núcleo de $U_A$: el núcleo
      $S\in NA$ dado como
      \[
        x\leq S(a) \iff U(x)\subseteq U(a)
      .\]
(los núcleos son cierto tipo de funciones definidas en los marcos y estos están en correspondencia biyectiva con los cocientes. Para un marco $A$, $NA$ denota el conjunto de todos sus núcleos).
    \item
      Además, por el lema adecuado, el
      núcleo $S$ de $U_A$ admite la descripción
      \[
        S(a)=\Inf \{p\in \pt A|a\leq p\}
      .\]
    \item
      Dados dos puntos $p,q\in \pt A$, se cumple
      \begin{align*}
        q\sqsubseteq p&\iff \overline{q}\subseteq \overline{p}\\
        &\iff (\forall x\in A)[q\in U(x)\Rightarrow p\in U(x)]\\
        &\iff (\forall x\in A)[x\leq p\Rightarrow x\leq q]\\
        &\iff p\leq q.
      \end{align*}
      Es decir, el preorden de especialización del espacio de puntos es
      el orden opuesto al orden heredado del marco:
      \[
        (\pt A,\sqsubseteq) = (\pt A,\leq)^\op.
      \]
      En particular, ya que su preorden de especialización es un orden
      parcial, esto prueba que el espacio de puntos es $T_0$.
  \end{enumerate}
\end{obs}

\subsection{Funtorialidad y naturalidad}
Queremos ver que la asignación $A\mapsto \pt A$ es un funtor y que
la reflexión espacial $U_A:A\to\Cal O\pt A$
es una transformación natural
\[U_\bullet\colon\id_{\Frm}\to\Cal O\pt(\_).\]

Lo primero es verificar que, dado un morfismo de marcos $f\colon A\to B$,
obtenemos una función continua $\pt f:\pt B\to\pt A$ entre los
espacios de puntos. De hecho, $\pt f$ será la restricción del adjunto
derecho $f_*\colon B\to A$ a los puntos de $B$, pero hay que verificar que
$f_*$ manda puntos a puntos.

Si $p$ es un punto de $\pt B$, veamos que $f_*p$ es un punto de $A$.
Primero, $f_*p$ no puede ser $1$, ya que en ese caso $1\leq f_\ast(p)$
implicaría $f(1)\leq p$ por la adjunción, pero esto es imposible ya
que $p\neq 1$.
Ahora veamos que $f_*p$ es $\inf$-irreducible. Si $x,y\in A$ son tales
que $x\inf y\leq f_\ast(p)$, por adjunción tenemos $f(x)\inf f(y)\leq p$.
En consecuencia, $f(x)\leq p$ ó $f(y)\leq p$, i.e., $x\leq f_\ast (p)$
o $y\leq f_\ast (p)$. Por lo tanto, $f_\ast (p)\in \pt A$.

En resumen, dado un morfismo de marcos $f\colon A\to B$, obtenemos una
función $\pt f \colon \pt B\to \pt A$ dada por la restricción de
$f_*:B\to A$.

Observemos que, para todo $p\in \Cal \pt B$, tenemos
\begin{align*}
    p\in (\pt f)^{-1} \left(U_A(a)\right)&\iff f_\ast (p)\in U_A(a)\\
    &\iff a\not\leq f_\ast (p)\\
    &\iff f(a)\not\leq p\\
    &\iff p\in U_B\left(f(a)\right).
\end{align*}
Por lo tanto $\pt f\colon \pt B\to \pt A$ es continua.
Es fácil ver que, dados morfismos $k:C\to B$ y $h:B\to A$,
se satisface $(hk)_*=k_*h_*$. Además, el adjunto derecho de $\id:A\to A$ también es la identidad
de $A$.
De estas observaciones se sigue que la asignación $\pt$
es un funtor (contravariante) $\pt\colon \Frm\to\Top$.

Además, en el párrafo anterior probamos que
\[
    \Cal O(\pt f)(U_A(a)) = U_B(f(a))
\]
para todo $a\in A$.
Es decir: el diagrama
\[
    \begin{tikzcd}
        A \ar[r,"f"] \ar[d,"U_A"'] & B \ar[d,"U_B"] \\
        \Cal O\pt A \ar[r,"\Cal O\pt f"'] & \Cal O\pt B
    \end{tikzcd}
\]
conmuta, así $U_\bullet=(U_A\mid A\in\Frm)$
es una transformación natural $U_\bullet:\id_\Frm\to\Cal O\pt$.

\subsection{El espacio de puntos del marco de abiertos}

¿Qué tanta información acerca de un espacio topológico se
puede recuperar a través de su marco de abiertos?
Comenzaremos preguntándonos
cómo se relacionan los puntos de un espacio $S$ con los puntos de
$\pt\Cal OS$.
Como dijimos al principio, un punto $s\in S$ se puede ver como una
función continua $\{s\}\to S$, la cual induce un morfismo
$\chi_s:\Cal O S \to 2$ como
\begin{equation}
  \chi_s(u) =
  \begin{cases}
    1, & s\in u  \\
    0, & s\not\in u.
  \end{cases}
\end{equation}
Por lo tanto, el $\inf$-irreducible que le corresponde a $s$ es
\begin{equation}
  \Phi_S(s) = \Sup\{u\in\Cal OS\mid \chi_s(u)=0\} \in \pt\Cal O S
  .\end{equation}
Esto nos da una función $\Phi_S:S\to\pt\Cal OS$. Esta descripción
se puede simplificar. Notemos que, dados $s\in S$ y $u\in\Cal OS$, tenemos
\[
  \chi_s(u) = 0
  \iff
  s\not\in u
  \iff
  s\in u'
  \iff
  \ol s \subseteq u'
  \iff
  u \subseteq {\ol s}'.
\]
Luego,
\begin{equation}
  \Phi_S(s)
  =
  \Sup\{u\in\Cal OS\mid u\subseteq {\ol s}'\}
  =
  {\ol s}'
  \in
  \pt\Cal OS
.\end{equation}
Ahora, recordemos que un abierto de $\pt\Cal OS$ es de la forma
\begin{align*}
  U_{\Cal OS}(u)
  &= \{p\in\pt\Cal OS \mid u\not\subseteq p\}
.\end{align*}
Tenemos
\begin{align*}
    s\in (\Phi_S)^{-1}(U_{\Cal OS}(u))
    &\iff \Phi_S(s) \in U_{\Cal OS}(u) \\
    &\iff {\ol s}' \in U_{\Cal OS}(u) \\
    &\iff u \nsubseteq {\ol s}' \\
    &\iff s\in u.
\end{align*}
Es decir,
\[
  (\Phi_S)^{-1}(U_{\Cal OS}(u)) = u
.\]
En particular, $(\Phi_S)^{-1}$ manda abiertos de $\pt\Cal OS$ en
abiertos de $S$, así que la función $\Phi_S:S\to\pt\Cal OS$ es continua.
Por último, observemos que, dada una función continua $\psi:S\to T$,
las funciones $\Phi_S:S\to\pt\Cal OS$ hacen conmutar el diagrama
\[
    \begin{tikzcd}
        S \ar[r,"\psi"] \ar[d,"\Phi_S"'] & T \ar[d,"\Phi_T"] \\
        \pt\Cal OS \ar[r,"\pt\Cal O\psi"'] & \pt\Cal OT
    \end{tikzcd}
\]
En efecto, para todo $v\in\Cal OT$, tenemos
\begin{align*}
    v\subseteq (\pt\Cal O\psi)(\Phi_S(s))
    &\iff v\subseteq (\Cal O\psi)_*(\Phi_S(s)) \\
    &\iff (\Cal O\psi)(v) \subseteq \Phi_S(s) \\
    &\iff \psi^{-1}(v) \subseteq \ol{s}' \\
    &\iff s\nin \psi^{-1}(v) \\
    &\iff \psi(s)\nin v \\
    &\iff v\subseteq \ol{\psi(s)}' \\
    &\iff v\subseteq \Phi_T(\psi(s)),
\end{align*}
por lo cual $(\pt\Cal O\psi)(\Phi_S(s))=\Phi_T(\psi(s))$.
Luego, la familia de funciones continuas
\[
    \Phi_\bullet=(\Phi_S:S\to\pt\Cal OS\mid S\in \Top)
\]
es una transformación natural
\[
    \Phi_\bullet : \id_\Top\to\pt\Cal O
.\]

\subsection{La adjunción}\label{ssec:adjuncion}

En la primera parte, vimos que todo morfismo de marcos
$f:A\to B$ induce una función continua $\pt f:\pt B\to\pt A$ dada
como la restricción del adjunto derecho $f_*:B\to A$ de $f$ y
probamos que esta asignación es un funtor $\pt:\Frm\to\Top$.
Ahora veremos que $\pt$ y el funtor de abiertos
$\Cal O:\Top\to\Frm$ son las mitades de una adjunción contravariante
entre $\Top$ y $\Frm$. En particular, construiremos un isomorfismo
\begin{equation}\label{eqn:adj_frm_top}
    \Frm(A,\Cal OS) \simeq \Top(S,\pt A)
\end{equation}
natural en $A$ y en $S$.

Cuando aprendimos sobre adjunciones,
vimos el caso covariante, en el cual el isomorfismo de
adjunción es equivalente a la existencia de dos transformaciones
naturales que satisfacen las identidades triangulares.

Ahora veremos que, en el caso contravariante,
tenemos el resultado análogo:
las identidades triangulares adecuadas
implican el isomorfismo natural.

Recordemos que las transformaciones naturales
$U_\bullet:\id_\Frm\to\Cal O\pt$ y
$\Phi_\bullet:\id_\Top\to\pt\Cal O$
tienen componentes dadas como
\begin{align*}
    U_A:A&\to \Cal O\pt A \\
    a &\mapsto U_A(a) = \{p\in \pt A \mid a\nleq p\}, \\
    \Phi_S:S&\to \pt\Cal O S \\
    s &\mapsto \Phi_S(s)=\ol{s}'.
\end{align*}
Primero veremos que se cumplen las identidades triangulares
\[
    \begin{tikzcd}[row sep=15mm]
        & \Cal OS \ar[d,"U_{\Cal OS}"] \ar[dl,"\id_{\Cal OS}"']
        \\
        \Cal OS
        & \Cal O\pt\Cal OS \ar[l,"\Cal O\Phi_S"]
    \end{tikzcd}
    \hspace{10mm}
    \begin{tikzcd}[row sep=15mm]
        & \pt A \ar[d,"\Phi_{\pt A}"] \ar[dl,"\id_{\pt A}"']
        \\
        \pt A
        & \pt \Cal O\pt A \ar[l,"\pt U_A"]
    \end{tikzcd}
\]
En efecto, usando las equivalencias
\begin{align*}
    u\subseteq \Phi_S(s) &\ssi s\nin u, \\
    x\in U_A(a) &\ssi a\nleq x,
\end{align*}
tenemos
\begin{align*}
    x\in (\Cal O\Phi_S)(U_{\Cal OS}(u))
    &\iff \Phi_S(x) \in U_{\Cal OS}(u) \\
    &\iff u\nleq \Phi_S(x) \\
    &\iff x\in u,
    \\
    a\leq (\pt U_A)(\Phi_{\pt A}(x))
    &\iff U_A(a) \leq \Phi_{\pt A}(x) \\
    &\iff x\nin U_A(a) \\
    &\iff a\leq x.
\end{align*}
Es decir, $(\Cal O\Phi_S)(U_{\Cal OS}(u))=u$
y $(\pt U_A)(\Phi_{\pt A}(x))=x$, como se quería.

Ahora, afirmamos que las funciones
\begin{align*}
    \Frm(A,\Cal OS) &\to \Top(S,\pt A) \\
    f &\mapsto \bar f = (\pt f)\Phi_S,
    \\
    \Frm(A,\Cal OS) &\leftarrow \Top(S,\pt A) \\
    (\Cal O\phi)U_A = \phi &\mapsto \bar\phi.
\end{align*}
conforman una biyección. En efecto,
la naturalidad de $\Phi_\bullet$, $U_\bullet$ y las identidades
triangulares implican la conmutatividad de los diagramas
\[
    \begin{tikzcd}[row sep=15mm]
        & \Cal OS \ar[d,"U_{\Cal OS}"] \ar[dl,"\id_{\Cal OS}"']
        & A \ar[l,"f"'] \ar[d,"U_A"]
        \\
        \Cal OS
        & \Cal O\pt\Cal OS \ar[l,"\Cal O\Phi_S"]
        & \Cal O\pt A \ar[l,"\Cal O\pt f"]
    \end{tikzcd}
    \hspace{10mm}
    \begin{tikzcd}[row sep=15mm]
        & \pt A \ar[d,"\Phi_{\pt A}"] \ar[dl,"\id_{\pt A}"']
        & S \ar[l,"\phi"'] \ar[d,"\Phi_S"]
        \\
        \pt A
        & \pt \Cal O\pt A \ar[l,"\pt U_A"]
        & \pt \Cal OS \ar[l,"\pt\Cal O\phi"]
    \end{tikzcd}
\]
por lo cual tenemos
\[
    \begin{aligned}
        \bar{\bar f}
        &= \ol{(\pt f)\Phi_S} \\
        &= \Cal O((\pt f)\Phi_S)U_A \\
        &= (\Cal O\Phi_S)(\Cal O\pt f)U_A \\
        &= f,
    \end{aligned}
    \hspace{20mm}
    \begin{aligned}
        \bar{\bar\phi}
        &= \ol{(\Cal O\phi)U_A} \\
        &= \pt((\Cal O\phi)U_A)\Phi_S \\
        &= (\pt U_A)(\pt\Cal O\phi)\Phi_S \\
        &= \phi.
    \end{aligned}
\]
Esto nos da la biyección mencionada al inicio de esta subsección.
De manera explícita, la biyección está dada como
$\Frm(A,\Cal OS)\ni f\leftrightarrow \phi\in \Top(S,\pt A)$, donde
\[
    s\in f(a) \ssi a\nleq \phi(s)
\]
para cualesquiera $s\in S$, $a\in A$, puesto que
\begin{align*}
    s\in f(a)
    &\iff f(a)\nleq \Phi_S(s) \\
    &\iff a \nleq (\pt f)(\Phi_S(s))=\bar f(s),
    \\
    a\nleq \phi(s)
    &\iff \phi(s) \in U_A(a) \\
    &\iff s \in (\Cal O\phi)(U_A(a)) = \bar\phi(a).
\end{align*}
Finalmente, veamos que la biyección es natural en $A$ y en $S$.
Dado un morfismo de marcos $g:A\to B$, el diagrama
\[
    \begin{tikzcd}
        \Frm(B,\Cal OS) \ar[d,"{-}\circ g"'] \ar[r,"f\mapsto\bar f"']
        & \Top(S,\pt B)
        \ar[d,"\pt g\circ{-}"]
        \\
        \Frm(A,\Cal OS) \ar[r,"h\mapsto\bar h"']
        & \Top(S,\pt A)
    \end{tikzcd}
\]
es conmutativo:
\begin{align*}
    \ol{fg}
    &= \pt(f g)\Phi_S \\
    &= (\pt g)(\pt f)\Phi_S \\
    &= (\pt g)\bar f.
\end{align*}
Similarmente, dada una función continua $\psi:S\to T$,
el diagrama
\[
    \begin{tikzcd}
        \Frm(A,\Cal OT)
        \ar[d,"\Cal O\psi\circ{-}"']
        & \Top(T,\pt A) \ar[l,"\bar \phi\mapsfrom \phi"']
        \ar[d,"{-}\circ \psi"]
        \\
        \Frm(A,\Cal OS)
        & \Top(S,\pt A) \ar[l,"\bar \xi\mapsfrom \xi"']
    \end{tikzcd}
\]
es conmutativo:
\begin{align*}
    \ol{\phi\psi}
    &= \Cal O(\phi\psi)U_A \\
    &= (\Cal O\psi)(\Cal O\phi)U_A \\
    &= (\Cal O\psi)\bar\phi.
\end{align*}

\subsection{La propiedad universal de las reflexiones}

Para esta subsección veremos que las biyecciones dadas por la adjunción revelan las propiedades universales del espacio de puntos y el marco de abiertos.
La biyección
\begin{align*}
    \Frm(A,\Cal OS) &\simeq \Top(S,\pt A) \\
    f &\mapsto \bar f = (\pt f)\Phi_S \\
    (\Cal O\phi)U_A = \bar\phi &\mapsfrom \phi.
\end{align*}
se puede leer como sigue:
dado un morfismo $f:A\to\Cal OS$, existe una única función continua
$\phi:S\to\pt A$ tal que el diagrama
\[
    \begin{tikzcd}
        A \ar[r,"f"] \ar[d,"U_A"'] & \Cal OS \\
        \Cal O\pt A \ar[ur,"\Cal O\phi"']
    \end{tikzcd}
\]
conmuta.
Similarmente, dada una función continua $\phi:S\to\pt A$, existe
un único morfismo $f:A\to\Cal OS$ tal que el diagrama
\[
    \begin{tikzcd}
        S \ar[r,"\phi"] \ar[d,"\Phi_S"'] & \pt A \\
        \pt\Cal OS \ar[ur,"\pt f"']
    \end{tikzcd}
\]
conmuta.

\subsection[Álgebras para la adjunción]{Álgebras para la adjunción $\spec\dashv \mathcal{O}$}

En esta sección usaremos lo aprendido en las sección \ref{SECMON} y la sección anterior, 

Primero trabajaremos con una extensión de la adjunción entre $\Frm$ y $\Top$, para esto consideremos la categoría de retículas distributivas 
\[\EuScript{D}Lt\] esta categoría tiene como subcategoría ala categoría de marcos, lo interesante es que podemos definir un funtor 
\[\spec\colon\EuScript{D}Lt\rightarrow\Top \] que a cada retícula distributiva le asocia su espacio espectral, es decir, su espectro (ideales primos), teniendo esto en mente, veremos que
tenemos una adjunción:

\[
  \begin{tikzcd}
    \Top \ar[d,shift right=2,"\Cal O"'{name=L}]
    \\
    \EuScript{D}Lt^\op \ar[u,shift right=2,"\spec"'{name=R}]
    \adj{L}{R}
  \end{tikzcd}
\]

que como se espera esta se restringe a una equivalencia, que se le conoce como la equivalencia de Stone entre retículas distributivas y espacios coherentes (espectrales).


\end{document}
%  COMPILE EL TRABAJO DOS VECES PARA QUITAR ALGUNOS WARNINGS 
